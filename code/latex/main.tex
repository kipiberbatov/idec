\documentclass[fleqn]{article}
\usepackage[margin = 0.6in, paper = a4paper]{geometry}
\usepackage{xcolor}
\usepackage[fleqn]{amsmath}
\usepackage{amssymb}
\usepackage{amsthm}
\usepackage{makecell}
\usepackage{bookmark}
\usepackage{hyperref}
\hypersetup{
  pdfauthor = {Kiprian Berbatov},
  pdftitle = {Intrinsic discrete exterior calculus},
  pdfsubject = {Discrete calculus},
  pdfkeywords = {discrete, microstructure, calculus, geometry},
  pdfcreator = {pdflatex},
  pdfproducer = {Latex2e with hyperref},
  colorlinks = true,
  linkcolor = blue,
  citecolor = green,
  urlcolor = cyan,
  bookmarksnumbered = true
}
\usepackage{cleveref}
\usepackage{graphicx}
\graphicspath{{build/release/pdf}}
\usepackage{subcaption}
\usepackage{cprotect}
\usepackage[most]{tcolorbox}
\newcommand\coloredcomponent[2]
{
  \tcolorboxenvironment{#1}
  {
    breakable,
    enhanced,
    colback = white,
    colframe = #2,
    boxrule = 1pt,
    left = 2pt,
    right = 2pt,
    top = 2pt,
    bottom = 2pt,
    sharp corners,
    before skip = \topsep,
    after skip = \topsep,
  }
}

\counterwithin{equation}{section}
\theoremstyle{definition}

\newtheorem{theorem}{Theorem}[section]
\coloredcomponent{theorem}{blue}

\newtheorem{proposition}[theorem]{Proposition}
\coloredcomponent{proposition}{blue}

\newtheorem{lemma}[theorem]{Lemma}
\coloredcomponent{lemma}{blue}

\newtheorem{corollary}[theorem]{Corollary}
\coloredcomponent{corollary}{blue}

\newtheorem{hypothesis}[theorem]{Hypothesis}
\coloredcomponent{hypothesis}{red}

\newtheorem{notation}[theorem]{Notation}
\coloredcomponent{notation}{green}

\newtheorem{definition}[theorem]{Definition}
\coloredcomponent{definition}{green}

\newtheorem{formulation}[theorem]{Formulation}
\coloredcomponent{formulation}{brown}

\newtheorem{discussion}[theorem]{Discussion}
\coloredcomponent{discussion}{yellow}

\newtheorem{example}[theorem]{Example}
\coloredcomponent{example}{purple}

\newtheorem{remark}[theorem]{Remark}
\coloredcomponent{remark}{orange}

\newtheorem{algorithm}[theorem]{Algorithm}
\coloredcomponent{algorithm}{cyan}

\newtheorem{solution}[theorem]{Solution}
\coloredcomponent{solution}{cyan}

\coloredcomponent{proof}{cyan}

\renewcommand{\thefootnote}{\arabic{footnote}}

\newcommand{\N}{\mathbb{N}}
\newcommand{\Z}{\mathbb{Z}}
\newcommand{\Q}{\mathbb{Q}}
\newcommand{\R}{\mathbb{R}}
\renewcommand{\S}{\mathbb{S}}

\newcommand{\set}[2]{\left\{ #1 \mid #2 \right\}}
\newcommand{\restrict}[2]{\left. #1 \right|_{#2}}

\newcommand{\norm}[1]{\left\lVert#1\right\rVert}
\newcommand{\abs}[1]{\left\lvert#1\right\rvert}
\newcommand{\inner}[2]{\langle#1,#2\rangle}

\newcommand{\linearspan}{\mathop{\rm span}\nolimits}
\newcommand{\Ker}{\mathop{\rm Ker}\nolimits}
\renewcommand{\Im}{\mathop{\rm Im}\nolimits}
\newcommand{\Hom}{\mathop{\rm Hom}\nolimits}
\newcommand{\id}{\mathop{\rm id}\nolimits}
\newcommand{\tr}{\mathop{\rm tr}\nolimits}
\newcommand{\sym}{\mathop{\rm sym}\nolimits}

\newcommand{\grad}{\mathop{\rm grad}\nolimits}
\renewcommand{\div}{\mathop{\rm div}\nolimits}

\newcommand{\Aff}{\mathop{\rm Aff}\nolimits}
\newcommand{\Con}{\mathop{\rm Con}\nolimits}

\newcommand{\Cl}{\mathop{\rm Cl}\nolimits}
\newcommand{\Link}{\mathop{\rm Link}\nolimits}

\newcommand{\sgn}{\mathop{\rm sgn}\nolimits}
\newcommand{\OR}{\mathop{\rm or}\nolimits}
\newcommand{\vol}{\mathop{\rm vol}\nolimits}
\newcommand{\usmile}{\underline{\smile}}
\newcommand{\uwedge}{\underline{\wedge}}

\newcommand{\newterm}[1]{\textbf{#1}}

\newcommand{\amount}{\mathsf{X}}
\newcommand{\potential}{\mathsf{Y}}
\newcommand{\mass}{\mathsf{M}}
\newcommand{\length}{\mathsf{L}}
\renewcommand{\time}{\mathsf{T}}
\newcommand{\temperature}{\theta}
\newcommand{\charge}{\mathsf{C}}

\newcommand{\topStrut}{\rule{0pt}{2.6ex}}

\setcounter{tocdepth}{4}
\setcounter{secnumdepth}{4}

\title{Intrinsic discrete exterior calculus (IDEC)}
\author{Kiprian Berbatov}
\date{26 February 2024}

\begin{document}

\maketitle

\tableofcontents

\section{Exterior algebra}
\label{section:exterior_algebra}
\begin{definition}
  Let
    $R$ be a commutative ring with unity,
    $V$ be a module over $R$.
  The \textbf{exterior algebra} of $V$, $\Lambda V$ is the smallest associative
  algebra with unity containing $V$ as a subspace and satisfying the
  \textit{alternating rule}:
  for every $v \in V$, if $\wedge$ is the multiplication on $\Lambda V$, then
  \begin{equation}
    v \wedge v = 0.
  \end{equation}
\end{definition}
\begin{corollary}
  Let
    $R$ be a commutative ring with unity,
    $V$ be a module over $R$,
    $v, w \in V$.
  Then on $\Lambda V$,
  \begin{equation}
    w \wedge v = - v \wedge w.
  \end{equation}
\end{corollary}
\begin{proof}
  By the alternating rule,
  \begin{equation}
    0 = (v + w) \wedge (v + w)
    = v \wedge v + v \wedge w + w \wedge v + w \wedge w
    = 0 + v \wedge w + w \wedge v + 0
    \Rightarrow w \wedge v = - v \wedge w.
  \end{equation}
\end{proof}
\begin{proposition}
  Let
    $R$ be a commutative ring with unity,
    $d \in \N$
    $V$ be a free module over $R$ of dimension $d$,
    $e_0, ..., e_{d - 1}$ be a basis of $V$.
  Then the $2^d$-element set $S(e)$,
  \begin{equation}
    S(e) :=
    \set{e_{i_0} \wedge ... \wedge e_{i_{p - 1}}}
    {p \in \{0, ..., d\},\ 0 \leq i_0 < ... < i_{p - 1} \leq d - 1}
  \end{equation}
  (for $p = 0$ the empty wedge product is defined to be $1$)
  forms a basis of $\Lambda V$.
\end{proposition}
\begin{remark}
   Let
    $R$ be a commutative ring with unity,
    $d \in \N$
    $V$ be a free module over $R$ of dimension $d$,
    $e_0, ..., e_{d - 1}$ be a basis of $V$,
  The elements of $S(e)$ which are wedge products of $p$ vectors are called
  \textbf{$p$-vectors}.
  Obviously, there are $\binom{n}{p}$ of them.
  Denote by $\Lambda^p V$ the space spanned by those $p$ vectors
  (it does not depend on the basis of $e$ -- in fact $\Lambda^p V$ is the space
  spanned by linear combinations of wedge products of arbitrary vectors).
  Then we have the module decomposition
  \begin{equation}
    \Lambda V := \sum_{p = 0}^d \Lambda^p V.
  \end{equation}
  This decompositions turns $\Lambda V$ into a graded algebra, that is for each
  $p, q \in \N$,
  \begin{equation}
    \omega_p \in \Lambda^p V,\ \eta_q \in \Lambda^q V
    \Rightarrow \omega_p \wedge \eta_q \in \Lambda^{p + q} V
  \end{equation}
  (here we define $\Lambda^r = 0$ for $r > d$).
\end{remark}
\begin{proposition}
  Let
    $R$ be a commutative ring with unity,
    $d \in \N$
    $V$ be a free module over $R$ of dimension $d$,
    $v_0, ..., v_{d - 1} \in V$.
  Then $(v_0, ..., v_{d - 1})$ form a basis of $V$
  if and only if
  \begin{equation}
    v_0 \wedge ... \wedge v_{d - 1}
  \end{equation}
  forms a basis of the $1$-dimensional module $\Lambda^d V$.
\end{proposition}
\begin{proposition}
  Let
    $R$ be a commutative ring with unity,
    $d \in \N$
    $V$ be a free module over $R$ of dimension $d$,
    $e_0, ..., e_{d - 1}$ be a basis of $V$,
    $v_0, ..., v_{d - 1} \in V$ such that for $j = 0, ..., d - 1$,
  \begin{equation}
    v_j = \sum_{i = 0}^{d - 1} e_i A_{i, j}
  \end{equation}
  (in matrix form, $(v_0, ..., v_{d - 1}) = (e_0, ..., e_{d - 1}) A$).
  Then
  \begin{equation}
    v_0 \wedge ... \wedge v_{d - 1} = (\det A) e_0 \wedge ... \wedge e_{d - 1}.
  \end{equation}
  As a consequence, the matrix $A$ is a change of basis matrix if and only if
  $\det A$ is an invertible element of $R$
  (a nonzero element when $R$ is a field).
\end{proposition}


\section{Inner products and Hodge star}
\label{section:inner_products_and_hodge_star}
\begin{definition}
  Let
    $R$ be a commutative ring with unity,
    $V$ be a module over $R$.
  The \textbf{exterior algebra} of $V$, $\Lambda V$ is the smallest associative
  algebra with unity containing $V$ as a subspace and satisfying the
  \textit{alternating rule}:
  for every $v \in V$, if $\wedge$ is the multiplication on $\Lambda V$, then
  \begin{equation}
    v \wedge v = 0.
  \end{equation}
\end{definition}
\begin{corollary}
  Let
    $R$ be a commutative ring with unity,
    $V$ be a module over $R$,
    $v, w \in V$.
  Then on $\Lambda V$,
  \begin{equation}
    w \wedge v = - v \wedge w.
  \end{equation}
\end{corollary}
\begin{proof}
  By the alternating rule,
  \begin{equation}
    0 = (v + w) \wedge (v + w)
    = v \wedge v + v \wedge w + w \wedge v + w \wedge w
    = 0 + v \wedge w + w \wedge v + 0
    \Rightarrow w \wedge v = - v \wedge w.
  \end{equation}
\end{proof}
\begin{proposition}
  Let
    $R$ be a commutative ring with unity,
    $d \in \N$
    $V$ be a free module over $R$ of dimension $d$,
    $e_0, ..., e_{d - 1}$ be a basis of $V$.
  Then the $2^d$-element set $S(e)$,
  \begin{equation}
    S(e) :=
    \set{e_{i_0} \wedge ... \wedge e_{i_{p - 1}}}
    {p \in \{0, ..., d\},\ 0 \leq i_0 < ... < i_{p - 1} \leq d - 1}
  \end{equation}
  (for $p = 0$ the empty wedge product is defined to be $1$)
  forms a basis of $\Lambda V$.
\end{proposition}
\begin{remark}
   Let
    $R$ be a commutative ring with unity,
    $d \in \N$
    $V$ be a free module over $R$ of dimension $d$,
    $e_0, ..., e_{d - 1}$ be a basis of $V$,
  The elements of $S(e)$ which are wedge products of $p$ vectors are called
  \textbf{$p$-vectors}.
  Obviously, there are $\binom{n}{p}$ of them.
  Denote by $\Lambda^p V$ the space spanned by those $p$ vectors
  (it does not depend on the basis of $e$ -- in fact $\Lambda^p V$ is the space
  spanned by linear combinations of wedge products of arbitrary vectors).
  Then we have the module decomposition
  \begin{equation}
    \Lambda V := \sum_{p = 0}^d \Lambda^p V.
  \end{equation}
  This decompositions turns $\Lambda V$ into a graded algebra, that is for each
  $p, q \in \N$,
  \begin{equation}
    \omega_p \in \Lambda^p V,\ \eta_q \in \Lambda^q V
    \Rightarrow \omega_p \wedge \eta_q \in \Lambda^{p + q} V
  \end{equation}
  (here we define $\Lambda^r = 0$ for $r > d$).
\end{remark}
\begin{proposition}
  Let
    $R$ be a commutative ring with unity,
    $d \in \N$
    $V$ be a free module over $R$ of dimension $d$,
    $v_0, ..., v_{d - 1} \in V$.
  Then $(v_0, ..., v_{d - 1})$ form a basis of $V$
  if and only if
  \begin{equation}
    v_0 \wedge ... \wedge v_{d - 1}
  \end{equation}
  forms a basis of the $1$-dimensional module $\Lambda^d V$.
\end{proposition}
\begin{proposition}
  Let
    $R$ be a commutative ring with unity,
    $d \in \N$
    $V$ be a free module over $R$ of dimension $d$,
    $e_0, ..., e_{d - 1}$ be a basis of $V$,
    $v_0, ..., v_{d - 1} \in V$ such that for $j = 0, ..., d - 1$,
  \begin{equation}
    v_j = \sum_{i = 0}^{d - 1} e_i A_{i, j}
  \end{equation}
  (in matrix form, $(v_0, ..., v_{d - 1}) = (e_0, ..., e_{d - 1}) A$).
  Then
  \begin{equation}
    v_0 \wedge ... \wedge v_{d - 1} = (\det A) e_0 \wedge ... \wedge e_{d - 1}.
  \end{equation}
  As a consequence, the matrix $A$ is a change of basis matrix if and only if
  $\det A$ is an invertible element of $R$
  (a nonzero element when $R$ is a field).
\end{proposition}


\section{Meshes}
\label{section:meshes}
\begin{definition}
  Let
    $d \in \N$,
    $K$ be a quasi-cubical mesh of dimension $d$,
    $\partial$ be a boundary operator on $K$,
    $p \in \{1, ..., d\}$.
  The \textbf{adjoint coboundary operator on $p$-cochain}
  $\delta^\star_p \colon C^p K \to C^{p - 1} K$
  is defined as the adjoint of $\delta_{p - 1}$
  with respect to the inner product of cochains.
  In other words, for any $\pi^p \in C^p K,\ \rho^{p - 1} \in C^{p - 1}$,
  \begin{equation}
    \inner{\delta_p^\star \pi^p}{\rho^{p - 1}}_{p - 1}
    = \inner{\pi^p}{\delta_{p - 1} \rho^{p - 1}}_p.
  \end{equation}
  The operator $\delta^\star_p$ has physical dimension $[L^{-2}]$.
\end{definition}


\section{Relative orientation on meshes}
\label{section:relative_orientation_on_meshes}
\begin{theorem}
  Let
    $d \in \N$,
    $M$ be a \hyperref[idec:mesh:definition]{mesh} of dimension $d$,
    $p \in \{2, ..., d\}$,
    $a_p \in M_p$,
    $c_{p - 2} \in M_{p - 2}$,
    $a_p \succ c_{p - 2}$.
  Then there exist exactly two $(p - 1)$-cells
  $b_{p - 1}, b_{p - 1}' \in M_{p - 1}$
  that are between $a_p$ $c_{p - 2}$, i.e.,
  \begin{equation}
    a_p \succ b_{p - 1} \succ c_{p - 2}\ \text{and}\
    a_p \succ b_{p - 1}'' \succ c_{p - 2}.
  \end{equation}
\end{theorem}

\begin{definition}
  Let
    $d \in \N$,
    $K$ be a quasi-cubical mesh of dimension $d$,
    $\partial$ be a boundary operator on $K$,
    $p \in \{1, ..., d\}$.
  The \textbf{adjoint coboundary operator on $p$-cochain}
  $\delta^\star_p \colon C^p K \to C^{p - 1} K$
  is defined as the adjoint of $\delta_{p - 1}$
  with respect to the inner product of cochains.
  In other words, for any $\pi^p \in C^p K,\ \rho^{p - 1} \in C^{p - 1}$,
  \begin{equation}
    \inner{\delta_p^\star \pi^p}{\rho^{p - 1}}_{p - 1}
    = \inner{\pi^p}{\delta_{p - 1} \rho^{p - 1}}_p.
  \end{equation}
  The operator $\delta^\star_p$ has physical dimension $[L^{-2}]$.
\end{definition}

\begin{remark}
  Note that the last condition in the above definition can be written as
  \begin{equation}
    \sum_{b_{p - 1} \in (c_{p - 2}, a_p)}
    \varepsilon(a_p, b_{p - 1}) \varepsilon(b_{p - 1}, c_{p - 2}) = 0.
  \end{equation}
\end{remark}

\begin{theorem}
  Let
    $d \in \N$,
    $M$ be a \hyperref[cmc:mesh:definition]{mesh} of dimension $d$,
    $R$ be a commutative ring with unity.
  Then there exists a relative orientation on $M$.
\end{theorem}


\section{Chains and boundary operator on meshes}
\label{section:chains_and_boundary_operator_on_meshes}
\begin{definition}
  \label{idec:mesh_chain:definition}
  Let
    $d \in \N$,
    $M$ be a \hyperref[idec:mesh:definition]{mesh} of dimension $d$,
    $p \in \N,\ p \in [0, d]$,
    $R$ be a commutative ring with unity (for instance, $R = \R$).
  The space $C_p(M; R)$ of \textbf{$p$-chains}
  on $M$ with coefficients in $R$ is the free $R$-module
  (vector space over $R$ when $R$ is a field, e.g., when $R = \R$)
  generated by $M_p$:
  \begin{equation}
    C_p(M; R) := {\rm Free}_R(M_p).
  \end{equation}
  In other words, the elements of $C_p(M; R)$ are the formal linear combinations
  of cells in $M_p$ in coefficients in $R$.
  An element $c_p$ of $C_p(M; R)$ has the form
  \begin{equation}
    c_p := \lambda_0 c(p_, h_0) + ... + \lambda_{n - 1} c(p, h_{n - 1}),
  \end{equation}
  where for $i = 0, ..., n - 1$, $\lambda_i \in R$ and $c(p, h_i) \in M_p$.
\end{definition}

\begin{definition}
  Let
    $d \in \N$,
    $K$ be a quasi-cubical mesh of dimension $d$,
    $\partial$ be a boundary operator on $K$,
    $p \in \{1, ..., d\}$.
  The \textbf{adjoint coboundary operator on $p$-cochain}
  $\delta^\star_p \colon C^p K \to C^{p - 1} K$
  is defined as the adjoint of $\delta_{p - 1}$
  with respect to the inner product of cochains.
  In other words, for any $\pi^p \in C^p K,\ \rho^{p - 1} \in C^{p - 1}$,
  \begin{equation}
    \inner{\delta_p^\star \pi^p}{\rho^{p - 1}}_{p - 1}
    = \inner{\pi^p}{\delta_{p - 1} \rho^{p - 1}}_p.
  \end{equation}
  The operator $\delta^\star_p$ has physical dimension $[L^{-2}]$.
\end{definition}

\begin{definition}
  Let
    $d \in \N$,
    $K$ be a quasi-cubical mesh of dimension $d$,
    $\partial$ be a boundary operator on $K$,
    $p \in \{1, ..., d\}$.
  The \textbf{adjoint coboundary operator on $p$-cochain}
  $\delta^\star_p \colon C^p K \to C^{p - 1} K$
  is defined as the adjoint of $\delta_{p - 1}$
  with respect to the inner product of cochains.
  In other words, for any $\pi^p \in C^p K,\ \rho^{p - 1} \in C^{p - 1}$,
  \begin{equation}
    \inner{\delta_p^\star \pi^p}{\rho^{p - 1}}_{p - 1}
    = \inner{\pi^p}{\delta_{p - 1} \rho^{p - 1}}_p.
  \end{equation}
  The operator $\delta^\star_p$ has physical dimension $[L^{-2}]$.
\end{definition}

\begin{proposition}
  Let
    $d \in \N$,
    $M$ be a \hyperref[cmc:mesh:definition]{mesh} of dimension $d$,
    $R$ be a commutative ring with unity,
    $\varepsilon$ be a relative orientation on $M$.
  Then the algebra $(C_\bullet M, \partial)$ is a chain complex, i.e.,
  \begin{equation}
    \partial^2 = 0.
  \end{equation}
\end{proposition}

\begin{proof}
  It is enough to prove that for any $p \in \{0, ..., d\}$, $c_p \in M_d$,
  \begin{equation}
    \partial^2 c_p = 0.
  \end{equation}
  The proposition is trivially true for $p = 0$ and $p = 1$
  because $\partial_0 = 0$.
  Assume that $p \geq 2$.
  Then
  \begin{equation}
    \begin{split}
      \partial^2 a_p
      & = \partial_{p - 1} (\partial_p a_p) \\
      & = \partial_{p - 1}
      \left(
        \sum_{b_{p - 1} \prec a_p} \varepsilon(a_p, b_{p - 1}) b_{p - 1}
      \right) \\
      & =
      \sum_{b_{p - 1} \prec a_p}
        \sum_{c_{p - 2} \prec b_{p - 1}}
            \varepsilon(a_p, b_{p - 1})
            \varepsilon(b_{p - 1}, c_{p - 2})
            c_{p - 2} \\
      & =
      \sum_{c_{p - 2} \prec a_p}
        \left(
          \sum _{b_{p - 1} \in (c_{p - 2}, a_p)}
            \varepsilon(a_p, b_{p - 1}) \varepsilon(b_{p - 1}, c_{p - 2})
        \right)
        c_{p - 2} \\
      & = 0
    \end{split}
  \end{equation}
  (the last equation follows from the last condition in the definition of
  \hyperref[cmc:relative_orientation:definition]{relative orientation}).
\end{proof}

\begin{proposition}
  Let
    $d \in \N$,
    $M$ be a \hyperref[idec:mesh:definition]{mesh} of dimension $d$,
    $R$ be a commutative ring with unity,
    $\varepsilon$ and $\varepsilon'$ be a relative orientations on $M$
      with corresponding boundary operators
      $\partial$ and $\partial'$ respectively.
  Then
  \begin{equation}
    (C_\bullet(M; R), \partial) \cong (C_\bullet(M; R), \partial')
  \end{equation}
  ($\cong$ is understood as isomorphism of chain complexes).
\end{proposition}

\begin{remark}
  The above proposition says that the boundary operator is essentially unique,
  i.e., up to isomorphism it does not depend on the chosen relative orientation.
  This motivates the notion of ``\emph{the} boundary operator'' on a mesh.
  Nevertheless, this does not exclude special choices of relative orientations
  in some particular cases like compatibly orientable meshes or regular grids.
\end{remark}


\section{Cochains and coboundary operator on meshes}
\label{section:cochains_and_coboundary_operator_on_meshes}
\begin{definition}
  Let
    $d \in \N$,
    $K$ be a quasi-cubical mesh of dimension $d$,
    $\partial$ be a boundary operator on $K$,
    $p \in \{1, ..., d\}$.
  The \textbf{adjoint coboundary operator on $p$-cochain}
  $\delta^\star_p \colon C^p K \to C^{p - 1} K$
  is defined as the adjoint of $\delta_{p - 1}$
  with respect to the inner product of cochains.
  In other words, for any $\pi^p \in C^p K,\ \rho^{p - 1} \in C^{p - 1}$,
  \begin{equation}
    \inner{\delta_p^\star \pi^p}{\rho^{p - 1}}_{p - 1}
    = \inner{\pi^p}{\delta_{p - 1} \rho^{p - 1}}_p.
  \end{equation}
  The operator $\delta^\star_p$ has physical dimension $[L^{-2}]$.
\end{definition}

\begin{definition}
  Let
    $d \in \N$,
    $K$ be a quasi-cubical mesh of dimension $d$,
    $\partial$ be a boundary operator on $K$,
    $p \in \{1, ..., d\}$.
  The \textbf{adjoint coboundary operator on $p$-cochain}
  $\delta^\star_p \colon C^p K \to C^{p - 1} K$
  is defined as the adjoint of $\delta_{p - 1}$
  with respect to the inner product of cochains.
  In other words, for any $\pi^p \in C^p K,\ \rho^{p - 1} \in C^{p - 1}$,
  \begin{equation}
    \inner{\delta_p^\star \pi^p}{\rho^{p - 1}}_{p - 1}
    = \inner{\pi^p}{\delta_{p - 1} \rho^{p - 1}}_p.
  \end{equation}
  The operator $\delta^\star_p$ has physical dimension $[L^{-2}]$.
\end{definition}

\begin{proposition}
  Let
    $d \in \N$,
    $M$ be a \hyperref[cmc:mesh:definition]{mesh} of dimension $d$,
    $R$ be a commutative ring with unity,
    $\partial$ be a boundary operator on $M$,
    $\delta$ be the corresponding coboundary operator.
  Then
  \begin{equation}
    (C^\bullet(M; R), \delta)
  \end{equation}
  is a cochain complex.
\end{proposition}

\begin{proof}
  A straightforward computation using the fact that $\partial^2 = 0$.
  Indeed, let $\omega \in \Omega^p M$.
  Then
  \begin{equation}
    \begin{split}
      D^2(\omega)
      & = D(D \omega) \\
      & = D(\omega \circ \partial - (-1)^p \partial \circ \omega) \\
      & =
      \omega \circ \partial \circ \partial
        - (-1)^{p + 1} \partial \circ \omega \circ \partial
        - (-1)^p \partial \circ \omega \circ \partial
        - (-1)^p (-1)^{p + 1} \partial \circ \partial \circ \omega \\
      & =
      0
      - (-1)^{p + 1}\partial \circ \omega \circ \partial
      + (-1)^{p + 1}\partial \circ \omega \circ \partial
      + 0 \\
      & = 0.
    \end{split}
  \end{equation}
\end{proof}


\section{Combinatorial differential forms and Forman subdivision}
\label{section:combinatorial_differential_forms_and_forman_subdivision}
\begin{definition}
  Let
    $d \in \N$,
    $K$ be a quasi-cubical mesh of dimension $d$,
    $\partial$ be a boundary operator on $K$,
    $p \in \{1, ..., d\}$.
  The \textbf{adjoint coboundary operator on $p$-cochain}
  $\delta^\star_p \colon C^p K \to C^{p - 1} K$
  is defined as the adjoint of $\delta_{p - 1}$
  with respect to the inner product of cochains.
  In other words, for any $\pi^p \in C^p K,\ \rho^{p - 1} \in C^{p - 1}$,
  \begin{equation}
    \inner{\delta_p^\star \pi^p}{\rho^{p - 1}}_{p - 1}
    = \inner{\pi^p}{\delta_{p - 1} \rho^{p - 1}}_p.
  \end{equation}
  The operator $\delta^\star_p$ has physical dimension $[L^{-2}]$.
\end{definition}

\begin{definition}
  Let
    $d \in \N$,
    $M$ be a \hyperref[cmc:mesh:definition]{mesh} of dimension $d$
    $\partial$ be a boundary operator on $M$.
  The \textbf{discrete differential} on $M$ is the linear map
  \begin{equation}
    D \colon \Omega^\bullet \to \Omega^\bullet
  \end{equation}
  which maps a $p$-form $\omega$ to a $(p + 1)$-form by the formula
  \begin{equation}
    D \omega := \omega \circ \partial - (-1)^p \partial \circ \omega.
  \end{equation}
\end{definition}

\begin{proposition}
  Let
    $d \in \N$,
    $M$ be a \hyperref[cmc:mesh:definition]{mesh} of dimension $d$,
    $R$ be a commutative ring with unity,
    $\partial$ be a boundary operator on $M$,
    $\delta$ be the corresponding coboundary operator.
  Then
  \begin{equation}
    (C^\bullet(M; R), \delta)
  \end{equation}
  is a cochain complex.
\end{proposition}

\begin{proof}
  A straightforward computation using the fact that $\partial^2 = 0$.
  Indeed, let $\omega \in \Omega^p M$.
  Then
  \begin{equation}
    \begin{split}
      D^2(\omega)
      & = D(D \omega) \\
      & = D(\omega \circ \partial - (-1)^p \partial \circ \omega) \\
      & =
      \omega \circ \partial \circ \partial
        - (-1)^{p + 1} \partial \circ \omega \circ \partial
        - (-1)^p \partial \circ \omega \circ \partial
        - (-1)^p (-1)^{p + 1} \partial \circ \partial \circ \omega \\
      & =
      0
      - (-1)^{p + 1}\partial \circ \omega \circ \partial
      + (-1)^{p + 1}\partial \circ \omega \circ \partial
      + 0 \\
      & = 0.
    \end{split}
  \end{equation}
\end{proof}

\begin{definition}
  Let
    $d \in \N$,
    $K$ be a quasi-cubical mesh of dimension $d$,
    $\partial$ be a boundary operator on $K$,
    $p \in \{1, ..., d\}$.
  The \textbf{adjoint coboundary operator on $p$-cochain}
  $\delta^\star_p \colon C^p K \to C^{p - 1} K$
  is defined as the adjoint of $\delta_{p - 1}$
  with respect to the inner product of cochains.
  In other words, for any $\pi^p \in C^p K,\ \rho^{p - 1} \in C^{p - 1}$,
  \begin{equation}
    \inner{\delta_p^\star \pi^p}{\rho^{p - 1}}_{p - 1}
    = \inner{\pi^p}{\delta_{p - 1} \rho^{p - 1}}_p.
  \end{equation}
  The operator $\delta^\star_p$ has physical dimension $[L^{-2}]$.
\end{definition}

\begin{example}
  \leavevmode
  \begin{itemize}
    \item
      $()$ and $(0, 1, 2)$ are jagged arrays of order $1$ (ordinary arrays);
    \item
      $((0, 1, 2), (3), ())$ is a jagged array of order $2$
      (note that its last element is the empty array of order $1$);
    \item
      $(((0, 1, 2), (3)),())$ is a jagged array of order $3$
      (note that its last element is the empty array of order $2$);
  \end{itemize}
\end{example}

\begin{figure}[!ht]
  \begin{subfigure}{.5\textwidth}
    \centering
    \includegraphics[scale=.3]{mesh/pentagon_edge_skeleton}
    \caption{Pentagon}
  \end{subfigure}
  \begin{subfigure}{.5\textwidth}
    \centering
    \includegraphics[scale=.3]{mesh/pentagon_forman_edge_skeleton}
    \caption{Forman subdivision of a pentagon}
  \end{subfigure}

  \begin{subfigure}{.5\textwidth}
    \centering
    \includegraphics[scale=.3]{mesh/circular_18_10_edge_skeleton}
    \caption{Circular mesh}
  \end{subfigure}
  \begin{subfigure}{.5\textwidth}
    \centering
    \includegraphics[scale=.3]{mesh/circular_18_10_forman_edge_skeleton}
    \caption{Forman subdivision of a circular mesh}
  \end{subfigure}

  \begin{subfigure}{.5\textwidth}
    \centering
    \includegraphics[scale=.3]{mesh/2d_10_grains_edge_skeleton}
    \caption{Irregular mesh (produced by Neper)}
  \end{subfigure}
  \begin{subfigure}{.5\textwidth}
    \centering
    \includegraphics[scale=.3]{mesh/2d_10_grains_forman_edge_skeleton}
    \caption{Forman subdivision of an irregular mesh}
  \end{subfigure}
  \caption{Examples of meshes and their Forman subdivisions}
  \label{figure:mesh/forman_subdivision_examples}
\end{figure}

\begin{definition}
  Let
    $d \in \N$,
    $M$ be a \hyperref[idec:mesh:definition]{mesh} of dimension $d$,
    $K$ be the Forman subdivision of $M$,
    $\varepsilon_M$ be the relative orientation on $M$.
  We construct the relative orientation $\varepsilon_K$ as follows.
  Let
    $p_f \in [1, d]$,
    $p \in [p_f, d]$,
    $s = p - p_f$,
    $c_K(p_f, i_f)$ be a $p_f$-cell on $K$,
  \begin{equation}
    c_K(p_f, i_f) = (c(p, i), c(s, l))\
    \text{for some $c(p, i) \in C_p M$ and $c(s, l) \in C_s M$}.
  \end{equation}
  Let $c_K(p_f - 1, j_f)$ be a hyperface of $c_K(p_f, i_f)$.
  Then there exist $q, r \in \N$ such that
  $p \geq q \geq r \geq s$ and $q - r = p_f - 1$,
  such that
  \begin{equation}
    c_K(p_f - 1, j_f) = (c(q, j), c(r, k))\
    \text{for some $c(q, j) \in C_p M$ and $c(r, k) \in C_r M$}.
  \end{equation}
  There are two possibilities for $q$ and $r$:
  $(q, r) = (p - 1, s)$ or $(q, r) = (p, s + 1)$.
  \begin{enumerate}
    \item
      If $(q, r) = (p - 1, s)$, then
      \begin{equation}
        c_K(p_f - 1, j_f) = (c(p - 1, j), c(s, l)),\
        \text{where $c(p, i) \succ c(p - 1, j) \succeq c(s, l)$}.
      \end{equation}
      In this case
      \begin{equation}
        \varepsilon_K(c_K(p_f, i_f), c_K(p_f - 1, j_f))
        = \varepsilon_M(c_M(p, i), c_M(p - 1, j)).
      \end{equation}
    \item
      If $(q, r) = (p, s + 1)$, then
      \begin{equation}
        c_K(p_f - 1, j_f) = (c(p, i), c(s + 1, k)),\
        \text{where $c(p, i) \succeq c(s + 1, k) \succ c(s, l)$}.
      \end{equation}
      In this case
      \begin{equation}
        \varepsilon_K(c_K(p_f, i_f), c_K(p_f - 1, j_f))
        = (-1)^{p_f} \varepsilon_M(c_M(s + 1, k), c_M(s, l)).
      \end{equation}
  \end{enumerate}
\end{definition}

\begin{theorem}
  Let
    $d \in \N$,
    $M$ be a \hyperref[idec:mesh:definition]{mesh} of dimension $d$,
    $\varepsilon_M$ be a relative orientation on $M$
      with corresponding boundary operator $\partial_M$
      and discrete differential $D_M$.
  Let $K$ be the Forman subdivision of $M$,
  $\varepsilon_K$ be the orientation on $K$ constructed above,
  $d_K$ be the corresponding coboundary operator on $K$.
  Then
  \begin{equation}
    (\Omega^p M, D_M) \cong (C^p K, d_K).
  \end{equation}
  with the isomorphism being the mapping of the basis forms to basis cochains
  introduced in the construction of $K$.
\end{theorem}

\begin{definition}
  Let
    $d \in \N$,
    $K$ be a quasi-cubical mesh of dimension $d$,
    $\partial$ be a boundary operator on $K$,
    $p \in \{1, ..., d\}$.
  The \textbf{adjoint coboundary operator on $p$-cochain}
  $\delta^\star_p \colon C^p K \to C^{p - 1} K$
  is defined as the adjoint of $\delta_{p - 1}$
  with respect to the inner product of cochains.
  In other words, for any $\pi^p \in C^p K,\ \rho^{p - 1} \in C^{p - 1}$,
  \begin{equation}
    \inner{\delta_p^\star \pi^p}{\rho^{p - 1}}_{p - 1}
    = \inner{\pi^p}{\delta_{p - 1} \rho^{p - 1}}_p.
  \end{equation}
  The operator $\delta^\star_p$ has physical dimension $[L^{-2}]$.
\end{definition}

\begin{example}
  \leavevmode
  \begin{itemize}
    \item
      $()$ and $(0, 1, 2)$ are jagged arrays of order $1$ (ordinary arrays);
    \item
      $((0, 1, 2), (3), ())$ is a jagged array of order $2$
      (note that its last element is the empty array of order $1$);
    \item
      $(((0, 1, 2), (3)),())$ is a jagged array of order $3$
      (note that its last element is the empty array of order $2$);
  \end{itemize}
\end{example}

\begin{definition}
  Let
    $d \in \N$,
    $K$ be a quasi-cubical mesh of dimension $d$,
    $\partial$ be a boundary operator on $K$,
    $p \in \{1, ..., d\}$.
  The \textbf{adjoint coboundary operator on $p$-cochain}
  $\delta^\star_p \colon C^p K \to C^{p - 1} K$
  is defined as the adjoint of $\delta_{p - 1}$
  with respect to the inner product of cochains.
  In other words, for any $\pi^p \in C^p K,\ \rho^{p - 1} \in C^{p - 1}$,
  \begin{equation}
    \inner{\delta_p^\star \pi^p}{\rho^{p - 1}}_{p - 1}
    = \inner{\pi^p}{\delta_{p - 1} \rho^{p - 1}}_p.
  \end{equation}
  The operator $\delta^\star_p$ has physical dimension $[L^{-2}]$.
\end{definition}

\begin{definition}
  Let
    $(P, \preceq)$ be a partially ordered set,
    $a, b \in P$ with $a \preceq b$.
  The \textbf{closed interval} $[a, b]$ is defined as
  \begin{equation}
    [a, b] := \set{x \in P}{a \preceq x\ \&\ x \preceq b}.
  \end{equation}
\end{definition}

\begin{definition}
  We say that a mesh $M$ is \textbf{interval-simplicial} if for any two cells
  $a, b \in M$ with $a \preceq b$, the interval $[a, b]$ is an abstract simplex.
\end{definition}

\begin{proposition}
  Let
    $M$ be a \hyperref[cmc:mesh:definition]{mesh}
    and $K$ be its Forman subdivision.
  Then $M$ is interval-simplicial if and only if $K$ is quasi-cubical.
\end{proposition}

\begin{proposition}
  Let $M$ be a \hyperref[cmc:mesh:definition]{mesh} of dimension at most $2$.
  Then $M$ is interval-simplicial.
  In particular, its Forman subdivision is quasi-cubical.
\end{proposition}

\begin{definition}
  Let $d \in \N$ and $P$ be a polytope of dimension $d$.
  We say that $P$ is a \textbf{simple polytope}
  if any of its nodes is connected to exactly $d$ edges.
\end{definition}

\begin{proposition}
  Let $M$ be a \hyperref[cmc:mesh:definition]{mesh} of dimension $3$.
  Then $M$ is interval-simplicial if and only if all $3$-cells of $M$
  are simple polytopes.
  In particular, if $K$ is the Forman subdivision of $M$,
  then $K$ is quasi-cubical if and only if all $3$-cells of $M$
  are simple polytopes.
\end{proposition}

\begin{proposition}
  Let
    $d \in \N$,
    $K$ be a quasi-cubical mesh of dimension $d$,
    $\partial$ be the topological (unsigned) boundary operator on $K$,
    $\delta$ be the topological (unsigned) coboundary operator on $K$,
    $\perp$ be the perpendicularity operator on $K$,
    $p \in \{0, ..., d\}$.
  Then
  \begin{equation}
    \partial_{d - p} \circ \perp_p = \perp_{p + 1} \circ \delta_p.
  \end{equation}
\end{proposition}


\section{Metric-dependent calculus on quasi-cubical meshes}
\label{section:metric_dependent_calculus_on_quasi_cubical_meshes}
\begin{discussion}
  As we saw, the Forman subdivision of an interval-simplicial mesh leads to a
  quasi-cubical mesh.
  Interval-simplicial meshes are not that uncommon:
  \begin{enumerate}
    \item
      all meshes of dimension at most $2$ are interval-simplicial;
    \item
      all $3$D meshes of simple polytopes are interval-simplicial;
    \item
      all simplicial and quasi-cubical meshes are interval-simplicial;
    \item
      the product of interval-simplicial meshes is an interval-simplicial mesh.
  \end{enumerate}
  For that reason we will build our calculus on quasi-cubical meshes thought as
  the Forman subdivision of an interval-simplicial mesh.
\end{discussion}

\begin{definition}
  Let
    $d \in \N$,
    $K$ be a quasi-cubical mesh of dimension $d$,
    $a_d \in M_d$,
    $p \in \{0, ..., d\}$,
    $b_p \in M_p$,
    $c_{d - p} \in M_{d - p}$.
  We say that $b_p$ and $c_{d - p}$ are \textbf{topologically orthogonal}
  with respect to $a_d$ if $b_p, c_{d - p} \preceq a_d$, and the intersection of
  $b_p$ with $c_{d - p}$ is a node in $a_d$.
  In this case we write
  \begin{equation}
    b_p \oplus c_{d - p} = a_p\
    \text{and}\
    b_p \perp c_{d - p}.
  \end{equation}
\end{definition}

\begin{notation}
  Let $d \in \N$,
  $P$ be a polytope of dimension $d$.
  The \textbf{(Euclidean) measure} of $P$ is denoted by
  \begin{equation}
    \mu_d(P) (= \mu(P)).
  \end{equation}
  If $P$ is with standard physical dimensions
  (e.g., it has not been non-dimensionalised),
  then $\mu_d(P)$ is of physical dimension $[L^d]$.
\end{notation}

\begin{definition}
  Let
    $d \in \N$,
    $K$ be a quasi-cubical mesh of dimension $d$,
    $\partial$ be a boundary operator on $K$,
    $p \in \{1, ..., d\}$.
  The \textbf{adjoint coboundary operator on $p$-cochain}
  $\delta^\star_p \colon C^p K \to C^{p - 1} K$
  is defined as the adjoint of $\delta_{p - 1}$
  with respect to the inner product of cochains.
  In other words, for any $\pi^p \in C^p K,\ \rho^{p - 1} \in C^{p - 1}$,
  \begin{equation}
    \inner{\delta_p^\star \pi^p}{\rho^{p - 1}}_{p - 1}
    = \inner{\pi^p}{\delta_{p - 1} \rho^{p - 1}}_p.
  \end{equation}
  The operator $\delta^\star_p$ has physical dimension $[L^{-2}]$.
\end{definition}

\begin{example}
  Let
    $d \in \N$,
    $h \in \R^+$
    $K$ be a regular cubical mesh of dimension $d$ with size $h$,
    $p \in \{0, ..., d\}$,
    $b_p$ be an internal $p$-cell in $K$.
  Then
  \begin{equation}
    \inner{b^p}{b^p}_p = h^{d - 2 p}.
  \end{equation}
\end{example}

\begin{definition}
  Let
    $d \in \N$,
    $K$ be a quasi-cubical mesh of dimension $d$,
    $\partial$ be a boundary operator on $K$,
    $p \in \{1, ..., d\}$.
  The \textbf{adjoint coboundary operator on $p$-cochain}
  $\delta^\star_p \colon C^p K \to C^{p - 1} K$
  is defined as the adjoint of $\delta_{p - 1}$
  with respect to the inner product of cochains.
  In other words, for any $\pi^p \in C^p K,\ \rho^{p - 1} \in C^{p - 1}$,
  \begin{equation}
    \inner{\delta_p^\star \pi^p}{\rho^{p - 1}}_{p - 1}
    = \inner{\pi^p}{\delta_{p - 1} \rho^{p - 1}}_p.
  \end{equation}
  The operator $\delta^\star_p$ has physical dimension $[L^{-2}]$.
\end{definition}

\begin{proposition}
  Let
    $d \in \N$,
    $K$ be a compatibly oriented quasi-cubical
    \hyperref[idec:mesh:definition]{mesh} of dimension $d$,
    $[K] := \sum_{c_d \in K_d} c^d$ be the fundamental class of $K$
    $\inner{\cdot}{\cdot}$ be an orthogonal inner product on $K$,
    $p \in \{0, ..., d\}$.
  The \hyperref[idec/mesh/quasi_cubical/hodge_star/concept-definition]
               {Hodge star operator}
  $\star_p \colon C^p K \to C^{d - p} K$ has the following closed form:
  for any $\pi^p \in C^p K$ and any $c^{d - p} \in C^{d - p} K$,
  \begin{equation}
    (\star_p \pi^p)(b_{d - p})
    = \sum_{a_p \perp b_{d - p}}
        \frac{(b^{d - p} \smile a^p)[K]}{\inner{b^{d - p}}{b^{d - p}}}
        \pi^p(a_p).
  \end{equation}
\end{proposition}

\begin{corollary}
  Let
    $d \in \N$,
    $h \in \R^+$
    $K$ a cubical grid of dimension $d$ with size $h$
      with its standard orientation,
    $p \in \{0, ..., d\}$,
    $\pi^p \in C^p K$,
    $b_{p - 1} \in K_{p - 1}$ be an internal cell.
  Then
  \begin{equation}
    (\delta^\star_p \pi^p) b_{p - 1} =
    \frac{1}{h^2}
    \sum_{b_{p - 1} \prec a_p}
      \varepsilon(a_p, b_{p - 1})
      \pi^p(a_p).
  \end{equation}
\end{corollary}

\begin{definition}
  Let
    $d \in \N$,
    $K$ be a quasi-cubical mesh of dimension $d$,
    $\partial$ be a boundary operator on $K$,
    $p \in \{1, ..., d\}$.
  The \textbf{adjoint coboundary operator on $p$-cochain}
  $\delta^\star_p \colon C^p K \to C^{p - 1} K$
  is defined as the adjoint of $\delta_{p - 1}$
  with respect to the inner product of cochains.
  In other words, for any $\pi^p \in C^p K,\ \rho^{p - 1} \in C^{p - 1}$,
  \begin{equation}
    \inner{\delta_p^\star \pi^p}{\rho^{p - 1}}_{p - 1}
    = \inner{\pi^p}{\delta_{p - 1} \rho^{p - 1}}_p.
  \end{equation}
  The operator $\delta^\star_p$ has physical dimension $[L^{-2}]$.
\end{definition}

\begin{proposition}
  Let
    $d \in \N$,
    $K$ be a compatibly oriented quasi-cubical
    \hyperref[idec:mesh:definition]{mesh} of dimension $d$,
    $[K] := \sum_{c_d \in K_d} c^d$ be the fundamental class of $K$
    $\inner{\cdot}{\cdot}$ be an orthogonal inner product on $K$,
    $p \in \{0, ..., d\}$.
  The \hyperref[idec/mesh/quasi_cubical/hodge_star/concept-definition]
               {Hodge star operator}
  $\star_p \colon C^p K \to C^{d - p} K$ has the following closed form:
  for any $\pi^p \in C^p K$ and any $c^{d - p} \in C^{d - p} K$,
  \begin{equation}
    (\star_p \pi^p)(b_{d - p})
    = \sum_{a_p \perp b_{d - p}}
        \frac{(b^{d - p} \smile a^p)[K]}{\inner{b^{d - p}}{b^{d - p}}}
        \pi^p(a_p).
  \end{equation}
\end{proposition}


\section{Approximating vector fields with 1-cochains}
\label{section:approximating_vector_fields_with_1_cochains}
\begin{definition}
  Let
    $d \in \N$,
    $K$ be a quasi-cubical mesh of dimension $d$,
    $\partial$ be a boundary operator on $K$,
    $p \in \{1, ..., d\}$.
  The \textbf{adjoint coboundary operator on $p$-cochain}
  $\delta^\star_p \colon C^p K \to C^{p - 1} K$
  is defined as the adjoint of $\delta_{p - 1}$
  with respect to the inner product of cochains.
  In other words, for any $\pi^p \in C^p K,\ \rho^{p - 1} \in C^{p - 1}$,
  \begin{equation}
    \inner{\delta_p^\star \pi^p}{\rho^{p - 1}}_{p - 1}
    = \inner{\pi^p}{\delta_{p - 1} \rho^{p - 1}}_p.
  \end{equation}
  The operator $\delta^\star_p$ has physical dimension $[L^{-2}]$.
\end{definition}

\begin{remark}
  Let $A$ be a matrix that has a physical dimension [X],
    $B$ be a Moore-Penrose inverse of $A$.
  Then $B\ [X^{-1}]$.
\end{remark}

\begin{theorem}
  Let $m, n \in \N$, $A$ be a real $m \times n$ matrix.
  Then $A$ has a unique Moore-Penrose inverse, denoted by $A^*$.
\end{theorem}

\begin{remark}
  If the matrix $A$ is of full rank there exists a closed formula for $A^*$.
  \begin{enumerate}
    \item
      If $A$ is a square matrix with full rank, i.e., an invertible one, then
      $A^* = A^{-1}$.
    \item
      If $m > n$ and $A$ is an $m \times n$ matrix of full rank, then its
      columns are linearly independent which means that $A^T A$ is
      symmetric and positive definite and hence invertible.
      (Its inverse $(A^T A)^{-1}$ is also symmetric and positive definite.)
      It is then easy to check that
      \begin{equation}
        B := (A^T A)^{-1} A^T
      \end{equation}
      is the Moore-Penrose inverse of $A$.
      Indeed, obviously $B$ is a left inverse of $A$, and
      \begin{subequations}
        \begin{alignat}{2}
          & A B A && = A (B A) = A I_n = A, \\
          & B A B && = (B A) B = I_n B = B, \\
          & (A B)^T && = (A (A^T A)^{-1} A^T)^T
            = (A^T)^T ((A^T A)^{-1})^T A^T = A (A^T A)^{-1} A^T = A B, \\
          & (B A)^T && = I_n^T = I_n = B A.
        \end{alignat}
      \end{subequations}
    \item
      If $m < n$ and $A$ is an $m \times n$ matrix of full rank, then an
      analogous reasoning to the previous point shows that
      \begin{equation}
        A^* = A^T (A A^T)^{-1}.
      \end{equation}
  \end{enumerate}
\end{remark}

\begin{definition}
  Let
    $d \in \N$,
    $K$ be a quasi-cubical mesh of dimension $d$,
    $\partial$ be a boundary operator on $K$,
    $p \in \{1, ..., d\}$.
  The \textbf{adjoint coboundary operator on $p$-cochain}
  $\delta^\star_p \colon C^p K \to C^{p - 1} K$
  is defined as the adjoint of $\delta_{p - 1}$
  with respect to the inner product of cochains.
  In other words, for any $\pi^p \in C^p K,\ \rho^{p - 1} \in C^{p - 1}$,
  \begin{equation}
    \inner{\delta_p^\star \pi^p}{\rho^{p - 1}}_{p - 1}
    = \inner{\pi^p}{\delta_{p - 1} \rho^{p - 1}}_p.
  \end{equation}
  The operator $\delta^\star_p$ has physical dimension $[L^{-2}]$.
\end{definition}

\begin{definition}
  Let:
  \begin{enumerate}
    \item
      $d \in \N,\ d \geq 1$;
    \item
      $K$ be a flat mesh of dimension $d$
      (with a chosen embedding in $\R^d$);
    \item
      $\epsilon$ be a relative orientation on $K$;
    \item
      $\mathcal{N}_i$ be a node in $K$ connected to $n > 0$ edges;
    \item
      $\mathcal{E}_{j_0}, ..., \mathcal{E}_{j_{n - 1}}$ be all the edges
      containing $\mathcal{N}_i$ as a node;
    \item
      $\mathcal{N}_{i_0}, ..., \mathcal{N}_{i_{n - 1}}$ be the other than
      other $\mathcal{N}_i$ of
      $\mathcal{E}_{j_0}, ..., \mathcal{E}_{j_{n - 1}}$ respectively.
  \end{enumerate}
  We define the \textbf{node matrix} $\mathcal{L}_{\mathcal{N}_i}$ of
  $\mathcal{N}_i$ by
  \begin{equation}
    \mathcal{L}_{\mathcal{N}_i} :=
    \begin{pmatrix}
      x_{i_0, 0} - x_{i, 0} & \cdots & x_{i_0, d - 1} - x_{i, d - 1} \\
      \vdots & \ddots & \vdots \\
      x_{i_{n - 1}, 0} - x_{i, 0} & \cdots & x_{i_{n - 1}, d - 1} - x_{i, d - 1}
    \end{pmatrix}
    \in \R^{n \times d}.
  \end{equation}
  The physical dimension of $\mathcal{L}_{\mathcal{N}_i}$ is $[L]$.
\end{definition}

\begin{definition}
  Let
    $d \in \N\ \text{with}\ d \geq 1$,
    $K$ be a mesh of dimension $d$,
    $\pi^1 \in C^1 K$,
    $\mathcal{N}_i \in K_0$.
  Define the \textbf{neighbor representation} $\widehat{\pi^1}_{\mathcal{N}_i}$
  of $\pi^1$ at $\mathcal{N}_i$ by
  \begin{equation}
    \widehat{\pi^1}_{\mathcal{N}_i} :=
    ( \epsilon_K(\mathcal{E}_{j_0}, \mathcal{N}_{i_0}) \pi^1(\mathcal{E}_{j_0}),
      \cdots,
        \epsilon_K(\mathcal{E}_{j_{k - 1}}, \mathcal{N}_{i_{k - 1}})
        \pi^1(\mathcal{E}_{j_{k - 1}})
    ) \in \R^n.
  \end{equation}
  The neighbor is a dimensionless operator.
\end{definition}

\begin{definition}
  Let
    $d \in \N\ \text{with}\ d \geq 1$,
    $K$ be a flat mesh of dimension $d$,
    $\pi^1 \in C^1 K$,
    $\mathcal{N}_i \in K_0$ with corresponding Euclidean coordinates $x_i$.
  Define the \textbf{1-cochain embedding} $\overline{\pi^1}(x_i)$
  of $\pi^1$ at $\mathcal{N}_i$ by
  \begin{equation}
    \overline{\pi^1}(x_i) :=
    \left(\mathcal{L}_{\mathcal{N}_i}\right)^* \widehat{\pi^1}_{\mathcal{N}_i}
    \in \R^d.
  \end{equation}
  The $1$-cochain embedding operator has physical dimension $[L^{-1}]$.
\end{definition}

\begin{example}
  Let $h \in \R^+$ and $K$ be a regular subdivision with size $h$
  of some interval,
  all the edges in $K$ are oriented from left to right,
  $\pi^1 \in C^1 K$.
  \begin{enumerate}
    \item
      Consider an interior point $\mathcal{N}_i$ with neighboring edges
      $\mathcal{E}_{i - 1}$ and $\mathcal{E}_i$ and corresponding neighboring
      nodes $\mathcal{N}_{i - 1}$ and $\mathcal{N}_{i + 1}$.
      Then
      \begin{equation}
        \mathcal{L}_{\mathcal{N}_i} =
        \begin{pmatrix}
          -h \\
          h
        \end{pmatrix}
        \Rightarrow
        (\mathcal{L}_{\mathcal{N}_i})^* =
          \frac{1}{2 h}
          \begin{pmatrix}
            -1 & 1
          \end{pmatrix}
      \end{equation}
      and
      \begin{equation}
        \widehat{\pi^1}_{\mathcal{N}_i} =
        \begin{pmatrix}
          - \pi^1 \mathcal{E}_{i - 1} \\
          \pi^1 \mathcal{E}_i
        \end{pmatrix}
        \Rightarrow
        \overline{\pi^1}(x_i) =
          \frac{1}{2 h}
          \begin{pmatrix}
            -1 & 1
          \end{pmatrix}
          \begin{pmatrix}
            - \pi^1 \mathcal{E}_{i - 1} \\
            \pi^1 \mathcal{E}_i
          \end{pmatrix}
        = \frac{1}{2 h}
          \left(\pi^1 \mathcal{E}_{i - 1} + \pi^1 \mathcal{E}_i\right).
      \end{equation}
  \end{enumerate}
\end{example}

\begin{definition}
  \label{idec/vector_field_to_1_cochain/definition}
  Let $K$ be an embedded flat mesh and $u$ be a vector field on $|K|$.
  Define the \textbf{approximation} $\underline{u} \in C^1 K$ as follows.
  Let $\mathcal{E}_i$ be an edge with endpoints $\mathcal{N}_{i_0}$ and
  $\mathcal{N}_{i_1}$, oriented from $\mathcal{N}_{i_0}$ to $\mathcal{N}_{i_1}$.
  Then
  \begin{equation}
    \underline{u}(\mathcal{E}_i) :=
      \frac{1}{2}
      ( - (\mathcal{L}_{\mathcal{N}_{i_0}} u(x_{i_0}))_0
        + (\mathcal{L}_{\mathcal{N}_{i_1}} u(x_{i_1}))_1
      ).
  \end{equation}
  The approximation operator is of physical dimension $[L]$.
\end{definition}

\begin{example}
  With the mesh of the previous example, we have
  \begin{equation}
    \label{idec/vector_field_to_1_cochain/1d_example:exact_value}
    \underline{u}(\mathcal{E}_i)
    = \frac{1}{2} (h u(x_i) + h u(x_{i + 1}))
    = h \frac{u(x_i) + u(x_{i + 1})}{2}.
  \end{equation}
  Let's calculate the consecutive application of approximation and embedding
  (and vice versa).
  \begin{equation}
    \underline{\left(\overline{\pi^1}\right)}(\mathcal{E}_i)
    = h
      \left(
        \frac{\overline{\pi^1}(x_i) + \overline{\pi^1}(x_{i + 1})}{2}
      \right)
    = \frac{h}{2}
      \frac{1}{2 h}
      ((\pi^1 \mathcal{E}_{i - 1} + \pi^1 \mathcal{E}_i)
       + (\pi^1 \mathcal{E}_i + \pi^1 \mathcal{E}_{i + 1}))
    = \frac
    {\pi^1 \mathcal{E}_{i - 1} + 2 \pi^1 \mathcal{E}_i
      + \pi^1 \mathcal{E}_{i + 1}}
    {4}.
  \end{equation}
  \begin{equation}
    \overline{\left(\underline{u}\right)}(x_i)
    = \frac{1}{2 h}
      \left(
        \underline{u} \mathcal{E}_{i - 1} + \underline{u} \mathcal{E}_i
      \right)
    = \frac{1}{2 h}
      \frac{h}{2}
      ((u(x_{i - 1}) + u(x_i)) + (u(x_i) + u(x_{i + 1})))
    = \frac{u(x_{i - 1}) + 2 u(x_i) + u(x_{i + 1})}{4}.
  \end{equation}
  In both cases of composition of embedding and approximation the final result
  is the identity operator when $\pi^1$ (respectively $u$) is linear with
  respect to the index $i$.
\end{example}

\begin{discussion}
  Let me summarize the operations relating cochains and embedding.
  We will use notation coming from dependent type theory for functions whose
  codomain depends on the domain.
  Namely, if
    $X$ is a type (set),
    $\{Y(x)\}_{x \in X}$ is a family of sets and
    $\{f(x) \in Y(x)\}_{x \in X}$,
  we will write
  \begin{equation}
    f \colon \prod_{x \in X} Y(x).
  \end{equation}
  Let
    $d \in \N$,
    $K$ be a quasi-cubical flat mesh of dimension $d$,
    $X$ be the manifold it encompasses.
  We define the following data:
  \begin{itemize}
    \item
      $n \colon K_0 \to \N$ denotes the number of node neighbors of a $0$-cell;
    \item
      $\widehat{\phantom{T}} \colon C^1 K \to
        \displaystyle \prod_{x_0 \in K_0} \R^{n(x_0)}$
      denotes the neighbor representation of a $1$-cochain,
      $\widehat{\phantom{T}}\ [1]$;
    \item
      $\displaystyle
        \mathcal{L} \colon \prod_{x_0 \in K_0} M_{n(x_0), d}(\R)$
      denotes the node neighbors matrix,
      $\mathcal{L}\ [L]$;
    \item
      $\displaystyle
        \star \colon \prod_{(m, n) \in \N^2} M_{m, n}(\R) \to M_{n, m}(\R)$
      denotes the Moore-Penrose inverse of a rectangular matrix,
      ($\star$ reverses physical dimensions);
    \item
      $\overline{\phantom{T}} \colon C^1 K \to \Hom_\R(C^0 K, \R^d)$
      denotes the approximation of a $1$-cochain as a Euclidean vector-valued
      $0$-cochain,
      \begin{equation}
        \overline{\pi^1} c_0 :=
        (\mathcal{L}_{c_0})^\star \cdot (\widehat{\pi^1})_{c_0},
      \end{equation}
      $\overline{\phantom{T}}\ [L^{-1}]$;
    \item
      $\underline{\phantom{T}} \colon \chi X \to C^1 K$
      denotes the discretization of a continuum vector field as a $1$-cochain,
      $\underline{\phantom{T}}\ [L]$;
  \end{itemize}
\end{discussion}


\section{Continuous heat transport}
\label{section:continuous_diffusion}
\begin{discussion}
  In this section we will consider the heat transport phenomenon in both
  transient and steady-state form.
  Our formulation will be represented in the language of differential forms
  because they better represent the meaning of physical quantities.
  Various (weak) reformulations will be presented -- those reformulations will
  give us hints on how to construct purely discrete formulations.

  We will formulate our phenomenon in arbitrary dimensions, although our model
  problem is in physical $3$-dimensional space.
  The reason is that we conduct tests in different dimensions and, also,
  transport phenomena can be applied in domains with different dimensions.
\end{discussion}

\begin{discussion}
  \label{idec/diffusion/continuous/model_with_differential_forms-discussion}
  Let:
  \begin{itemize}
    \item
      $D$ be a positive integer (space dimension);
    \item
      $X$ be an open region in $\R^D$ (the space region);
    \item
      $t_0 [T] \in \R$ be the initial time;
    \item
      $I := [t_0, \infty)$;
  \end{itemize}
  The main physical quantities in our model are:
  \begin{itemize}
    \item
      temperature $u^0 [\Theta] \colon I \to \Omega^0 X$:
      for any moment $t \in I$ and any point $x \in X$,
      \begin{equation}
        \text{``temperature $[\Theta]$ on $x$ at time $t$''}
        = u^0(t)(x) := u^0(t, x);
      \end{equation}
    \item
      heat energy $Q^D [E] \colon I \to \Omega^D X$:
      for any moment $t \in I$ and any volume $V_D \subseteq X$,
      \begin{equation}
        \text{``total heat energy $[E]$ of the system on $V$ at time $t$''}
        = \int_{V_D} Q^D(t);
      \end{equation}
    \item
      heat flow rate $q^{D - 1} [E T^{-1}] \colon I \to \Omega^{D - 1} X$:
      for any time interval $[t_1, t_2] \subset I$
      and any hypersurface $S_{D - 1} \subset X$,
      \begin{equation}
        \text{``total flow $[E]$ through $S_{D - 1}$ in $[t_1, t_2]$''}
        = \int_{t_1}^{t_2}\left(\int_{S_{D - 1}} q^{D - 1}(t)\right)\, d t.
      \end{equation}
      (Here we assume that $S_{D - 1}$ is oriented.
       Let $U_D$ and $V_D$ be adjacent regions having $S_{D - 1}$ as a common
       boundary, such that $\varepsilon(U_D, S_{D - 1}) = -1$,
       $\varepsilon(V_D, S_{D - 1}) = 1$.
       Then the above equation measures the total flow from $U_D$ to $V_D$.)
  \end{itemize}
  We will also need the dual variables of heat energy, temperature, and
  flow rate.
  \begin{itemize}
    \item
      dual temperature $\tilde{u}^D [\Theta L^D] \colon I \to \Omega^D X$
      defined by
      \begin{equation}
        \tilde{u}^D := \star_0 u^0
      \end{equation}
      (althiugh using non-zero based temperature scale might make $\star_0$ not
      well defined, this will not cause problems as we will always take
      temperature differences when substituting in equations);
    \item
      heat energy density $\tilde{Q}^0 [E L^{-D}] \colon I \to \Omega^0 X$
      defined by
      \begin{equation}
        \tilde{Q}^0 := \star_D Q^D;
      \end{equation}
    \item
      dual flow rate
      $\tilde{q}^1 [E T L^{2 - D}] \colon I \to \Omega^1 X$
      defined by
      \begin{equation}
        \tilde{q} := \star_{D - 1} q^{D - 1};
      \end{equation}
  \end{itemize}
  The governing laws are formulated as follows.
  \begin{itemize}
    \item
      Let $f^D [E T^{-1}] \colon I \to \Omega^D X$ be an external heat source:
      for any time interval $[t_1, t_2] \subset I$
      and any volume $V_D \subseteq X$,
      \begin{equation}
        \text{``total net heat production $[E]$ in $V_D$ in $[t_1, t_2]$''}
        = \int_{t_1}^{t_2} \left(\int_{V_D} f^D(t) \right)\, d t.
      \end{equation}
      \textbf{Conservation of heat energy} is given by the following relation:
      for any time interval $[t_1, t_2] \subset I$
      and any volume $V_D \subseteq X$,
      \begin{equation}
        \begin{split}
        \text{``heat difference on $V_D$ between moments $t_2$ and $t_1$''}
        & =
          \text{``heat inflow through the boundary of $V_D$
          in $[t_1, t_2]$''} \\
        & +
          \text{``heat production inside $V_D$ in $[t_1, t_2]$''}.
        \end{split}
      \end{equation}
      In symbolic terms, the last equation is written as
      \begin{equation}
        \int_{V_D} (Q^D(t_2) - Q^D(t_1))
        = \int_{t_1}^{t_2}
          \left(\int_{\partial V_D} q^{D - 1}(t) \right)\, d t
        + \int_{t_1}^{t_2} \left(\int_{V_D} f^D(t) \right)\, d t.
      \end{equation}
      Using Stokes' theorem twice, we get the equation
      \begin{equation}
        \int_{t_1}^{t_2}
          \left(\int_{V_D} \frac{\partial Q^D}{\partial t}\right)\, d t =
          \int_{t_1}^{t_2} \left(\int_{V_D} d_{D - 1} q^{D - 1} \right)\, d t
        + \int_{t_1}^{t_2} \left(\int_{V_D} f^D \right)\, d t.
      \end{equation}
      Since the time interval $[t_1, t_2]$ and the volume $V_D$ are arbitrary,
      we can drop integrals and arrive at the differential form
      \begin{equation}
        \frac{\partial Q^D}{\partial t} = d_{D - 1} q^{D - 1} + f^D.
      \end{equation}
    \item
      Let
        $u_0 [\Theta] \in \Omega^0 X$ be the initial temperature.
      The \textbf{initial condition} is the prescribed initial temperature:
      \begin{equation}
        u^0(t_0) = u_0.
      \end{equation}
    \item
      Let $\pi_0 [E L^{-D} \Theta^{-1}] \colon \Omega^0 X \to \Omega^0 X$
      be the volumetric heat capacity.
      The \textbf{relation between temperature change and heat energy change}
      is given by
      \begin{equation}
        \frac{\partial Q^D}{\partial t}
        = \star_0 \left(\frac{\partial \tilde{Q}^0}{\partial t}\right)
        = \star_0 \left(\pi_0 \frac{\partial u^0}{\partial t}\right).
      \end{equation}
    \item
      Consider two adjacent volumes $U_D$ and $V_D$
      with a common surface $S_{D - 1}$, such that
      $\varepsilon(U_D, S_{D - 1}) = -1$ and
      $\varepsilon(V_D, S_{D - 1}) = 1$.
      According to the second law of thermodynamics, heat flows from regions of
      higher temperature to regions of lower temperatures.
      Therefore, the net flow through $S_{D - 1}$ is in the negative direction
      of the temperature difference between $U_D$ and $V_D$.

      Let
      $\kappa_{D - 1} [E L^{2 - D} T^{-1} \Theta^{-1}]
      \colon \Omega^{D - 1} X \to \Omega^{D - 1} X$
      be the thermal conductivity.
      The \textbf{Fourier's constitutive relation}
      quantifies the above relation by using $\kappa_{D - 1}$
      as a proportionality factor:
      \begin{equation}
        q^{D - 1}
        = - \kappa_{D - 1} d_D^\star \tilde{u}^D
        = - \kappa_{D - 1} d_D^\star \star_0 u^0
        = (-1)^{D - 1} \kappa_{D - 1} \star_1 d_0 u^0
        = (-1)^{D - 1} \star_1 \tilde{\kappa}_1 d_0 u^0,
      \end{equation}
      where we have denoted
      \begin{equation}
        \tilde{\kappa}_1
        :=\star_1^{-1} \kappa_{D - 1} \star_1 [E L^{2 - D} T^{-1} \Theta^{-1}]
        \colon \Omega^1 X \to \Omega^1 X.
      \end{equation}
      This leads to
      \begin{equation}
        \tilde{q}^1
        = \star_{D - 1} q^{D - 1}
        = (-1)^{D - 1} \star_{D - 1} \star_1 \tilde{\kappa}_1 d_0 u^0
        = (-1)^{D - 1} (-1)^{D - 1} \tilde{\kappa}_1 d_0 u^0
        = \tilde{\kappa}_1 d_0 u^0.
      \end{equation}
      This leads to
      \begin{equation}
        \star_D d_{D - 1} q^{D - 1}
        = (-1)^{D - 1} \star_D d_{D - 1} \star_1 \tilde{\kappa}_1 d_0 u^0
        = - d^\star_1 \tilde{\kappa}_1 d_0 u^0.
      \end{equation}
  \end{itemize}
  We complete our model with boundary conditions.
  Let $\Gamma_D, \Gamma_N$ form a partition of $\partial X$
  into Dirichlet and Neumann boundary.
  \begin{itemize}
    \item
      Let $g_D^0 [\Theta] \colon I \to \Omega^0 \Gamma_D$
      be the prescribed temperature on the Dirichlet boundary $\Gamma_D$.
      The \textbf{Dirichlet boundary condition} is given by
      \begin{equation}
        \tr_{\Gamma_D, 0} u^0 := \restrict{u}{\Gamma_D} = g_D^0.
      \end{equation}
    \item
      Let $g_N^{D - 1} [E T^{-1}] \colon I \to \Omega^2 \Gamma_N$
      be the prescribed flow rate on the Neumann boundary $\Gamma_N$.
      The \textbf{Neumann boundary condition} is given by
      \begin{equation}
        \tr_{\Gamma_N, D - 1} q^{D - 1} = g_N^{D - 1}.
      \end{equation}
  \end{itemize}
\end{discussion}

\subsection{Primal strong formulation}
\subsubsection{Transient}
\input{diffusion/continuous/transient/primal_strong-formulation.tex}
\subsubsection{Steady-state}
\input{diffusion/continuous/steady_state/primal_strong-formulation.tex}
\subsection{Primal weak formulation}
\subsubsection{Transient}
\input{diffusion/continuous/transient/primal_weak-discussion.tex}
\begin{formulation}
  \label{idec/diffusion/discrete/transient/primal_weak-formulation}
  [Primal weak formulation for the transient discrete heat equation
    with discrete differential forms]
  The following formulation is a discrete version of
  \Cref{idec/diffusion/continuous/transient/primal_weak-formulation}.
  Let:
  \begin{itemize}
    \item
      Let $d$ be a positive integer (space dimension);
    \item
      $K$ be an oriented quasi-cubical \hyperref[idec:mesh:definition]{mesh} of
      dimension $d$ representing the material body;
    \item
      $[K]$ be the fundamental class of $K$;
    \item
      $t_0 \in \R$ be the initial time;
    \item
      $I = [t_0, \infty)$;
    \item
      $f^d [E T^{-1}] \colon I \to C^d K$ be the heat source;
    \item
      $u_0 [\Theta] \in C^0 K$ be the initial temperature;
    \item
      $\pi_0 [E L^{-d} \Theta^{-1}] \colon I \times C^0 K \to C^0 K$
      be the heat capacity of the material;
    \item
      $\tilde{\kappa}_1 [E L^{2 - d} T^{-1} \Theta^{-1}]
      \colon I \times C^1 K \to C^1 K$
      be the thermal conductivity of the material, such that at any moment
      $t \in I$ and for any edge $c_1 \in K_1$, there exists some $\lambda > 0$
      such that $\tilde{\kappa}_1(c^1) = \lambda c^1$;
    \item
      $\partial K = \Gamma_D \cup \Gamma_N$ be the partition of the boundary of
      $K$ into Dirichlet ($\Gamma_D$) and Neumann ($\Gamma_N$) regions;
    \item
      $[\Gamma_N]$ be the fundamental class of $\Gamma_N$, where $\Gamma_N$
      has the boundary orientation induced from $K$;
    \item
      $g_D^0 [\Theta] \colon I \to C^0 \Gamma_D$
      be the prescribed temperature on the Dirichlet boundary;
    \item
      $g_N^{d - 1} [E T^{-1}] \colon I \to C^{d - 1} \Gamma_N$
      be the prescribed flow rate on the Neumann boundary.
  \end{itemize}
  define the following operators:
  \begin{subequations}
    \begin{alignat}{3}
      & A \colon C^0 K \times (I \to C^0 K) \to \R, \quad
      && A(v^0, w^0)
        := \inner{\delta_0 v^0}{\tilde{\kappa}_1 \delta_0 w^0}_{K, 0} \qquad
      && [E T^{-1} \Theta^{-1}], \\
%
      & B \colon C^0 K \times (I \to C^0 K) \to \R, \quad
      && B(v^0, w^0) := \inner{v^0}{\pi_0 w^0}_{K, 0} \qquad
      && [E \Theta^{-1}], \\
%
      & G \colon C^0 K \to \R, \quad
      && G(v^0) := (\tr_{\Gamma_N} v^0 \smile g_N^{d - 1})[\Gamma_N] \qquad
      && [E T^{-1}], \\
%
      & F \colon C^0 K \to \R, \quad
      && F(v^0) := (v^0 \smile f^d)[K] \qquad
      && [E T^{-1}].
    \end{alignat}
  \end{subequations}
  Our unknowns is temperature $u^0 [\Theta] \colon I \to C^0 K$.
  We are solving the following problem for $u^0$:
  \begin{subequations}
    \begin{alignat}{4}
      & \forall v^0 [\Theta] \in \Ker \tr_{\Gamma_D, 0}, \quad
      && B(v^0, \frac{\partial u^0} {\partial t}) + A(v^0, u^0)
      && = G(v^0) + F(v^0) \qquad
      && [E T^{-1} \Theta], \\
%
      &
      && \tr_{\Gamma_D, 0} u^0
      && = g_D^0 \qquad
      && [\Theta], \\
%
      &
      && u^0(t_0)
      && = u_0 \qquad
      && [\Theta].
    \end{alignat}
  \end{subequations}
  The flow rate $q^{d - 1} [E T^{-1}] \colon I \to C^{d - 1} K$
  is calculated in the post-processing phase by the formula
  \begin{equation}
    q^{d - 1}(t, c_{d - 1}) =
    \begin{cases}
      ((-1)^{d - 1} \star_1 \tilde{\kappa}_1 \delta_0 u^0)(t, c_{d - 1}),
        & c_{d - 1} \in K_{d - 1} \setminus (\Gamma_N)_{d - 1} \\
      g_N^{d - 1}(t, c_{d - 1}), & c_{d - 1} \in (\Gamma_N)_{d - 1}
    \end{cases},\ t \in I.
  \end{equation}
\end{formulation}

\subsubsection{Steady-state}
\begin{formulation}
  \label{idec/diffusion/discrete/transient/primal_weak-formulation}
  [Primal weak formulation for the transient discrete heat equation
    with discrete differential forms]
  The following formulation is a discrete version of
  \Cref{idec/diffusion/continuous/transient/primal_weak-formulation}.
  Let:
  \begin{itemize}
    \item
      Let $d$ be a positive integer (space dimension);
    \item
      $K$ be an oriented quasi-cubical \hyperref[idec:mesh:definition]{mesh} of
      dimension $d$ representing the material body;
    \item
      $[K]$ be the fundamental class of $K$;
    \item
      $t_0 \in \R$ be the initial time;
    \item
      $I = [t_0, \infty)$;
    \item
      $f^d [E T^{-1}] \colon I \to C^d K$ be the heat source;
    \item
      $u_0 [\Theta] \in C^0 K$ be the initial temperature;
    \item
      $\pi_0 [E L^{-d} \Theta^{-1}] \colon I \times C^0 K \to C^0 K$
      be the heat capacity of the material;
    \item
      $\tilde{\kappa}_1 [E L^{2 - d} T^{-1} \Theta^{-1}]
      \colon I \times C^1 K \to C^1 K$
      be the thermal conductivity of the material, such that at any moment
      $t \in I$ and for any edge $c_1 \in K_1$, there exists some $\lambda > 0$
      such that $\tilde{\kappa}_1(c^1) = \lambda c^1$;
    \item
      $\partial K = \Gamma_D \cup \Gamma_N$ be the partition of the boundary of
      $K$ into Dirichlet ($\Gamma_D$) and Neumann ($\Gamma_N$) regions;
    \item
      $[\Gamma_N]$ be the fundamental class of $\Gamma_N$, where $\Gamma_N$
      has the boundary orientation induced from $K$;
    \item
      $g_D^0 [\Theta] \colon I \to C^0 \Gamma_D$
      be the prescribed temperature on the Dirichlet boundary;
    \item
      $g_N^{d - 1} [E T^{-1}] \colon I \to C^{d - 1} \Gamma_N$
      be the prescribed flow rate on the Neumann boundary.
  \end{itemize}
  define the following operators:
  \begin{subequations}
    \begin{alignat}{3}
      & A \colon C^0 K \times (I \to C^0 K) \to \R, \quad
      && A(v^0, w^0)
        := \inner{\delta_0 v^0}{\tilde{\kappa}_1 \delta_0 w^0}_{K, 0} \qquad
      && [E T^{-1} \Theta^{-1}], \\
%
      & B \colon C^0 K \times (I \to C^0 K) \to \R, \quad
      && B(v^0, w^0) := \inner{v^0}{\pi_0 w^0}_{K, 0} \qquad
      && [E \Theta^{-1}], \\
%
      & G \colon C^0 K \to \R, \quad
      && G(v^0) := (\tr_{\Gamma_N} v^0 \smile g_N^{d - 1})[\Gamma_N] \qquad
      && [E T^{-1}], \\
%
      & F \colon C^0 K \to \R, \quad
      && F(v^0) := (v^0 \smile f^d)[K] \qquad
      && [E T^{-1}].
    \end{alignat}
  \end{subequations}
  Our unknowns is temperature $u^0 [\Theta] \colon I \to C^0 K$.
  We are solving the following problem for $u^0$:
  \begin{subequations}
    \begin{alignat}{4}
      & \forall v^0 [\Theta] \in \Ker \tr_{\Gamma_D, 0}, \quad
      && B(v^0, \frac{\partial u^0} {\partial t}) + A(v^0, u^0)
      && = G(v^0) + F(v^0) \qquad
      && [E T^{-1} \Theta], \\
%
      &
      && \tr_{\Gamma_D, 0} u^0
      && = g_D^0 \qquad
      && [\Theta], \\
%
      &
      && u^0(t_0)
      && = u_0 \qquad
      && [\Theta].
    \end{alignat}
  \end{subequations}
  The flow rate $q^{d - 1} [E T^{-1}] \colon I \to C^{d - 1} K$
  is calculated in the post-processing phase by the formula
  \begin{equation}
    q^{d - 1}(t, c_{d - 1}) =
    \begin{cases}
      ((-1)^{d - 1} \star_1 \tilde{\kappa}_1 \delta_0 u^0)(t, c_{d - 1}),
        & c_{d - 1} \in K_{d - 1} \setminus (\Gamma_N)_{d - 1} \\
      g_N^{d - 1}(t, c_{d - 1}), & c_{d - 1} \in (\Gamma_N)_{d - 1}
    \end{cases},\ t \in I.
  \end{equation}
\end{formulation}

\subsection{Mixed weak formulation}
\subsubsection{Transient}
\input{diffusion/continuous/transient/mixed_weak-discussion.tex}
\begin{formulation}
  Let
  \begin{itemize}
    \item
      $M$ be a manifold-like flat mesh of dimension $3$;
    \item
      $K$ be the Forman subdivision of $M$;
    \item
      $K' := K \setminus \partial K$ be the interior of $K$;
    \item
      $t_0 [T]\in \R$ be the initial time;
    \item
      $I := [t_0, \infty)$
    \item
      $u_0 [\Theta] \in C^0 K$ be the initial temperature;
    \item
      $f [E T^{-1}] \colon I \to C^3 K'$ be the heat source;
    \item
      $\Gamma_D, \Gamma_N$ form a partition of $(\partial K)$;
    \item
      $g_D [\Theta] \colon I \to C^0 \Gamma_D$
      be the Dirichlet boundary condition;
    \item
      $g_N [E T^{-1}] \colon I \to C^2 \Gamma_N$
      be the Neumann boundary condition;
    \item
      $\pi_0 [E L^{-3} \Theta^{-1}] \colon C^0 K \to C^0 K$
      be the volumetric heat capacity
      (its matrix in the standard basis is diagonal);
    \item
      $\pi_2 [E L^{-1} T^{-1} \Theta^{-1}] \colon C^2 K \to C^2 K$
      be the thermal conductivity
      (its matrix in the standard basis is diagonal).
  \end{itemize}
  We are solving the following problem:
  \begin{equation}
    \begin{split}
      & \text{Find $u [\Theta] \colon I \to C^0 K$ and
        $q [E T^{-1}] \colon I \to C^2 K$ such that} \\
      &
      \begin{cases}
        \forall r [E T^{-1}]\in \Ker \tr_{\Gamma_N, 2},\
          \inner{r}{\pi_2^{-1} q}_{K, 2} + (u \smile \delta_2 r)[K]
          = (\tr_{\Gamma_N, 2} r \smile g_D)[\Gamma_D] & [E T^{-1} \Theta], \\
        \forall v [\Theta]\in C^0 K,\
          (v \smile \delta_2 q)[K] - \frac{d}{d t} \inner{v}{\pi_0 u}_{K, 0}
          = - (v \smile f)[K] & [E T^{-1} \Theta], \\
        u(t_0, \cdot) = u_0 & [\Theta], \\
        \tr_{\Gamma_N, 2} q = g_N & [E T^{-1}].
      \end{cases}
    \end{split}
  \end{equation}
  In the steady-state model we do not have the initial temperature $u$
  and all quantities of type $f \colon I \to X$ become of type $f \in X$.
  It reads as follows:
  \begin{equation}
    \begin{split}
      & \text{Find $u [\Theta] \in C^0 K$ and
        $q [E T^{-1}] \in C^2 K$ such that} \\
      &
      \begin{cases}
        \forall r [E T^{-1}]\in \Ker \tr_{\Gamma_N, 2},\
          \inner{r}{\pi_2^{-1} q}_{K, 2} + (u \smile \delta_2 r)[K]
          = (\tr_{\Gamma_N, 2} r \smile g_D)[\Gamma_D] & [E T^{-1} \Theta], \\
        \forall v [\Theta]\in C^0 K,\
          (v \smile \delta_2 q)[K] = - (v \smile f)[K] & [E T^{-1} \Theta], \\
        u(t_0, \cdot) = u_0 & [\Theta], \\
        \tr_{\Gamma_N, 2} q = g_N & [E T^{-1}].
      \end{cases}
    \end{split}
  \end{equation}
  In both cases in the post-processing phase we have:
  \begin{subequations}
    \begin{alignat}{2}
      & \tilde{Q} && := \pi_0 u, \\
      & Q && \approx \star_0 \tilde{Q}.
    \end{alignat}
  \end{subequations}
\end{formulation}

\subsubsection{Steady-state}
\begin{formulation}
  Let
  \begin{itemize}
    \item
      $M$ be a manifold-like flat mesh of dimension $3$;
    \item
      $K$ be the Forman subdivision of $M$;
    \item
      $K' := K \setminus \partial K$ be the interior of $K$;
    \item
      $t_0 [T]\in \R$ be the initial time;
    \item
      $I := [t_0, \infty)$
    \item
      $u_0 [\Theta] \in C^0 K$ be the initial temperature;
    \item
      $f [E T^{-1}] \colon I \to C^3 K'$ be the heat source;
    \item
      $\Gamma_D, \Gamma_N$ form a partition of $(\partial K)$;
    \item
      $g_D [\Theta] \colon I \to C^0 \Gamma_D$
      be the Dirichlet boundary condition;
    \item
      $g_N [E T^{-1}] \colon I \to C^2 \Gamma_N$
      be the Neumann boundary condition;
    \item
      $\pi_0 [E L^{-3} \Theta^{-1}] \colon C^0 K \to C^0 K$
      be the volumetric heat capacity
      (its matrix in the standard basis is diagonal);
    \item
      $\pi_2 [E L^{-1} T^{-1} \Theta^{-1}] \colon C^2 K \to C^2 K$
      be the thermal conductivity
      (its matrix in the standard basis is diagonal).
  \end{itemize}
  We are solving the following problem:
  \begin{equation}
    \begin{split}
      & \text{Find $u [\Theta] \colon I \to C^0 K$ and
        $q [E T^{-1}] \colon I \to C^2 K$ such that} \\
      &
      \begin{cases}
        \forall r [E T^{-1}]\in \Ker \tr_{\Gamma_N, 2},\
          \inner{r}{\pi_2^{-1} q}_{K, 2} + (u \smile \delta_2 r)[K]
          = (\tr_{\Gamma_N, 2} r \smile g_D)[\Gamma_D] & [E T^{-1} \Theta], \\
        \forall v [\Theta]\in C^0 K,\
          (v \smile \delta_2 q)[K] - \frac{d}{d t} \inner{v}{\pi_0 u}_{K, 0}
          = - (v \smile f)[K] & [E T^{-1} \Theta], \\
        u(t_0, \cdot) = u_0 & [\Theta], \\
        \tr_{\Gamma_N, 2} q = g_N & [E T^{-1}].
      \end{cases}
    \end{split}
  \end{equation}
  In the steady-state model we do not have the initial temperature $u$
  and all quantities of type $f \colon I \to X$ become of type $f \in X$.
  It reads as follows:
  \begin{equation}
    \begin{split}
      & \text{Find $u [\Theta] \in C^0 K$ and
        $q [E T^{-1}] \in C^2 K$ such that} \\
      &
      \begin{cases}
        \forall r [E T^{-1}]\in \Ker \tr_{\Gamma_N, 2},\
          \inner{r}{\pi_2^{-1} q}_{K, 2} + (u \smile \delta_2 r)[K]
          = (\tr_{\Gamma_N, 2} r \smile g_D)[\Gamma_D] & [E T^{-1} \Theta], \\
        \forall v [\Theta]\in C^0 K,\
          (v \smile \delta_2 q)[K] = - (v \smile f)[K] & [E T^{-1} \Theta], \\
        u(t_0, \cdot) = u_0 & [\Theta], \\
        \tr_{\Gamma_N, 2} q = g_N & [E T^{-1}].
      \end{cases}
    \end{split}
  \end{equation}
  In both cases in the post-processing phase we have:
  \begin{subequations}
    \begin{alignat}{2}
      & \tilde{Q} && := \pi_0 u, \\
      & Q && \approx \star_0 \tilde{Q}.
    \end{alignat}
  \end{subequations}
\end{formulation}


\section{Discrete heat transport}
\label{section:discrete_diffusion}
\begin{notation}
  Let
    $S$ be a set,
    $T$ be a subset of $S$,
    $V$ be a real vector space,
    $u \in \Hom_\R({\rm Free}_\R(S), V)$.
  We will denote by
  \begin{equation}
    \restrict{u}{T} \in \Hom_\R({\rm Free}_\R(T), V)
  \end{equation}
  the map defined in the same way of $u$ but acting on formal linear
  combinations of the elements in $T$.
\end{notation}

\begin{notation}
  Let
    $d \in \N$,
    $K$ be a flat mesh of dimension $d$,
    $c_0 \in C_0(\partial K)$ with corresponding point $x \in \R^d$.
  By ${\bf n}_{c_0}$ we will denote the exterior unit normal at $x$ to $K$.
  When $c_{0}$ has more than one non-parallel adjacent hyperfaces
  (for instance, in $3$D, it can lie on an edge or at a corner), we will take
  some average of the normals to those faces.
  The easiest one is to sum all exterior unit normals and divide by the length
  of the sum.
  This is the approach taken in the software implementation.

  When $\Gamma \subseteq (\partial K)_0$, we will understand ${\bf n}$ as
  a function from $\Gamma$ to $\R^d$ or as a linear map in
  $\Hom({\rm Free}_\R(\Gamma), \R^d)$.
\end{notation}

\begin{discussion}
  \label{idec/diffusion/continuous/model_with_differential_forms-discussion}
  Let:
  \begin{itemize}
    \item
      $D$ be a positive integer (space dimension);
    \item
      $X$ be an open region in $\R^D$ (the space region);
    \item
      $t_0 [T] \in \R$ be the initial time;
    \item
      $I := [t_0, \infty)$;
  \end{itemize}
  The main physical quantities in our model are:
  \begin{itemize}
    \item
      temperature $u^0 [\Theta] \colon I \to \Omega^0 X$:
      for any moment $t \in I$ and any point $x \in X$,
      \begin{equation}
        \text{``temperature $[\Theta]$ on $x$ at time $t$''}
        = u^0(t)(x) := u^0(t, x);
      \end{equation}
    \item
      heat energy $Q^D [E] \colon I \to \Omega^D X$:
      for any moment $t \in I$ and any volume $V_D \subseteq X$,
      \begin{equation}
        \text{``total heat energy $[E]$ of the system on $V$ at time $t$''}
        = \int_{V_D} Q^D(t);
      \end{equation}
    \item
      heat flow rate $q^{D - 1} [E T^{-1}] \colon I \to \Omega^{D - 1} X$:
      for any time interval $[t_1, t_2] \subset I$
      and any hypersurface $S_{D - 1} \subset X$,
      \begin{equation}
        \text{``total flow $[E]$ through $S_{D - 1}$ in $[t_1, t_2]$''}
        = \int_{t_1}^{t_2}\left(\int_{S_{D - 1}} q^{D - 1}(t)\right)\, d t.
      \end{equation}
      (Here we assume that $S_{D - 1}$ is oriented.
       Let $U_D$ and $V_D$ be adjacent regions having $S_{D - 1}$ as a common
       boundary, such that $\varepsilon(U_D, S_{D - 1}) = -1$,
       $\varepsilon(V_D, S_{D - 1}) = 1$.
       Then the above equation measures the total flow from $U_D$ to $V_D$.)
  \end{itemize}
  We will also need the dual variables of heat energy, temperature, and
  flow rate.
  \begin{itemize}
    \item
      dual temperature $\tilde{u}^D [\Theta L^D] \colon I \to \Omega^D X$
      defined by
      \begin{equation}
        \tilde{u}^D := \star_0 u^0
      \end{equation}
      (althiugh using non-zero based temperature scale might make $\star_0$ not
      well defined, this will not cause problems as we will always take
      temperature differences when substituting in equations);
    \item
      heat energy density $\tilde{Q}^0 [E L^{-D}] \colon I \to \Omega^0 X$
      defined by
      \begin{equation}
        \tilde{Q}^0 := \star_D Q^D;
      \end{equation}
    \item
      dual flow rate
      $\tilde{q}^1 [E T L^{2 - D}] \colon I \to \Omega^1 X$
      defined by
      \begin{equation}
        \tilde{q} := \star_{D - 1} q^{D - 1};
      \end{equation}
  \end{itemize}
  The governing laws are formulated as follows.
  \begin{itemize}
    \item
      Let $f^D [E T^{-1}] \colon I \to \Omega^D X$ be an external heat source:
      for any time interval $[t_1, t_2] \subset I$
      and any volume $V_D \subseteq X$,
      \begin{equation}
        \text{``total net heat production $[E]$ in $V_D$ in $[t_1, t_2]$''}
        = \int_{t_1}^{t_2} \left(\int_{V_D} f^D(t) \right)\, d t.
      \end{equation}
      \textbf{Conservation of heat energy} is given by the following relation:
      for any time interval $[t_1, t_2] \subset I$
      and any volume $V_D \subseteq X$,
      \begin{equation}
        \begin{split}
        \text{``heat difference on $V_D$ between moments $t_2$ and $t_1$''}
        & =
          \text{``heat inflow through the boundary of $V_D$
          in $[t_1, t_2]$''} \\
        & +
          \text{``heat production inside $V_D$ in $[t_1, t_2]$''}.
        \end{split}
      \end{equation}
      In symbolic terms, the last equation is written as
      \begin{equation}
        \int_{V_D} (Q^D(t_2) - Q^D(t_1))
        = \int_{t_1}^{t_2}
          \left(\int_{\partial V_D} q^{D - 1}(t) \right)\, d t
        + \int_{t_1}^{t_2} \left(\int_{V_D} f^D(t) \right)\, d t.
      \end{equation}
      Using Stokes' theorem twice, we get the equation
      \begin{equation}
        \int_{t_1}^{t_2}
          \left(\int_{V_D} \frac{\partial Q^D}{\partial t}\right)\, d t =
          \int_{t_1}^{t_2} \left(\int_{V_D} d_{D - 1} q^{D - 1} \right)\, d t
        + \int_{t_1}^{t_2} \left(\int_{V_D} f^D \right)\, d t.
      \end{equation}
      Since the time interval $[t_1, t_2]$ and the volume $V_D$ are arbitrary,
      we can drop integrals and arrive at the differential form
      \begin{equation}
        \frac{\partial Q^D}{\partial t} = d_{D - 1} q^{D - 1} + f^D.
      \end{equation}
    \item
      Let
        $u_0 [\Theta] \in \Omega^0 X$ be the initial temperature.
      The \textbf{initial condition} is the prescribed initial temperature:
      \begin{equation}
        u^0(t_0) = u_0.
      \end{equation}
    \item
      Let $\pi_0 [E L^{-D} \Theta^{-1}] \colon \Omega^0 X \to \Omega^0 X$
      be the volumetric heat capacity.
      The \textbf{relation between temperature change and heat energy change}
      is given by
      \begin{equation}
        \frac{\partial Q^D}{\partial t}
        = \star_0 \left(\frac{\partial \tilde{Q}^0}{\partial t}\right)
        = \star_0 \left(\pi_0 \frac{\partial u^0}{\partial t}\right).
      \end{equation}
    \item
      Consider two adjacent volumes $U_D$ and $V_D$
      with a common surface $S_{D - 1}$, such that
      $\varepsilon(U_D, S_{D - 1}) = -1$ and
      $\varepsilon(V_D, S_{D - 1}) = 1$.
      According to the second law of thermodynamics, heat flows from regions of
      higher temperature to regions of lower temperatures.
      Therefore, the net flow through $S_{D - 1}$ is in the negative direction
      of the temperature difference between $U_D$ and $V_D$.

      Let
      $\kappa_{D - 1} [E L^{2 - D} T^{-1} \Theta^{-1}]
      \colon \Omega^{D - 1} X \to \Omega^{D - 1} X$
      be the thermal conductivity.
      The \textbf{Fourier's constitutive relation}
      quantifies the above relation by using $\kappa_{D - 1}$
      as a proportionality factor:
      \begin{equation}
        q^{D - 1}
        = - \kappa_{D - 1} d_D^\star \tilde{u}^D
        = - \kappa_{D - 1} d_D^\star \star_0 u^0
        = (-1)^{D - 1} \kappa_{D - 1} \star_1 d_0 u^0
        = (-1)^{D - 1} \star_1 \tilde{\kappa}_1 d_0 u^0,
      \end{equation}
      where we have denoted
      \begin{equation}
        \tilde{\kappa}_1
        :=\star_1^{-1} \kappa_{D - 1} \star_1 [E L^{2 - D} T^{-1} \Theta^{-1}]
        \colon \Omega^1 X \to \Omega^1 X.
      \end{equation}
      This leads to
      \begin{equation}
        \tilde{q}^1
        = \star_{D - 1} q^{D - 1}
        = (-1)^{D - 1} \star_{D - 1} \star_1 \tilde{\kappa}_1 d_0 u^0
        = (-1)^{D - 1} (-1)^{D - 1} \tilde{\kappa}_1 d_0 u^0
        = \tilde{\kappa}_1 d_0 u^0.
      \end{equation}
      This leads to
      \begin{equation}
        \star_D d_{D - 1} q^{D - 1}
        = (-1)^{D - 1} \star_D d_{D - 1} \star_1 \tilde{\kappa}_1 d_0 u^0
        = - d^\star_1 \tilde{\kappa}_1 d_0 u^0.
      \end{equation}
  \end{itemize}
  We complete our model with boundary conditions.
  Let $\Gamma_D, \Gamma_N$ form a partition of $\partial X$
  into Dirichlet and Neumann boundary.
  \begin{itemize}
    \item
      Let $g_D^0 [\Theta] \colon I \to \Omega^0 \Gamma_D$
      be the prescribed temperature on the Dirichlet boundary $\Gamma_D$.
      The \textbf{Dirichlet boundary condition} is given by
      \begin{equation}
        \tr_{\Gamma_D, 0} u^0 := \restrict{u}{\Gamma_D} = g_D^0.
      \end{equation}
    \item
      Let $g_N^{D - 1} [E T^{-1}] \colon I \to \Omega^2 \Gamma_N$
      be the prescribed flow rate on the Neumann boundary $\Gamma_N$.
      The \textbf{Neumann boundary condition} is given by
      \begin{equation}
        \tr_{\Gamma_N, D - 1} q^{D - 1} = g_N^{D - 1}.
      \end{equation}
  \end{itemize}
\end{discussion}

\subsection{Primal strong formulation}
\subsubsection{Steady-state}
\begin{formulation}
  \label{idec/diffusion/discrete/steady_state/primal_strong_with_normals-discussion}
  [Primal strong formulation for the steady-state discrete heat equation
    with discrete differential forms]
  Let:
  \begin{itemize}
    \item
      $d \in \N$;
    \item
      $M$ be a manifold-like flat mesh of dimension $d$;
    \item
      $K$ be the Forman subdivision of $M$;
    \item
      $K' := K \setminus \partial K$ be the interior of $K$;
    \item
      $f \in C^0 K'$ be the external heat source, $f\ [E L^{-3} T^{-1}]$;
    \item
      $\Gamma_D, \Gamma_N$ form a partition of $(\partial K)_0$;
    \item
      ${\bf n} \colon \Gamma_N \to \R^d$
      be the generalized exterior unit normal, ${\bf n}\ [1]$;
    \item
      $g_D [\Theta]\in ({\rm Free}_\R(\Gamma_D))^*$
      be the Dirichlet boundary condition;
    \item
      $g_N [E L^{-1} T^{-1}] \in ({\rm Free}_\R(\Gamma_N))^*$
      be the Neumann boundary condition;
    \item
      $\pi_0 [E L^{-3} \Theta^{-1}] \colon C^0 K \to C^0 K$ be a material
      property of the nodes of $K$
      (its matrix in the standard basis is diagonal);
    \item
      $\kappa_1 [E L^{-1} T^{-1} \Theta^{-1}] \colon C^1 K \to C^1 K$ be a
      material property of the edges of $K$
      (its matrix in the standard basis is diagonal).
  \end{itemize}
  We are solving the following problem.
  \begin{equation}
    \begin{split}
      & \text{Find $u [\Theta] \in C^0 K$, such that} \\
      &
      \begin{cases}
        \restrict{((\delta_1^\star \circ \kappa_1 \circ \delta_0) u)}{K'_0} + f
        = 0
        & (\text{balance of heat energy},\ [E L^{-3} T^{-1}]), \\
        %
        \restrict{u}{\Gamma_D} = g_D
        & (\text{Dirichlet boundary condition},\ [\Theta]), \\
        %
        \restrict{\overline{(\kappa_1 \circ \delta_0) u}}{\Gamma_N}
        \cdot {\bf n} = g_N
        & (\text{Neumann boundary condition},\ [E L^{-1} T^{-1}]).
      \end{cases}
    \end{split}
  \end{equation}
\end{formulation}

\subsubsection{Transient}
\input{diffusion/discrete/transient/primal_strong_with_normals-formulation.tex}
\input{diffusion/discrete/transient/primal_strong_with_normals_solve_trapezoidal-discussion.tex}
\subsection{Primal weak formulation}
\subsubsection{Steady-state}
\begin{formulation}
  \label{idec/diffusion/discrete/transient/primal_weak-formulation}
  [Primal weak formulation for the transient discrete heat equation
    with discrete differential forms]
  The following formulation is a discrete version of
  \Cref{idec/diffusion/continuous/transient/primal_weak-formulation}.
  Let:
  \begin{itemize}
    \item
      Let $d$ be a positive integer (space dimension);
    \item
      $K$ be an oriented quasi-cubical \hyperref[idec:mesh:definition]{mesh} of
      dimension $d$ representing the material body;
    \item
      $[K]$ be the fundamental class of $K$;
    \item
      $t_0 \in \R$ be the initial time;
    \item
      $I = [t_0, \infty)$;
    \item
      $f^d [E T^{-1}] \colon I \to C^d K$ be the heat source;
    \item
      $u_0 [\Theta] \in C^0 K$ be the initial temperature;
    \item
      $\pi_0 [E L^{-d} \Theta^{-1}] \colon I \times C^0 K \to C^0 K$
      be the heat capacity of the material;
    \item
      $\tilde{\kappa}_1 [E L^{2 - d} T^{-1} \Theta^{-1}]
      \colon I \times C^1 K \to C^1 K$
      be the thermal conductivity of the material, such that at any moment
      $t \in I$ and for any edge $c_1 \in K_1$, there exists some $\lambda > 0$
      such that $\tilde{\kappa}_1(c^1) = \lambda c^1$;
    \item
      $\partial K = \Gamma_D \cup \Gamma_N$ be the partition of the boundary of
      $K$ into Dirichlet ($\Gamma_D$) and Neumann ($\Gamma_N$) regions;
    \item
      $[\Gamma_N]$ be the fundamental class of $\Gamma_N$, where $\Gamma_N$
      has the boundary orientation induced from $K$;
    \item
      $g_D^0 [\Theta] \colon I \to C^0 \Gamma_D$
      be the prescribed temperature on the Dirichlet boundary;
    \item
      $g_N^{d - 1} [E T^{-1}] \colon I \to C^{d - 1} \Gamma_N$
      be the prescribed flow rate on the Neumann boundary.
  \end{itemize}
  define the following operators:
  \begin{subequations}
    \begin{alignat}{3}
      & A \colon C^0 K \times (I \to C^0 K) \to \R, \quad
      && A(v^0, w^0)
        := \inner{\delta_0 v^0}{\tilde{\kappa}_1 \delta_0 w^0}_{K, 0} \qquad
      && [E T^{-1} \Theta^{-1}], \\
%
      & B \colon C^0 K \times (I \to C^0 K) \to \R, \quad
      && B(v^0, w^0) := \inner{v^0}{\pi_0 w^0}_{K, 0} \qquad
      && [E \Theta^{-1}], \\
%
      & G \colon C^0 K \to \R, \quad
      && G(v^0) := (\tr_{\Gamma_N} v^0 \smile g_N^{d - 1})[\Gamma_N] \qquad
      && [E T^{-1}], \\
%
      & F \colon C^0 K \to \R, \quad
      && F(v^0) := (v^0 \smile f^d)[K] \qquad
      && [E T^{-1}].
    \end{alignat}
  \end{subequations}
  Our unknowns is temperature $u^0 [\Theta] \colon I \to C^0 K$.
  We are solving the following problem for $u^0$:
  \begin{subequations}
    \begin{alignat}{4}
      & \forall v^0 [\Theta] \in \Ker \tr_{\Gamma_D, 0}, \quad
      && B(v^0, \frac{\partial u^0} {\partial t}) + A(v^0, u^0)
      && = G(v^0) + F(v^0) \qquad
      && [E T^{-1} \Theta], \\
%
      &
      && \tr_{\Gamma_D, 0} u^0
      && = g_D^0 \qquad
      && [\Theta], \\
%
      &
      && u^0(t_0)
      && = u_0 \qquad
      && [\Theta].
    \end{alignat}
  \end{subequations}
  The flow rate $q^{d - 1} [E T^{-1}] \colon I \to C^{d - 1} K$
  is calculated in the post-processing phase by the formula
  \begin{equation}
    q^{d - 1}(t, c_{d - 1}) =
    \begin{cases}
      ((-1)^{d - 1} \star_1 \tilde{\kappa}_1 \delta_0 u^0)(t, c_{d - 1}),
        & c_{d - 1} \in K_{d - 1} \setminus (\Gamma_N)_{d - 1} \\
      g_N^{d - 1}(t, c_{d - 1}), & c_{d - 1} \in (\Gamma_N)_{d - 1}
    \end{cases},\ t \in I.
  \end{equation}
\end{formulation}

\begin{discussion}
  \label{idec/diffusion/discrete/steady_state/primal_weak_solve-discussion}
  We are going to derive a solution to
  \Cref{idec/diffusion/discrete/steady_state/primal_weak-formulation}.
  For any $p \in \{0, ..., d\}$ denote
  \begin{equation}
    n_p := \abs{K_p} = \dim(C_p K) = \dim(C^p K).
  \end{equation}
  The cochains $(N^0, ..., N^{n_0 - 1})$ form the standard basis of $C^0 K$.
  Define the matrix $A \in M_{n_0 \times n_0}(\R)$ by
  \begin{equation}
    A_{i, j} := \inner{\delta N^i}{\tilde{\kappa}_1 \delta_0 N^i}_{K, 1}, \
    i, j = 0, ..., n_0 - 1,
  \end{equation}
  and the vectors $b, b^N, b^f \in \R^{n_0}$ by
  \begin{subequations}
    \begin{alignat}{3}
      & b_i
      && := (\tr_{\Gamma_N, 0} N^i \smile g_N^{d - 1})[\Gamma_N], \enspace
      && i = 0, ..., n_0 - 1, \\
%
      & b^f_i
      && := (N^i \smile f^d)[K], \enspace
      && i = 0, ..., n_0 - 1, \\
%
      & b
      && := b^N + b^f.
      &&
    \end{alignat}
  \end{subequations}
  Denote the unknown coefficients of $u^0$ as $\{x_j\}_{j = 0}^{n_0 - 1}$, i.e.,
  \begin{equation}
    u^0 = \sum_{j = 0}^{n_0 - 1} x_j N^j
  \end{equation}
  Finally, let $I$ be the set of nodes on the Dirichlet boundary $\Gamma_D$, and
  $J := \{0, ..., n_0 - 1\} \setminus I$.
  We get the system
  \begin{subequations}
    \begin{alignat}{3}
      & \sum_{j = 0}^{n_0 - 1} A_{i, j} x_j
      && = b_i, \enspace
      && i \in J, \\
%
      & x_i
      && = g_D^0(N_i), \enspace
      && i \in I.
    \end{alignat}
  \end{subequations}
  This leads to the system of equations
  \begin{equation}
    \sum_{j \in J} A_{i, j} x_j = b_i - \sum_{j \in I} A_{i, j} g_D^0(N_j),\
    i \in J.
  \end{equation}
  If we denote by $A^J$ the restriction of $A$ to the rows and columns in $J$,
  by $b^J$ the right-hand side of the above equation (again only for the
  indices in J), and by $x^J$ the restriction of $x$ on indices in $J$,
  we arrive at the final linear system with positive-definite matrix $A^J$:
  \begin{equation}
    A^J x^J = b^J.
  \end{equation}
  After solving it, we get the final solution
  \begin{equation}
    x_i =
    \begin{cases}
      x^J_i, & i \in J \\
      g_D^0(N_i), & i \in I
    \end{cases}.
  \end{equation}
\end{discussion}

\subsubsection{Transient}
\begin{formulation}
  \label{idec/diffusion/discrete/transient/primal_weak-formulation}
  [Primal weak formulation for the transient discrete heat equation
    with discrete differential forms]
  The following formulation is a discrete version of
  \Cref{idec/diffusion/continuous/transient/primal_weak-formulation}.
  Let:
  \begin{itemize}
    \item
      Let $d$ be a positive integer (space dimension);
    \item
      $K$ be an oriented quasi-cubical \hyperref[idec:mesh:definition]{mesh} of
      dimension $d$ representing the material body;
    \item
      $[K]$ be the fundamental class of $K$;
    \item
      $t_0 \in \R$ be the initial time;
    \item
      $I = [t_0, \infty)$;
    \item
      $f^d [E T^{-1}] \colon I \to C^d K$ be the heat source;
    \item
      $u_0 [\Theta] \in C^0 K$ be the initial temperature;
    \item
      $\pi_0 [E L^{-d} \Theta^{-1}] \colon I \times C^0 K \to C^0 K$
      be the heat capacity of the material;
    \item
      $\tilde{\kappa}_1 [E L^{2 - d} T^{-1} \Theta^{-1}]
      \colon I \times C^1 K \to C^1 K$
      be the thermal conductivity of the material, such that at any moment
      $t \in I$ and for any edge $c_1 \in K_1$, there exists some $\lambda > 0$
      such that $\tilde{\kappa}_1(c^1) = \lambda c^1$;
    \item
      $\partial K = \Gamma_D \cup \Gamma_N$ be the partition of the boundary of
      $K$ into Dirichlet ($\Gamma_D$) and Neumann ($\Gamma_N$) regions;
    \item
      $[\Gamma_N]$ be the fundamental class of $\Gamma_N$, where $\Gamma_N$
      has the boundary orientation induced from $K$;
    \item
      $g_D^0 [\Theta] \colon I \to C^0 \Gamma_D$
      be the prescribed temperature on the Dirichlet boundary;
    \item
      $g_N^{d - 1} [E T^{-1}] \colon I \to C^{d - 1} \Gamma_N$
      be the prescribed flow rate on the Neumann boundary.
  \end{itemize}
  define the following operators:
  \begin{subequations}
    \begin{alignat}{3}
      & A \colon C^0 K \times (I \to C^0 K) \to \R, \quad
      && A(v^0, w^0)
        := \inner{\delta_0 v^0}{\tilde{\kappa}_1 \delta_0 w^0}_{K, 0} \qquad
      && [E T^{-1} \Theta^{-1}], \\
%
      & B \colon C^0 K \times (I \to C^0 K) \to \R, \quad
      && B(v^0, w^0) := \inner{v^0}{\pi_0 w^0}_{K, 0} \qquad
      && [E \Theta^{-1}], \\
%
      & G \colon C^0 K \to \R, \quad
      && G(v^0) := (\tr_{\Gamma_N} v^0 \smile g_N^{d - 1})[\Gamma_N] \qquad
      && [E T^{-1}], \\
%
      & F \colon C^0 K \to \R, \quad
      && F(v^0) := (v^0 \smile f^d)[K] \qquad
      && [E T^{-1}].
    \end{alignat}
  \end{subequations}
  Our unknowns is temperature $u^0 [\Theta] \colon I \to C^0 K$.
  We are solving the following problem for $u^0$:
  \begin{subequations}
    \begin{alignat}{4}
      & \forall v^0 [\Theta] \in \Ker \tr_{\Gamma_D, 0}, \quad
      && B(v^0, \frac{\partial u^0} {\partial t}) + A(v^0, u^0)
      && = G(v^0) + F(v^0) \qquad
      && [E T^{-1} \Theta], \\
%
      &
      && \tr_{\Gamma_D, 0} u^0
      && = g_D^0 \qquad
      && [\Theta], \\
%
      &
      && u^0(t_0)
      && = u_0 \qquad
      && [\Theta].
    \end{alignat}
  \end{subequations}
  The flow rate $q^{d - 1} [E T^{-1}] \colon I \to C^{d - 1} K$
  is calculated in the post-processing phase by the formula
  \begin{equation}
    q^{d - 1}(t, c_{d - 1}) =
    \begin{cases}
      ((-1)^{d - 1} \star_1 \tilde{\kappa}_1 \delta_0 u^0)(t, c_{d - 1}),
        & c_{d - 1} \in K_{d - 1} \setminus (\Gamma_N)_{d - 1} \\
      g_N^{d - 1}(t, c_{d - 1}), & c_{d - 1} \in (\Gamma_N)_{d - 1}
    \end{cases},\ t \in I.
  \end{equation}
\end{formulation}

\input{diffusion/discrete/transient/primal_weak_solve_trapezoidal-discussion.tex}
\begin{algorithm}[Algorithm for solving the transient primal weak formulation
  for the discrete heat transfer phenomenon using trapezoidal rule for time
  integration, assuming time-independent input data]
  \label{idec/diffusion/discrete/transient/primal_weak_solve_trapezoidal-algorithm}
  Let:
  \begin{itemize}
    \item
      Let $d$ be a positive integer (space dimension);
    \item
      $K$ be an oriented quasi-cubical \hyperref[idec:mesh:definition]{mesh} of
      dimension $d$ representing the material body;
    \item
      $[K]$ be the fundamental class of $K$;
    \item
      $t_0 [T] \in \R$ be the initial time;
    \item
      $\tau [T] \in \R^+$ be the time step;
    \item
      $f^d [E T^{-1}] \in C^d K$ be the heat source;
    \item
      $u_0 [\Theta] \in C^0 K$ be the initial temperature;
    \item
      $\pi_0 [E L^{-d} \Theta^{-1}] \colon I \times C^0 K \to C^0 K$
      be the heat capacity of the material;
    \item
      $\tilde{\kappa}_1 [E L^{2 - d} T^{-1} \Theta^{-1}] \colon C^1 K \to C^1 K$
      be the thermal conductivity of the material;
    \item
      $\partial K = \Gamma_D \cup \Gamma_N$ be the partition of the boundary of
      $K$ into Dirichlet ($\Gamma_D$) and Neumann ($\Gamma_N$) regions;
    \item
      $[\Gamma_N]$ be the fundamental class of $\Gamma_N$, where $\Gamma_N$
      has the boundary orientation induced from $K$;
    \item
      $g_D^0 [\Theta] \in C^0 \Gamma_D$
      be the prescribed temperature on the Dirichlet boundary;
    \item
      $g_N^{d - 1} [E T^{-1}] \in C^{d - 1} \Gamma_N$
      be the prescribed flow rate on the Neumann boundary.
  \end{itemize}
  Our algorithm has $3$ phases.
  \begin{enumerate}
    \item
      \textbf{Time-independent phase.}
      Do the following calculations:
      \begin{itemize}
        \item
          $n_0 := \abs{K_0}$;
        \item
          the sparse matrix $A \in M_{n_0 \times n_0}(\R)$,
          \begin{equation}
            A_{i, j}
            := \inner{\delta_0 N^j}{\tilde{\kappa}_1 \delta_0 N^i}_{K, 1},\
            i, j = 0, ..., n_0 - 1;
          \end{equation}
        \item
          the diagonal matrix $B \in M_{n_0 \times n_0}(\R)$,
          \begin{equation}
            B_{i, j} := \inner{N^j}{\pi_0 N^i}_{K, 0},\ i, j = 0, ..., n_0 - 1;
          \end{equation}
        \item
          the right-hand side vectors $F, G, H \in \R^{n_0}$,
          \begin{subequations}
            \begin{alignat}{1}
              & F_i := (v^0 \smile f^d)[K], i = 0, ..., n_0 - 1, \\
              & G_i := (\tr_{\Gamma_N, 0} N^i \smile g_N^{d - 1})[\Gamma_N],
                i = 0, ..., n_0 - 1, \\
              & H := F + G;
            \end{alignat}
          \end{subequations}
        \item
          the sparse matrices (having the same stencil as $A$)
          $L_\tau, R_\tau \in M_{n_0 \times n_0}(\R)$,
          \begin{subequations}
            \begin{alignat}{1}
              & L_\tau := B - \frac{\tau}{2} A, \\
              & R_\tau := B + \frac{\tau}{2} A;
            \end{alignat}
          \end{subequations}
        \item
          the sets $I_D := (\Gamma_D)_0$ and
          $\overline{I_D} := \{0, ..., n_0 - 1\} \setminus I_D$;
        \item
          the restricted matrix $\overline{L_\tau}$, constructed out of $L_\tau$
          with rows and columns in $I_D$ removed, and the restricted vector
          $\overline{c_\tau} \in \R^{\abs{\overline{I_D}}}$
          \begin{equation}
            \overline{c_\tau}_j
            := \tau H_j - \sum_{k \in I_D} (R_\tau)_{j, k} g_D^0(N_k),\
            j \in \overline{I};
          \end{equation}
        \item
          the Cholesky decomposition
          \begin{equation}
            \overline{L_\tau} = \overline{S_\tau} \overline{S_\tau}^T,
          \end{equation}
          where $\overline{S_\tau}$ is lower-triangular sparse matrix;
        \item
          the time independent part of the restricted solution
          \begin{equation}
            \overline{z_\tau}
            := \overline{L_\tau}^{-1} \overline{c_\tau}
            = \overline{S_\tau}^{-T} \overline{S_\tau}^{-1} \overline{c_\tau}
          \end{equation}
          (of course, we do not find the inverses of $\overline{S_\tau}$ and its
          transpose but apply forward and back substitution);
        \item
          the initial coordinates $x^0 \in \R^{n_0}$ of the temperature,
          \begin{equation}
            x^0_j := u_0(N_i),\ j = 0, ..., n_0 - 1.
          \end{equation}
      \end{itemize}
    \item
      \textbf{Time-dependent (loop) phase.}
      For any $i \geq 0$ (until some predefined final moment is reached or some
      condition for small error is fulfilled) calculate:
      \begin{itemize}
        \item
          the time-dependent part $\overline{v_\tau}$ of the right-hand side
          (allocated only once, there is no need to store it on every step),
          \begin{equation}
            \overline{v_\tau}^i := \overline{(R_\tau x^i)};
          \end{equation}
        \item
          the time-dependent part $\overline{v_\tau}$ of the solution
          (allocated only once, there is no need to store it on every step),
          \begin{equation}
            \overline{w_\tau}^i :=
            \overline{S_\tau}^{-T} \overline{S_\tau}^{-1} \overline{v_\tau}
          \end{equation}
          (with forward and back substitution);
        \item
          the solution $\overline{x}^{i + 1}$ on the non-Dirichlet nodes
          (allocated only once, there is no need to store it on every step),
          \begin{equation}
            \overline{x}^{i + 1} := \overline{w_\tau}^i + \overline{z_\tau};
          \end{equation}
        \item
          the final solution
          \begin{equation}
            x^{i + 1}_j :=
            \begin{cases}
              \overline{x}^{i + 1}_j, & j \in J \\
              g_D^0(N_j), & j \in I
            \end{cases}.
          \end{equation}
      \end{itemize}
    \item
      \textbf{Post-processing.}
      For each time moment $t_i$ the flow rate $(q^{d - 1})^i \in C^{d - 1} K$
      is calculated by the formula
      \begin{equation}
        (q^{d - 1})^i(c_{d - 1}) :=
        \begin{cases}
          ((-1)^{d - 1} \star_1 \tilde{\kappa}_1 \delta_0 u^{0,i})(c_{d - 1}),
            & c_{d - 1} \in K_{d - 1} \setminus (\Gamma_N)_{d - 1} \\
          g_N^{d - 1}(c_{d - 1}), & c_{d - 1} \in (\Gamma_N)_{d - 1}
        \end{cases}.
      \end{equation}
  \end{enumerate}
\end{algorithm}

\subsection{Mixed weak formulation}
\subsubsection{Steady-state}
\begin{formulation}
  Let
  \begin{itemize}
    \item
      $M$ be a manifold-like flat mesh of dimension $3$;
    \item
      $K$ be the Forman subdivision of $M$;
    \item
      $K' := K \setminus \partial K$ be the interior of $K$;
    \item
      $t_0 [T]\in \R$ be the initial time;
    \item
      $I := [t_0, \infty)$
    \item
      $u_0 [\Theta] \in C^0 K$ be the initial temperature;
    \item
      $f [E T^{-1}] \colon I \to C^3 K'$ be the heat source;
    \item
      $\Gamma_D, \Gamma_N$ form a partition of $(\partial K)$;
    \item
      $g_D [\Theta] \colon I \to C^0 \Gamma_D$
      be the Dirichlet boundary condition;
    \item
      $g_N [E T^{-1}] \colon I \to C^2 \Gamma_N$
      be the Neumann boundary condition;
    \item
      $\pi_0 [E L^{-3} \Theta^{-1}] \colon C^0 K \to C^0 K$
      be the volumetric heat capacity
      (its matrix in the standard basis is diagonal);
    \item
      $\pi_2 [E L^{-1} T^{-1} \Theta^{-1}] \colon C^2 K \to C^2 K$
      be the thermal conductivity
      (its matrix in the standard basis is diagonal).
  \end{itemize}
  We are solving the following problem:
  \begin{equation}
    \begin{split}
      & \text{Find $u [\Theta] \colon I \to C^0 K$ and
        $q [E T^{-1}] \colon I \to C^2 K$ such that} \\
      &
      \begin{cases}
        \forall r [E T^{-1}]\in \Ker \tr_{\Gamma_N, 2},\
          \inner{r}{\pi_2^{-1} q}_{K, 2} + (u \smile \delta_2 r)[K]
          = (\tr_{\Gamma_N, 2} r \smile g_D)[\Gamma_D] & [E T^{-1} \Theta], \\
        \forall v [\Theta]\in C^0 K,\
          (v \smile \delta_2 q)[K] - \frac{d}{d t} \inner{v}{\pi_0 u}_{K, 0}
          = - (v \smile f)[K] & [E T^{-1} \Theta], \\
        u(t_0, \cdot) = u_0 & [\Theta], \\
        \tr_{\Gamma_N, 2} q = g_N & [E T^{-1}].
      \end{cases}
    \end{split}
  \end{equation}
  In the steady-state model we do not have the initial temperature $u$
  and all quantities of type $f \colon I \to X$ become of type $f \in X$.
  It reads as follows:
  \begin{equation}
    \begin{split}
      & \text{Find $u [\Theta] \in C^0 K$ and
        $q [E T^{-1}] \in C^2 K$ such that} \\
      &
      \begin{cases}
        \forall r [E T^{-1}]\in \Ker \tr_{\Gamma_N, 2},\
          \inner{r}{\pi_2^{-1} q}_{K, 2} + (u \smile \delta_2 r)[K]
          = (\tr_{\Gamma_N, 2} r \smile g_D)[\Gamma_D] & [E T^{-1} \Theta], \\
        \forall v [\Theta]\in C^0 K,\
          (v \smile \delta_2 q)[K] = - (v \smile f)[K] & [E T^{-1} \Theta], \\
        u(t_0, \cdot) = u_0 & [\Theta], \\
        \tr_{\Gamma_N, 2} q = g_N & [E T^{-1}].
      \end{cases}
    \end{split}
  \end{equation}
  In both cases in the post-processing phase we have:
  \begin{subequations}
    \begin{alignat}{2}
      & \tilde{Q} && := \pi_0 u, \\
      & Q && \approx \star_0 \tilde{Q}.
    \end{alignat}
  \end{subequations}
\end{formulation}

\begin{discussion}
  \label{cmc/diffusion/discrete/steady_state/mixed_weak_solve-discussion}
  We are going to derive a solution to
  \Cref{cmc/diffusion/discrete/steady_state/mixed_weak-formulation}.
  For any $p \in \{0, ..., d\}$ denote
  \begin{equation}
    n_p := \abs{K_p} = \dim(C_p K) = \dim(C^p K).
  \end{equation}
  The cochains $(c^{p, 0}, ..., c^{p, n_0 - 1})$ form the standard basis of
  $C^p K$.
  Define the diagonal matrix $A \in M_{n_{d - 1} \times n_{d - 1}}(\R)$,
  and the sparse matrix $B \in M_{n_d \times n_{d - 1}}(\R)$ by
  \begin{subequations}
    \begin{alignat}{3}
      & A_{i, j}
      && := \inner{c^{d - 1, j}}{\kappa_{d - 1}^{-1} c^{d - 1, i}}, \enspace
      && i, j = 0, ..., n_{d - 1} - 1, \\
%
      & B_{k, i}
      && := \inner{\delta_{d - 1} c^{d - 1, i}}{c^{d, k}}, \enspace
      && k = 0, ..., n_d - 1,\ i = 0, ..., n_{d - 1} - 1.
    \end{alignat}
  \end{subequations}
  th
  and the vectors $F \in \R^{n_d}$, $G \in \R^{n_{d - 1}}$ by
  \begin{subequations}
    \begin{alignat}{3}
      & F_k
      && := \inner{f^d}{c^{d, k}}, \enspace
      && k = 0, ..., n_d - 1, \\
%
      & G_i
      && := (\tr_{\Gamma_D, d - 1} c^{d - 1, i} \smile g_D^0)[\Gamma_D],
        \enspace
      && i = 0, ..., n_{d - 1} - 1.
    \end{alignat}
  \end{subequations}
  Denote the unknown coefficients of $q^{d - 1}$ as
  $\{x_j\}_{j = 0}^{n_{d - 1} - 1}$, i.e.,
  \begin{equation}
    q^{d - 1} = \sum_{j = 0}^{n_{s - 1} - 1} x_j c^{d - 1, j},
  \end{equation}
  and the unknown coefficients of $\tilde{u}^d$ as
  $\{y_k\}_{k = 0}^{n_d - 1}$, i.e.,
  \begin{equation}
    \tilde{u}^d = \sum_{k = 0}^{n_d - 1} y_k c^{d, k}.
  \end{equation}
  Finally, let $J$ be the set of $(d - 1)$-cells on the Neumann boundary
  $\Gamma_N$, and $\overline{J} := \{0, ..., n_{d - 1} - 1\} \setminus I$.
  We get the system
  \begin{subequations}
    \begin{alignat}{3}
      & \sum_{j = 0}^{n_{d - 1} - 1} A_{i, j} x_j
        - \sum_{k = 0}^{n_d - 1} (B^T)_{i, k} y_k
      && = G_i, \enspace
      && i \in \overline{J}, \\
%
      & \sum_{i = 0}^{n_{d - 1} - 1} B_{k, i} x_i
      && = - F_k, \enspace
      && i \in \overline{J}, \\
%
      & x_i
      && = g_N^{d - 1}(c{d - 1, i}), \enspace
      && i \in J.
    \end{alignat}
  \end{subequations}
  This leads to the system of equations
  \begin{subequations}
    \begin{alignat}{3}
      & \sum_{j \in \overline{J}} A_{i, j} x_j
        - \sum_{k = 0}^{n_d - 1} (B^T)_{i, k} y_k
      && = G_i - \sum_{j \in J} A_{i, j} g_N^{d - 1}(c_{d - 1, j}), \enspace
      && i \in \overline{J}, \\
%
      & \sum_{i \in J} B_{k, i} x_i
      && = - F_k - \sum_{i \in J} B_{k, i} g_N^{d - 1}(c_{d - 1, i}), \enspace
      && i \in \overline{J}.
    \end{alignat}
  \end{subequations}
  (Note that since $A$ is diagonal, $A_{i, j} = 0$ when $i \in \overline{J}$ and
  $j \in J$.)
  Denote by $\overline{A},\ \overline{B},\ \overline{F},\ \overline{G},\
  \overline{x}$ the espective restricted matrices and vectors.
  We get the system of equations
  \begin{subequations}
    \begin{alignat}{2}
      & \overline{A} \overline{x} + \overline{B}^T y
      && = \overline{G}, \\
%
      & \overline{B} \overline{x}
      && = - \overline{F}.
    \end{alignat}
  \end{subequations}
  In general, when $A$ is sparse but not diagonal, it is not beneficial to use
  the inverse of $A$ in calculations since it will be a dense matrix.
  (This is the case in mixed finite element methods.)
  However, in our case $A$ is diagonal, so the following calculation makes sense
  computationally.
  We can solve for $\overline{x}$ by
  \begin{equation}
    \overline{x} = \overline{A}^{-1} (\overline{G} - \overline{B}^T y).
  \end{equation}
  Hence,
  \begin{equation}
    - \overline{F}
    = \overline{B} \overline{x}
    = \overline{B} \overline{A}^{-1} (\overline{G} - \overline{B}^T y),
  \end{equation}
  which translates to
  \begin{equation}
    \overline{B} \overline{A}^{-1} \overline{B}^T y
    = \overline{B} \overline{A}^{-1} \overline{G} + \overline{F}.
  \end{equation}
  Using the Cholesky decomposition, we can solve for $y$ and then substitute to
  find $\overline{x}$.
\end{discussion}

\subsubsection{Transient}
\begin{formulation}
  Let
  \begin{itemize}
    \item
      $M$ be a manifold-like flat mesh of dimension $3$;
    \item
      $K$ be the Forman subdivision of $M$;
    \item
      $K' := K \setminus \partial K$ be the interior of $K$;
    \item
      $t_0 [T]\in \R$ be the initial time;
    \item
      $I := [t_0, \infty)$
    \item
      $u_0 [\Theta] \in C^0 K$ be the initial temperature;
    \item
      $f [E T^{-1}] \colon I \to C^3 K'$ be the heat source;
    \item
      $\Gamma_D, \Gamma_N$ form a partition of $(\partial K)$;
    \item
      $g_D [\Theta] \colon I \to C^0 \Gamma_D$
      be the Dirichlet boundary condition;
    \item
      $g_N [E T^{-1}] \colon I \to C^2 \Gamma_N$
      be the Neumann boundary condition;
    \item
      $\pi_0 [E L^{-3} \Theta^{-1}] \colon C^0 K \to C^0 K$
      be the volumetric heat capacity
      (its matrix in the standard basis is diagonal);
    \item
      $\pi_2 [E L^{-1} T^{-1} \Theta^{-1}] \colon C^2 K \to C^2 K$
      be the thermal conductivity
      (its matrix in the standard basis is diagonal).
  \end{itemize}
  We are solving the following problem:
  \begin{equation}
    \begin{split}
      & \text{Find $u [\Theta] \colon I \to C^0 K$ and
        $q [E T^{-1}] \colon I \to C^2 K$ such that} \\
      &
      \begin{cases}
        \forall r [E T^{-1}]\in \Ker \tr_{\Gamma_N, 2},\
          \inner{r}{\pi_2^{-1} q}_{K, 2} + (u \smile \delta_2 r)[K]
          = (\tr_{\Gamma_N, 2} r \smile g_D)[\Gamma_D] & [E T^{-1} \Theta], \\
        \forall v [\Theta]\in C^0 K,\
          (v \smile \delta_2 q)[K] - \frac{d}{d t} \inner{v}{\pi_0 u}_{K, 0}
          = - (v \smile f)[K] & [E T^{-1} \Theta], \\
        u(t_0, \cdot) = u_0 & [\Theta], \\
        \tr_{\Gamma_N, 2} q = g_N & [E T^{-1}].
      \end{cases}
    \end{split}
  \end{equation}
  In the steady-state model we do not have the initial temperature $u$
  and all quantities of type $f \colon I \to X$ become of type $f \in X$.
  It reads as follows:
  \begin{equation}
    \begin{split}
      & \text{Find $u [\Theta] \in C^0 K$ and
        $q [E T^{-1}] \in C^2 K$ such that} \\
      &
      \begin{cases}
        \forall r [E T^{-1}]\in \Ker \tr_{\Gamma_N, 2},\
          \inner{r}{\pi_2^{-1} q}_{K, 2} + (u \smile \delta_2 r)[K]
          = (\tr_{\Gamma_N, 2} r \smile g_D)[\Gamma_D] & [E T^{-1} \Theta], \\
        \forall v [\Theta]\in C^0 K,\
          (v \smile \delta_2 q)[K] = - (v \smile f)[K] & [E T^{-1} \Theta], \\
        u(t_0, \cdot) = u_0 & [\Theta], \\
        \tr_{\Gamma_N, 2} q = g_N & [E T^{-1}].
      \end{cases}
    \end{split}
  \end{equation}
  In both cases in the post-processing phase we have:
  \begin{subequations}
    \begin{alignat}{2}
      & \tilde{Q} && := \pi_0 u, \\
      & Q && \approx \star_0 \tilde{Q}.
    \end{alignat}
  \end{subequations}
\end{formulation}

\begin{discussion}
  \label{cmc/diffusion/discrete/transient/mixed_weak_solve_trapezoidal-discussion}
  We are going to derive a solution to
  \Cref{cmc/diffusion/discrete/transient/mixed_weak-formulation}
  using the trapezoidal rule for time integration.
  We will assume that the heat capacity $\pi_0$ is time-independent which will
  allow us to rearrange the time derivative:
  \begin{equation}
    C(v^d, \frac{\partial \tilde{u}^d} {\partial t})
    = \frac{d}{d t} C(v^d, \tilde{u}^d).
  \end{equation}
  For further simplicity we will also assume that all the rest input data (heat
  source, thermal conductivity, boundary conditions) are also time-independent.
  We can then integrate the equation
  \begin{equation}
    B(v^d, q^{d - 1}) - \frac{d}{d t} C(v^d, \tilde{u}^d) = -F(v^d)
  \end{equation}
  with respect to $t$ in the interval $[\alpha, \beta] \subset I$ to get
  \begin{equation}
    B(v^d, \int_\alpha^\beta q^{d - 1}\, d t)
    - (C(v^d, \tilde{u}^d(\beta)) - C(v^d, \tilde{u}^d(\alpha)))
    = - (\beta - \alpha) F(v^d).
  \end{equation}
  If we partition $I$ into segements with size $\tau$ with moments of
  time $\{t_s := t_0 + \tau s\}_{s \geq 0}$, and if we denote
  \begin{subequations}
    \begin{alignat}{3}
      & Q^s
      && := q^{d - 1}(t_s), \enspace
      && s \geq 0, \\
      %
      & U^s
      && := \tilde{u}^d(t_s), \enspace
      && s \geq 0,
    \end{alignat}
  \end{subequations}
  we get
  \begin{equation}
    \frac{\tau}{2} (B(v^d, Q^s) + B(v^d, Q^{s + 1}))
    - (C(v^d, U^{s + 1}) - C(v^d, U^s))
    = - \tau F(v^d).
  \end{equation}
  By multiplying the above equation with $2 / \tau$ and rearranging we get:
  \begin{equation}
    B(v^d, Q^{s + 1}) - \frac{2}{\tau} C(v^d, U^{s + 1})
    = - B(v^d, Q^s) - \frac{2}{\tau} C(v^d, U^s) - 2 F(v^d).
  \end{equation}
  Define $H^s(v^d) := B(v^d, Q^s) + \frac{2}{\tau} C(v^d, U^s)$.
  We arrive at the following problem: find
  $\{(Q^s, U^s) \in C^{d - 1} K \times C^d K\}_{s \geq 0}$
  such that
  \begin{subequations}
    \begin{alignat}{4}
      & \forall r^{d - 1} [E T^{-1}] \in \Ker \tr_{\Gamma_N, D - 1}, \:
      && A(r^{d - 1}, Q^{s + 1}) + B^T(r^{d - 1}, U^{s + 1})
      && = G(r^{d - 1}), \:
      && s \geq 0, \\
%
      & \forall v^d [\Theta L^d] \in C^d K, \:
      && B(v^d, Q^{s + 1}) - \frac{2}{\tau} C(v^d, U^{s + 1})
      && = - H^s(v^d) - 2 F(v^d), \:
      && s \geq 0, \\
%
      &
      && \tr_{\Gamma_N, d - 1} Q^s
      && = g_N^{d - 1}, \:
      && s > 0, \\
%
      &
      && Q^0
      % && = ((-1)^{d - 1} \kappa_{d - 1} \star_1 \delta_0)(u_0),
      && = {\rm flow\_rate}(u_0),
      && \\
%
      &
      && U^0
      && = \star_0 u_0,
      &&
    \end{alignat}
  \end{subequations}
  where
  \begin{equation}
    {\rm flow\_rate}(y^0) := ((-1)^{d - 1} \kappa_{d - 1} \star_1 \delta_0)(y^0),\
    y^0 \in C^0 K.
  \end{equation}
  Let
  \begin{subequations}
    \begin{alignat}{2}
      & J
      && := \set{i \in \{0, ..., n_{d - 1}\}}
        {c_{d - 1, i} \in (\Gamma_N)_{d - 1}}, \\
%
      & \overline{J}
      && := \{0, ..., n_{d - 1}\} \setminus J.
    \end{alignat}
  \end{subequations}
  Initial conditions give us $Q^0$ and $U^0$.
  Denoting the matrices ${\bf A}$, ${\bf B}$, ${\bf C}$ and vectors
  ${\bf Q}^s$, ${\bf U}^s$, ${\bf F}$, ${\bf G}$
  of the corresponding operators in standard bases, we get the system
  for ${\bf Q}^s$ and ${\bf U}^s$, $s > 0$.
  \begin{subequations}
    \begin{alignat}{3}
      & \sum_{j = 0}^{n_{d - 1} - 1} {\bf A}_{i, j} {\bf Q}^{s + 1}_j
        - \sum_{k = 0}^{n_d - 1} ({\bf B}^T)_{i, k} {\bf U}^{s + 1}_k
      && = {\bf G}_i, \:
      && i \in \overline{J}, \\
%
      & \sum_{i = 0}^{n_{d - 1} - 1} {\bf B}_{k, i} {\bf Q}^{s +1}
        -\frac{2}{\tau} {\bf C} {\bf U}^{s + 1}
      && = (- {\bf B} {\bf Q}^s - \frac{2}{\tau} {\bf C} {\bf U}^s)_k
        - 2 {\bf F}_k, \:
      && k \in \{0, ..., n_d - 1\}, \\
%
      & x_i
      && = g_N^{d - 1}(c_{d - 1, i}), \:
      && i \in J.
    \end{alignat}
  \end{subequations}
  This leads to the follwoing system of equations for any $s \in \N$:
  \begin{subequations}
    \begin{alignat}{3}
      & \sum_{j \in \overline{J}} {\bf A}_{i, j} {\bf Q}^{s + 1}_j
        - \sum_{k = 0}^{n_d - 1} ({\bf B}^T)_{i, k} {\bf U}^{s + 1}_k
      && = {\bf G}_i -
        \sum_{j \in J} {\bf A}_{i, j} g_N^{d - 1}(c_{d - 1, j}), \:
      && i \in \overline{J}, \\
%
      & \sum_{i \in J} {\bf B}_{k, i} {\bf Q}^{s + 1}_i
        - \sum_{l = 0}^{n_d - 1}
          \frac{2}{\tau}{\bf C}_{k, l} {\bf U}^{s + 1}_l
      && = (- {\bf B} {\bf Q}^s - \frac{2}{\tau} {\bf C} {\bf U}^s)_k
        - 2 {\bf F}_k - \sum_{i \in J} {\bf B}_{k, i} g_N^{d - 1}(c_{d - 1, i}),
        \:
      && i \in \overline{J}.
    \end{alignat}
  \end{subequations}
  Let $\overline{\bf A}$ be the restriction of ${\bf A}$ to the rows and colums in
  $\overline{J}$,
  $\overline{\bf B}$ be the restristion of ${\bf B}$ to the colums in
  $\overline{J}$,
  $\overline{\bf Q}^{s + 1}$ be the restriction of ${\bf Q}^{s}$ to the
  indices in $\overline{J}$,
  $\widetilde{\bf F} \in \R^{n_d}$ be defined as,
  \begin{equation}
    \widetilde{\bf F}_k
    := 2 {\bf F}_k + \sum_{i \in J} {\bf B}_{k, i} g_N^{d - 1}(c_{d - 1, i}),\
    k = 0, ..., n_d - 1,
  \end{equation}
  $\overline{\bf G}$ be the restriction of ${\bf G}$ to the indices in
  $\overline{J}$ (since $A$ is diagonal, ${\bf G}$ is not modified before
  restriction).
  For any $s \in \N$ we get the mixed system
  \begin{subequations}
    \begin{alignat}{2}
      & \overline{\bf A}\, \overline{\bf Q}^{s + 1}
        + \overline{\bf B}^T {\bf U}^{s + 1}
      && = \overline{\bf G}, \\
%
      & \overline{\bf B}\, \overline{\bf Q}^{s + 1}
        - \frac{2}{\tau} {\bf C} {\bf U}^{s + 1}
      && = - {\bf B} {\bf Q}^s - \frac{2}{\tau} {\bf C} {\bf U}^s
        - \widetilde{\bf F}.
    \end{alignat}
  \end{subequations}
  We can solve for $\overline{\bf Q}^{s + 1}$, $s \in \N$ as follows:
  \begin{equation}
    \overline{\bf Q}^{s + 1}
    = \overline{\bf A}^{-1}
      (\overline{\bf G} - \overline{\bf B}^T {\bf U}^{s + 1}).
  \end{equation}
  Hence,
  \begin{equation}
    - {\bf B} {\bf Q}^s - \frac{2}{\tau} {\bf C} {\bf U}^s - \widetilde{\bf F}
    = \overline{\bf B}\, \overline{\bf Q}^{s + 1}
      - \frac{2}{\tau} {\bf C} {\bf U}^{s + 1}
    = \overline{\bf B}\, \overline{\bf A}^{-1}
      (\overline{\bf G} - \overline{\bf B}^T {\bf U}^{s + 1})
      - \frac{2}{\tau} {\bf C} {\bf U}^{s + 1},
  \end{equation}
  which translates to
  \begin{equation}
    (\overline{\bf B}\, \overline{\bf A}^{-1} \overline{\bf B}^T
      + \frac{2}{\tau} {\bf C}) {\bf U}^{s + 1}
    = \overline{\bf B} \overline{\bf A}^{-1} \overline{\bf G}
      + \widetilde{\bf F} + {\bf B} {\bf Q}^s + \frac{2}{\tau} {\bf C} {\bf U}^s.
  \end{equation}
  Define the left-hand side matrix ${\bf N}_\tau \in M_{n_d \times n_d}(\R)$,
  \begin{equation}
    {\bf N}_\tau
    := \overline{\bf B}\, \overline{\bf A}^{-1} \overline{\bf B}^T
      + \frac{2}{\tau} {\bf C},
  \end{equation}
  and the constant right-hand side vector ${\bf Z} \in \R^{n_d}$,
  \begin{equation}
    {\bf Z}
    := \overline{\bf B} \overline{\bf A}^{-1} \overline{\bf G}
      + \widetilde{\bf F}.
  \end{equation}
  This leads to the following linear $n_d \times n_d$ system:
  \begin{equation}
    {\bf N}_\tau {\bf U}^{s + 1}
    = {\bf Z} + {\bf B} {\bf Q}^s + \frac{2}{\tau} {\bf C} {\bf U}^s.
  \end{equation}
  We can apply the Cholesky decomposition to the left-hand side to arrive at
  the following algorithm.
\end{discussion}

\input{diffusion/discrete/transient/mixed_weak_solve_trapezoidal-algorithm.tex}

\section{Examples of diffusion}
\label{section:examples_of_transport_phenomena}
\subsection{Steady-state}
\begin{example}
  \label{idec/diffusion/continuous/transient/examples/2d_d00_p00-example}
  Consider the transient continuous heat transport problem
  (\Cref{idec/diffusion/continuous/transient/primal_strong-formulation},
   \Cref{idec/diffusion/continuous/transient/primal_weak-formulation},
   \Cref{idec/diffusion/continuous/transient/mixed_weak-formulation})
  with input data \verb|2d_d00_p00| in the nomenclature of the C codebase.

  Concretely,
    $X = [0, 1]^2$,
    $\pi_0 \equiv 1$,
    $\kappa_1 \equiv 1$,
    $u_0(x, y) =
      \begin{cases}
        100, & (x, y) = (0.5, 0.5) \\
        0, & \text{else}
      \end{cases}$,
    $f \equiv 0$,
    $G_D = \partial X$,
    $G_N = \emptyset$,
    $g_D(x, y) = 0$.

  This problem has the following exact solution in steady-state:
  \begin{subequations}
    \begin{alignat}{3}
      & u && \equiv && 0, \\
      & q && \equiv && 0.
    \end{alignat}
  \end{subequations}
  Consider a mesh $M$ for $X$ consisting of $2 \times 2$ squares (each axis is
  divided into $2$ segments) with Forman subdivision $K$ ($4 \times 4$ squares).
  Its potential and flow rate on $K$ consisting of the $3$ discussed cochain
  methods are shown on
  \Cref{figure:idec/diffusion/transient/continuous_2d_d00_p00/brick_2d_2_forman_trapezoidal_0p001_1000_potential}
  and
  \Cref{figure:idec/diffusion/transient/continuous_2d_d00_p00/brick_2d_2_forman_trapezoidal_0p001_1000_flow_rate}.
\end{example}

\begin{figure}[!ht]
  \begin{subfigure}{.32\textwidth}
    \centering
    \includegraphics[scale=.2, page=1]
    {diffusion/transient/continuous_2d_d00_p00/primal_strong_cochain_brick_2d_2_forman_trapezoidal_0p001_1000_potential}
    \caption{Primal strong potential, 1}
  \end{subfigure}
  \begin{subfigure}{.32\textwidth}
    \centering
    \includegraphics[scale=.2, page=151]
    {diffusion/transient/continuous_2d_d00_p00/primal_strong_cochain_brick_2d_2_forman_trapezoidal_0p001_1000_potential}
    \caption{Primal strong potential, 151}
  \end{subfigure}
  \begin{subfigure}{.32\textwidth}
    \centering
    \includegraphics[scale=.2, page=301]
    {diffusion/transient/continuous_2d_d00_p00/primal_strong_cochain_brick_2d_2_forman_trapezoidal_0p001_1000_potential}
    \caption{Primal strong potential, 301}
  \end{subfigure}

  \begin{subfigure}{.32\textwidth}
    \centering
    \includegraphics[scale=.2, page=1]
    {diffusion/transient/continuous_2d_d00_p00/primal_weak_cochain_brick_2d_2_forman_trapezoidal_0p001_1000_potential}
    \caption{Primal weak potential, 1}
  \end{subfigure}
  \begin{subfigure}{.32\textwidth}
    \centering
    \includegraphics[scale=.2, page=151]
    {diffusion/transient/continuous_2d_d00_p00/primal_weak_cochain_brick_2d_2_forman_trapezoidal_0p001_1000_potential}
    \caption{Primal weak potential, 151}
  \end{subfigure}
  \begin{subfigure}{.32\textwidth}
    \centering
    \includegraphics[scale=.2, page=301]
    {diffusion/transient/continuous_2d_d00_p00/primal_weak_cochain_brick_2d_2_forman_trapezoidal_0p001_1000_potential}
    \caption{Primal weak potential, 301}
  \end{subfigure}

  \begin{subfigure}{.32\textwidth}
    \centering
    \includegraphics[scale=.2, page=1]
    {diffusion/transient/continuous_2d_d00_p00/mixed_weak_cochain_brick_2d_2_forman_trapezoidal_0p001_1000_potential}
    \caption{Mixed weak potential, 1}
  \end{subfigure}
  \begin{subfigure}{.32\textwidth}
    \centering
    \includegraphics[scale=.2, page=151]
    {diffusion/transient/continuous_2d_d00_p00/mixed_weak_cochain_brick_2d_2_forman_trapezoidal_0p001_1000_potential}
    \caption{Mixed weak potential, 151}
  \end{subfigure}
  \begin{subfigure}{.32\textwidth}
    \centering
    \includegraphics[scale=.2, page=301]
    {diffusion/transient/continuous_2d_d00_p00/mixed_weak_cochain_brick_2d_2_forman_trapezoidal_0p001_1000_potential}
    \caption{Mixed weak potential, 301}
  \end{subfigure}

  \begin{subfigure}{.32\textwidth}
    \centering
    \includegraphics[scale=.2, page=1]
    {diffusion/transient/continuous_2d_d00_p00/primal_strong_cochain_brick_2d_2_forman_trapezoidal_0p001_1000_flow}
    \caption{Primal strong flow, 1}
  \end{subfigure}
  \begin{subfigure}{.32\textwidth}
    \centering
    \includegraphics[scale=.2, page=151]
    {diffusion/transient/continuous_2d_d00_p00/primal_strong_cochain_brick_2d_2_forman_trapezoidal_0p001_1000_flow}
    \caption{Primal strong flow, 151}
  \end{subfigure}
  \begin{subfigure}{.32\textwidth}
    \centering
    \includegraphics[scale=.2, page=301]
    {diffusion/transient/continuous_2d_d00_p00/primal_strong_cochain_brick_2d_2_forman_trapezoidal_0p001_1000_flow}
    \caption{Primal strong flow, 301}
  \end{subfigure}

  \begin{subfigure}{.32\textwidth}
    \centering
    \includegraphics[scale=.2, page=1]
    {diffusion/transient/continuous_2d_d00_p00/primal_weak_cochain_brick_2d_2_forman_trapezoidal_0p001_1000_flow}
    \caption{Primal weak flow, 1}
  \end{subfigure}
  \begin{subfigure}{.32\textwidth}
    \centering
    \includegraphics[scale=.2, page=151]
    {diffusion/transient/continuous_2d_d00_p00/primal_weak_cochain_brick_2d_2_forman_trapezoidal_0p001_1000_flow}
    \caption{Primal weak flow, 151}
  \end{subfigure}
  \begin{subfigure}{.32\textwidth}
    \centering
    \includegraphics[scale=.2, page=301]
    {diffusion/transient/continuous_2d_d00_p00/primal_weak_cochain_brick_2d_2_forman_trapezoidal_0p001_1000_flow}
    \caption{Primal weak flow, 301}
  \end{subfigure}
  
  \begin{subfigure}{.32\textwidth}
    \centering
    \includegraphics[scale=.2, page=1]
    {diffusion/transient/continuous_2d_d00_p00/mixed_weak_cochain_brick_2d_2_forman_trapezoidal_0p001_1000_flow}
    \caption{Mixed weak flow, 1}
  \end{subfigure}
  \begin{subfigure}{.32\textwidth}
    \centering
    \includegraphics[scale=.2, page=151]
    {diffusion/transient/continuous_2d_d00_p00/mixed_weak_cochain_brick_2d_2_forman_trapezoidal_0p001_1000_flow}
    \caption{Mixed weak flow, 151}
  \end{subfigure}
  \begin{subfigure}{.32\textwidth}
    \centering
    \includegraphics[scale=.2, page=301]
    {diffusion/transient/continuous_2d_d00_p00/mixed_weak_cochain_brick_2d_2_forman_trapezoidal_0p001_1000_flow}
    \caption{Mixed weak flow, 301}
  \end{subfigure}
  \cprotect\caption{Potential and flow for the \verb|2d_d00_p00| problem}
  \label{figure:idec/diffusion/transient/continuous_2d_d00_p00/cochain_brick_2d_2_forman_trapezoidal_0p001_1000}
\end{figure}

\begin{example}
  Consider the steady-state continuous heat transport problem
  (\Cref{idec/heat_transport/continuous/primal_strong_steady_state-formulation},
   \Cref{idec/heat_transport/continuous/primal_weak_steady_state-formulation},
   \Cref{idec/heat_transport/continuous/mixed_weak_steady_state-formulation})
  with input data \verb|2d_d00_p01| in the nomenclature of the C codebase.

  Concretely,
    $X = [0, 1]^2$,
    $\kappa_1 \equiv 1$,
    $f \equiv 0$,
    $G_D = \{0, 1\} \times [0, 1]$,
    $G_N = [0, 1] \times \{0, 1\}$,
    $g_D(x, y) = \begin{cases} -100, & x = 0 \\ 100, & x = 1 \end{cases}$,
    $g_N \equiv 0$.

  This problem has the following exact solution:
  \begin{subequations}
    \begin{alignat}{3}
      & u(x, y) && = && 100 (2 x - 1), \\
      & q(x, y) && = && 200\, d y.
    \end{alignat}
  \end{subequations}
  Consider a mesh $M$ for $X$ consisting of $2 \times 2$ squares (each axis is
  divided into $2$ segments) with Forman subdivision $K$ ($4 \times 4$ squares).
  Its potential and flow on $K$ consisting of the exact solution and the $3$
  discussed cochain methods are shown on
  \Cref{figure:idec/diffusion/steady_state/continuous_2d_d00_p01/brick_2d_2_forman}.
\end{example}

\begin{figure}[!ht]
  \begin{subfigure}{.24\textwidth}
    \centering
    \includegraphics[scale=.19, page=1]
    {diffusion/transient/continuous_2d_d00_p01/primal_strong_cochain_brick_2d_5_forman_trapezoidal_0p001_2500_potential}
    \caption{Primal strong, 1}
  \end{subfigure}
  \begin{subfigure}{.24\textwidth}
    \centering
    \includegraphics[scale=.19, page=501]
    {diffusion/transient/continuous_2d_d00_p01/primal_strong_cochain_brick_2d_5_forman_trapezoidal_0p001_2500_potential}
    \caption{Primal strong, 501}
  \end{subfigure}
  \begin{subfigure}{.24\textwidth}
    \centering
    \includegraphics[scale=.19, page=1001]
    {diffusion/transient/continuous_2d_d00_p01/primal_strong_cochain_brick_2d_5_forman_trapezoidal_0p001_2500_potential}
    \caption{Primal strong, 1001}
  \end{subfigure}
  \begin{subfigure}{.24\textwidth}
    \centering
    \includegraphics[scale=.19, page=1501]
    {diffusion/transient/continuous_2d_d00_p01/primal_strong_cochain_brick_2d_5_forman_trapezoidal_0p001_2500_potential}
    \caption{Primal strong, 1501}
  \end{subfigure}

  \begin{subfigure}{.24\textwidth}
    \centering
    \includegraphics[scale=.19, page=1]
    {diffusion/transient/continuous_2d_d00_p01/primal_weak_cochain_brick_2d_5_forman_trapezoidal_0p001_2500_potential}
    \caption{Primal weak, 1}
  \end{subfigure}
  \begin{subfigure}{.24\textwidth}
    \centering
    \includegraphics[scale=.19, page=501]
    {diffusion/transient/continuous_2d_d00_p01/primal_weak_cochain_brick_2d_5_forman_trapezoidal_0p001_2500_potential}
    \caption{Primal weak, 501}
  \end{subfigure}
  \begin{subfigure}{.24\textwidth}
    \centering
    \includegraphics[scale=.19, page=1001]
    {diffusion/transient/continuous_2d_d00_p01/primal_weak_cochain_brick_2d_5_forman_trapezoidal_0p001_2500_potential}
    \caption{Primal weak, 1001}
  \end{subfigure}
  \begin{subfigure}{.24\textwidth}
    \centering
    \includegraphics[scale=.19, page=1501]
    {diffusion/transient/continuous_2d_d00_p01/primal_weak_cochain_brick_2d_5_forman_trapezoidal_0p001_2500_potential}
    \caption{Primal weak, 1501}
  \end{subfigure}

  \begin{subfigure}{.24\textwidth}
    \centering
    \includegraphics[scale=.19, page=1]
    {diffusion/transient/continuous_2d_d00_p01/mixed_weak_cochain_brick_2d_5_forman_trapezoidal_0p001_2500_potential}
    \caption{Mixed weak, 1}
  \end{subfigure}
  \begin{subfigure}{.24\textwidth}
    \centering
    \includegraphics[scale=.19, page=501]
    {diffusion/transient/continuous_2d_d00_p01/mixed_weak_cochain_brick_2d_5_forman_trapezoidal_0p001_2500_potential}
    \caption{Mixed weak, 501}
  \end{subfigure}
  \begin{subfigure}{.24\textwidth}
    \centering
    \includegraphics[scale=.19, page=1001]
    {diffusion/transient/continuous_2d_d00_p01/mixed_weak_cochain_brick_2d_5_forman_trapezoidal_0p001_2500_potential}
    \caption{Mixed weak, 1001}
  \end{subfigure}
  \begin{subfigure}{.24\textwidth}
    \centering
    \includegraphics[scale=.19, page=1501]
    {diffusion/transient/continuous_2d_d00_p01/mixed_weak_cochain_brick_2d_5_forman_trapezoidal_0p001_2500_potential}
    \caption{Mixed weak, 1501}
  \end{subfigure}
  \cprotect
  \caption{%
    \verb|diffusion/transient/continuous_2d_d00_p01|
    (\Cref{cmc/diffusion/continuous/transient/examples/2d_d00_p01-example}):
    solutions for potential}
  \label{figure:cmc/diffusion/transient/continuous_2d_d00_p01/brick_2d_5_forman_trapezoidal_0p001_2500_potential}
\end{figure}
\begin{figure}[!ht]
  \begin{subfigure}{.24\textwidth}
    \centering
    \includegraphics[scale=.19, page=1]
    {diffusion/transient/continuous_2d_d00_p01/primal_strong_cochain_brick_2d_5_forman_trapezoidal_0p001_2500_flow_rate}
    \caption{Primal strong, 1}
  \end{subfigure}
  \begin{subfigure}{.24\textwidth}
    \centering
    \includegraphics[scale=.19, page=501]
    {diffusion/transient/continuous_2d_d00_p01/primal_strong_cochain_brick_2d_5_forman_trapezoidal_0p001_2500_flow_rate}
    \caption{Primal strong, 501}
  \end{subfigure}
  \begin{subfigure}{.24\textwidth}
    \centering
    \includegraphics[scale=.19, page=1001]
    {diffusion/transient/continuous_2d_d00_p01/primal_strong_cochain_brick_2d_5_forman_trapezoidal_0p001_2500_flow_rate}
    \caption{Primal strong, 1001}
  \end{subfigure}
  \begin{subfigure}{.24\textwidth}
    \centering
    \includegraphics[scale=.19, page=1501]
    {diffusion/transient/continuous_2d_d00_p01/primal_strong_cochain_brick_2d_5_forman_trapezoidal_0p001_2500_flow_rate}
    \caption{Primal strong, 1501}
  \end{subfigure}

  \begin{subfigure}{.24\textwidth}
    \centering
    \includegraphics[scale=.19, page=1]
    {diffusion/transient/continuous_2d_d00_p01/primal_weak_cochain_brick_2d_5_forman_trapezoidal_0p001_2500_flow_rate}
    \caption{Primal weak, 1}
  \end{subfigure}
  \begin{subfigure}{.24\textwidth}
    \centering
    \includegraphics[scale=.19, page=501]
    {diffusion/transient/continuous_2d_d00_p01/primal_weak_cochain_brick_2d_5_forman_trapezoidal_0p001_2500_flow_rate}
    \caption{Primal weak, 501}
  \end{subfigure}
  \begin{subfigure}{.24\textwidth}
    \centering
    \includegraphics[scale=.19, page=1001]
    {diffusion/transient/continuous_2d_d00_p01/primal_weak_cochain_brick_2d_5_forman_trapezoidal_0p001_2500_flow_rate}
    \caption{Primal weak, 1001}
  \end{subfigure}
  \begin{subfigure}{.24\textwidth}
    \centering
    \includegraphics[scale=.19, page=1501]
    {diffusion/transient/continuous_2d_d00_p01/primal_weak_cochain_brick_2d_5_forman_trapezoidal_0p001_2500_flow_rate}
    \caption{Primal weak, 1501}
  \end{subfigure}

  \begin{subfigure}{.24\textwidth}
    \centering
    \includegraphics[scale=.19, page=1]
    {diffusion/transient/continuous_2d_d00_p01/mixed_weak_cochain_brick_2d_5_forman_trapezoidal_0p001_2500_flow_rate}
    \caption{Mixed weak, 1}
  \end{subfigure}
  \begin{subfigure}{.24\textwidth}
    \centering
    \includegraphics[scale=.19, page=501]
    {diffusion/transient/continuous_2d_d00_p01/mixed_weak_cochain_brick_2d_5_forman_trapezoidal_0p001_2500_flow_rate}
    \caption{Mixed weak, 501}
  \end{subfigure}
  \begin{subfigure}{.24\textwidth}
    \centering
    \includegraphics[scale=.19, page=1001]
    {diffusion/transient/continuous_2d_d00_p01/mixed_weak_cochain_brick_2d_5_forman_trapezoidal_0p001_2500_flow_rate}
    \caption{Mixed weak, 1001}
  \end{subfigure}
  \begin{subfigure}{.24\textwidth}
    \centering
    \includegraphics[scale=.19, page=1501]
    {diffusion/transient/continuous_2d_d00_p01/mixed_weak_cochain_brick_2d_5_forman_trapezoidal_0p001_2500_flow_rate}
    \caption{Mixed weak, 1501}
  \end{subfigure}
  \cprotect
  \caption{%
    \verb|diffusion/transient/continuous_2d_d00_p01|
    (\Cref{cmc/diffusion/continuous/transient/examples/2d_d00_p01-example}):
    solutions for flow rate}
  \label{figure:cmc/diffusion/transient/continuous_2d_d00_p01/brick_2d_5_forman_trapezoidal_0p001_2500_flow_rate}
\end{figure}

\begin{example}
  Consider the transient continuous heat transport problem
  (\Cref{idec/diffusion/continuous/transient/primal_strong-formulation},
   \Cref{idec/diffusion/continuous/transient/primal_weak-formulation},
   \Cref{idec/diffusion/continuous/transient/mixed_weak-formulation})
  with input data \verb|2d_d00_p02| in the nomenclature of the C codebase.

  Concretely,
    $X = [0, 1]^2$,
    $\pi_0 \equiv 1$,
    $\kappa_1 \equiv 1$,
    $u_0(x, y) = x^2 + y^2$,
    $f \equiv - 4\, d x \wedge d y$,
    $G_D = \partial X$,
    $G_N = \emptyset$,
    $g_D(x, y) = x^2 + y^2$.

  This problem is in steady-state mode from the beginning
  with the following exact solution:
  \begin{subequations}
    \begin{alignat}{3}
      & u(x, y) && = && x^2 + y^2, \\
      & q(x, y) && = && 2 y\, d x - 2 x\, d y.
    \end{alignat}
  \end{subequations}
  Consider a mesh $M$ for $X$ consisting of $2 \times 2$ squares (each axis is
  divided into $2$ segments) with Forman subdivision $K$ ($4 \times 4$ squares).
  Its potential and flow on $K$ consisting of the exact solution and the $3$
  discussed cochain methods are shown on
  \Cref{figure:idec/diffusion/transient/continuous_2d_d00_p02/brick_2d_5_forman_trapezoidal_0p001_1000}.
\end{example}

\begin{figure}[!ht]
  \begin{subfigure}{.24\textwidth}
    \centering
    \includegraphics[scale=.19, page=1]
    {diffusion/transient/continuous_2d_d00_p02/primal_strong_cochain_brick_2d_5_forman_trapezoidal_0p001_1000_potential}
    \caption{Primal strong, 1}
  \end{subfigure}
  \begin{subfigure}{.24\textwidth}
    \centering
    \includegraphics[scale=.19, page=334]
    {diffusion/transient/continuous_2d_d00_p02/primal_strong_cochain_brick_2d_5_forman_trapezoidal_0p001_1000_potential}
    \caption{Primal strong, 334}
  \end{subfigure}
  \begin{subfigure}{.24\textwidth}
    \centering
    \includegraphics[scale=.19, page=667]
    {diffusion/transient/continuous_2d_d00_p02/primal_strong_cochain_brick_2d_5_forman_trapezoidal_0p001_1000_potential}
    \caption{Primal strong, 667}
  \end{subfigure}
  \begin{subfigure}{.24\textwidth}
    \centering
    \includegraphics[scale=.19, page=1001]
    {diffusion/transient/continuous_2d_d00_p02/primal_strong_cochain_brick_2d_5_forman_trapezoidal_0p001_1000_potential}
    \caption{Primal strong, 1001}
  \end{subfigure}

  \begin{subfigure}{.24\textwidth}
    \centering
    \includegraphics[scale=.19, page=1]
    {diffusion/transient/continuous_2d_d00_p02/primal_weak_cochain_brick_2d_5_forman_trapezoidal_0p001_1000_potential}
    \caption{Primal weak, 1}
  \end{subfigure}
  \begin{subfigure}{.24\textwidth}
    \centering
    \includegraphics[scale=.19, page=334]
    {diffusion/transient/continuous_2d_d00_p02/primal_weak_cochain_brick_2d_5_forman_trapezoidal_0p001_1000_potential}
    \caption{Primal weak, 334}
  \end{subfigure}
  \begin{subfigure}{.24\textwidth}
    \centering
    \includegraphics[scale=.19, page=667]
    {diffusion/transient/continuous_2d_d00_p02/primal_weak_cochain_brick_2d_5_forman_trapezoidal_0p001_1000_potential}
    \caption{Primal weak, 667}
  \end{subfigure}
  \begin{subfigure}{.24\textwidth}
    \centering
    \includegraphics[scale=.19, page=1001]
    {diffusion/transient/continuous_2d_d00_p02/primal_weak_cochain_brick_2d_5_forman_trapezoidal_0p001_1000_potential}
    \caption{Primal weak, 1001}
  \end{subfigure}

  \begin{subfigure}{.24\textwidth}
    \centering
    \includegraphics[scale=.19, page=1]
    {diffusion/transient/continuous_2d_d00_p02/mixed_weak_cochain_brick_2d_5_forman_trapezoidal_0p001_1000_potential}
    \caption{Mixed weak, 1}
  \end{subfigure}
  \begin{subfigure}{.24\textwidth}
    \centering
    \includegraphics[scale=.19, page=334]
    {diffusion/transient/continuous_2d_d00_p02/mixed_weak_cochain_brick_2d_5_forman_trapezoidal_0p001_1000_potential}
    \caption{Mixed weak, 334}
  \end{subfigure}
  \begin{subfigure}{.24\textwidth}
    \centering
    \includegraphics[scale=.19, page=667]
    {diffusion/transient/continuous_2d_d00_p02/mixed_weak_cochain_brick_2d_5_forman_trapezoidal_0p001_1000_potential}
    \caption{Mixed weak, 667}
  \end{subfigure}
  \begin{subfigure}{.24\textwidth}
    \centering
    \includegraphics[scale=.19, page=1001]
    {diffusion/transient/continuous_2d_d00_p02/mixed_weak_cochain_brick_2d_5_forman_trapezoidal_0p001_1000_potential}
    \caption{Mixed weak, 1001}
  \end{subfigure}
  \cprotect
  \caption{%
    \verb|diffusion/transient/continuous_2d_d00_p02|
    (\Cref{cmc/diffusion/continuous/transient/examples/2d_d00_p02-example}):
    solutions for potential}
  \label{figure:cmc/diffusion/transient/continuous_2d_d00_p02/brick_2d_5_forman_trapezoidal_0p001_1000_potential}
\end{figure}
\begin{figure}[!ht]
  \begin{subfigure}{.24\textwidth}
    \centering
    \includegraphics[scale=.19, page=1]
    {diffusion/transient/continuous_2d_d00_p02/primal_strong_cochain_brick_2d_5_forman_trapezoidal_0p001_1000_flow_rate}
    \caption{Primal strong, 1}
  \end{subfigure}
  \begin{subfigure}{.24\textwidth}
    \centering
    \includegraphics[scale=.19, page=334]
    {diffusion/transient/continuous_2d_d00_p02/primal_strong_cochain_brick_2d_5_forman_trapezoidal_0p001_1000_flow_rate}
    \caption{Primal strong, 334}
  \end{subfigure}
  \begin{subfigure}{.24\textwidth}
    \centering
    \includegraphics[scale=.19, page=667]
    {diffusion/transient/continuous_2d_d00_p02/primal_strong_cochain_brick_2d_5_forman_trapezoidal_0p001_1000_flow_rate}
    \caption{Primal strong, 667}
  \end{subfigure}
  \begin{subfigure}{.24\textwidth}
    \centering
    \includegraphics[scale=.19, page=1001]
    {diffusion/transient/continuous_2d_d00_p02/primal_strong_cochain_brick_2d_5_forman_trapezoidal_0p001_1000_flow_rate}
    \caption{Primal strong, 1001}
  \end{subfigure}

  \begin{subfigure}{.24\textwidth}
    \centering
    \includegraphics[scale=.19, page=1]
    {diffusion/transient/continuous_2d_d00_p02/primal_weak_cochain_brick_2d_5_forman_trapezoidal_0p001_1000_flow_rate}
    \caption{Primal weak, 1}
  \end{subfigure}
  \begin{subfigure}{.24\textwidth}
    \centering
    \includegraphics[scale=.19, page=334]
    {diffusion/transient/continuous_2d_d00_p02/primal_weak_cochain_brick_2d_5_forman_trapezoidal_0p001_1000_flow_rate}
    \caption{Primal weak, 334}
  \end{subfigure}
  \begin{subfigure}{.24\textwidth}
    \centering
    \includegraphics[scale=.19, page=667]
    {diffusion/transient/continuous_2d_d00_p02/primal_weak_cochain_brick_2d_5_forman_trapezoidal_0p001_1000_flow_rate}
    \caption{Primal weak, 667}
  \end{subfigure}
  \begin{subfigure}{.24\textwidth}
    \centering
    \includegraphics[scale=.19, page=1001]
    {diffusion/transient/continuous_2d_d00_p02/primal_weak_cochain_brick_2d_5_forman_trapezoidal_0p001_1000_flow_rate}
    \caption{Primal weak, 1001}
  \end{subfigure}

  \begin{subfigure}{.24\textwidth}
    \centering
    \includegraphics[scale=.19, page=1]
    {diffusion/transient/continuous_2d_d00_p02/mixed_weak_cochain_brick_2d_5_forman_trapezoidal_0p001_1000_flow_rate}
    \caption{Mixed weak, 1}
  \end{subfigure}
  \begin{subfigure}{.24\textwidth}
    \centering
    \includegraphics[scale=.19, page=334]
    {diffusion/transient/continuous_2d_d00_p02/mixed_weak_cochain_brick_2d_5_forman_trapezoidal_0p001_1000_flow_rate}
    \caption{Mixed weak, 334}
  \end{subfigure}
  \begin{subfigure}{.24\textwidth}
    \centering
    \includegraphics[scale=.19, page=667]
    {diffusion/transient/continuous_2d_d00_p02/mixed_weak_cochain_brick_2d_5_forman_trapezoidal_0p001_1000_flow_rate}
    \caption{Mixed weak, 667}
  \end{subfigure}
  \begin{subfigure}{.24\textwidth}
    \centering
    \includegraphics[scale=.19, page=1001]
    {diffusion/transient/continuous_2d_d00_p02/mixed_weak_cochain_brick_2d_5_forman_trapezoidal_0p001_1000_flow_rate}
    \caption{Mixed weak, 1001}
  \end{subfigure}
  \cprotect
  \caption{%
    \verb|diffusion/transient/continuous_2d_d00_p02|
    (\Cref{cmc/diffusion/continuous/transient/examples/2d_d00_p02-example}):
    solutions for flow rate}
  \label{figure:cmc/diffusion/transient/continuous_2d_d00_p02/brick_2d_5_forman_trapezoidal_0p001_1000_flow_rate}
\end{figure}

\begin{example}
  \label{idec/diffusion/continuous/steady_state/examples/2d_d00_p03-example}
  Consider the steady-state continuous heat transport problem
  (\Cref{idec/diffusion/continuous/steady_state/primal_strong-formulation},
   \Cref{idec/diffusion/continuous/steady_state/primal_weak-formulation},
   \Cref{idec/diffusion/continuous/steady_state/mixed_weak-formulation})
  with input data \verb|2d_d00_p03| in the nomenclature of the C codebase.

  Concretely,
    $X = [0, 1]^2$,
    $\kappa_1 \equiv 1$,
    $f = -2 \, d x \wedge d y$,
    $G_D = \{0, 1\} \times [0, 1]$,
    $G_N = [0, 1] \times \{0, 1\}$,
    $g_D \equiv 0$,
    $g_N \equiv 0$.

  This problem has the following exact solution:
  \begin{subequations}
    \begin{alignat}{3}
      & u(x, y) && = && x (x - 1), \\
      & q(x, y) && = && (2 x - 1)\, d y.
    \end{alignat}
  \end{subequations}
  Consider a mesh $M$ for $X$ consisting of $2 \times 2$ squares (each axis is
  divided into $2$ segments) with Forman subdivision $K$ ($4 \times 4$ squares).
  Its potential and flow on $K$ consisting of the exact solution and the $3$
  discussed cochain methods are shown on
  \Cref{figure:idec/diffusion/steady_state/continuous_2d_d00_p03/brick_2d_2_forman_potential}
  and
  \Cref{figure:idec/diffusion/steady_state/continuous_2d_d00_p03/brick_2d_2_forman_flow_rate}.
\end{example}

\begin{figure}[!ht]
  \begin{subfigure}{.24\textwidth}
    \centering
    \includegraphics[scale=.23]
    {diffusion/steady_state/continuous_2d_d00_p03/exact_brick_2d_2_forman_potential}
    \caption{Exact}
  \end{subfigure}
  \begin{subfigure}{.24\textwidth}
    \centering
    \includegraphics[scale=.23]
    {diffusion/steady_state/continuous_2d_d00_p03/primal_strong_cochain_brick_2d_2_forman_potential}
    \caption{Primal strong}
  \end{subfigure}
  \begin{subfigure}{.24\textwidth}
    \centering
    \includegraphics[scale=.23]
    {diffusion/steady_state/continuous_2d_d00_p03/primal_weak_cochain_brick_2d_2_forman_potential}
    \caption{Primal weak}
  \end{subfigure}
  \begin{subfigure}{.24\textwidth}
    \centering
    \includegraphics[scale=.23]
    {diffusion/steady_state/continuous_2d_d00_p03/mixed_weak_cochain_brick_2d_2_forman_potential}
    \caption{Mixed weak}
  \end{subfigure}
  \cprotect
  \caption{%
    \verb|diffusion/steady_state/continuous_2d_d00_p03|
    (\Cref{cmc/diffusion/continuous/steady_state/examples/2d_d00_p03-example}):
    solutions for potential}
  \label{figure:cmc/diffusion/steady_state/continuous_2d_d00_p03/brick_2d_2_forman_potential}
\end{figure}
\begin{figure}[!ht]
  \begin{subfigure}{.24\textwidth}
    \centering
    \includegraphics[scale=.23]
    {diffusion/steady_state/continuous_2d_d00_p03/exact_brick_2d_2_forman_flow_rate}
    \caption{Exact}
  \end{subfigure}
  \begin{subfigure}{.24\textwidth}
    \centering
    \includegraphics[scale=.23]
    {diffusion/steady_state/continuous_2d_d00_p03/primal_strong_cochain_brick_2d_2_forman_flow_rate}
    \caption{Primal strong}
  \end{subfigure}
  \begin{subfigure}{.24\textwidth}
    \centering
    \includegraphics[scale=.23]
    {diffusion/steady_state/continuous_2d_d00_p03/primal_weak_cochain_brick_2d_2_forman_flow_rate}
    \caption{Primal weak}
  \end{subfigure}
  \begin{subfigure}{.24\textwidth}
    \centering
    \includegraphics[scale=.23]
    {diffusion/steady_state/continuous_2d_d00_p03/mixed_weak_cochain_brick_2d_2_forman_flow_rate}
    \caption{Mixed weak}
  \end{subfigure}
  \cprotect
  \caption{%
    \verb|diffusion/steady_state/continuous_2d_d00_p03|
    (\Cref{cmc/diffusion/continuous/steady_state/examples/2d_d00_p03-example}):
    solutions for flow rate}
  \label{figure:cmc/diffusion/steady_state/continuous_2d_d00_p03/brick_2d_2_forman_flow_rate}
\end{figure}

\begin{example}
  \label{idec/diffusion/continuous/steady_state/examples/2d_d00_p04-example}
  Consider the steady-state continuous heat transport problem
  (\Cref{idec/diffusion/continuous/steady_state/primal_strong-formulation},
   \Cref{idec/diffusion/continuous/steady_state/primal_weak-formulation},
   \Cref{idec/diffusion/continuous/steady_state/mixed_weak-formulation})
  with input data \verb|2d_d00_p04| in the nomenclature of the C codebase.

  Concretely,
    $X = [0, 1]^2$,
    $\kappa_1 \equiv 1$,
    $f = -4 \, d x \wedge d y$,
    $G_D = \{0, 1\} \times [0, 1]$,
    $G_N = [0, 1] \times \{0, 1\}$,
    $g_D(x, y) = y (y - 1)$,
    $g_N(x, y) = (1 - 2 x) \, d x$.

  This problem has the following exact solution:
  \begin{subequations}
    \begin{alignat}{3}
      & u(x, y) && = && x (x - 1) + y (y - 1), \\
      & q(x, y) && = && (1 - 2 y)\, d x + (2 x - 1)\, d y.
    \end{alignat}
  \end{subequations}
  Consider a mesh $M$ for $X$ consisting of $5 \times 5$ squares (each axis is
  divided into $5$ segments) with Forman subdivision $K$
  ($10 \times 10$ squares).
  Its potential and flow on $K$ consisting of the exact solution and the $3$
  discussed cochain methods are shown on
  \Cref{figure:idec/diffusion/steady_state/continuous_2d_d00_p04/brick_2d_5_forman_potential}
  and
  \Cref{figure:idec/diffusion/steady_state/continuous_2d_d00_p04/brick_2d_5_forman_flow_rate}.
\end{example}

\begin{figure}[!ht]
  \begin{subfigure}{.32\textwidth}
    \centering
    \includegraphics[scale=.2, page=1]
    {diffusion/transient/continuous_2d_d00_p04/primal_strong_cochain_brick_2d_5_forman_trapezoidal_0p001_2500_potential}
    \caption{Primal strong potential, 1}
  \end{subfigure}
  \begin{subfigure}{.32\textwidth}
    \centering
    \includegraphics[scale=.2, page=1251]
    {diffusion/transient/continuous_2d_d00_p04/primal_strong_cochain_brick_2d_5_forman_trapezoidal_0p001_2500_potential}
    \caption{Primal strong potential, 1251}
  \end{subfigure}
  \begin{subfigure}{.32\textwidth}
    \centering
    \includegraphics[scale=.2, page=2501]
    {diffusion/transient/continuous_2d_d00_p04/primal_strong_cochain_brick_2d_5_forman_trapezoidal_0p001_2500_potential}
    \caption{Primal strong potential, 2501}
  \end{subfigure}

  \begin{subfigure}{.32\textwidth}
    \centering
    \includegraphics[scale=.2, page=1]
    {diffusion/transient/continuous_2d_d00_p04/mixed_weak_cochain_brick_2d_5_forman_trapezoidal_0p001_2500_potential}
    \caption{Mixed weak potential, 1}
  \end{subfigure}
  \begin{subfigure}{.32\textwidth}
    \centering
    \includegraphics[scale=.2, page=1251]
    {diffusion/transient/continuous_2d_d00_p04/mixed_weak_cochain_brick_2d_5_forman_trapezoidal_0p001_2500_potential}
    \caption{Mixed weak potential, 1251}
  \end{subfigure}
  \begin{subfigure}{.32\textwidth}
    \centering
    \includegraphics[scale=.2, page=2501]
    {diffusion/transient/continuous_2d_d00_p04/mixed_weak_cochain_brick_2d_5_forman_trapezoidal_0p001_2500_potential}
    \caption{Mixed weak potential, 2501}
  \end{subfigure}

  \begin{subfigure}{.32\textwidth}
    \centering
    \includegraphics[scale=.2, page=1]
    {diffusion/transient/continuous_2d_d00_p04/primal_strong_cochain_brick_2d_5_forman_trapezoidal_0p001_2500_flow}
    \caption{Primal strong flow, 1}
  \end{subfigure}
  \begin{subfigure}{.32\textwidth}
    \centering
    \includegraphics[scale=.2, page=1251]
    {diffusion/transient/continuous_2d_d00_p04/primal_strong_cochain_brick_2d_5_forman_trapezoidal_0p001_2500_flow}
    \caption{Primal strong flow, 1251}
  \end{subfigure}
  \begin{subfigure}{.32\textwidth}
    \centering
    \includegraphics[scale=.2, page=2501]
    {diffusion/transient/continuous_2d_d00_p04/primal_strong_cochain_brick_2d_5_forman_trapezoidal_0p001_2500_flow}
    \caption{Primal strong flow, 2501}
  \end{subfigure}
  
  \begin{subfigure}{.32\textwidth}
    \centering
    \includegraphics[scale=.2, page=1]
    {diffusion/transient/continuous_2d_d00_p04/mixed_weak_cochain_brick_2d_5_forman_trapezoidal_0p001_2500_flow}
    \caption{Mixed weak flow, 1}
  \end{subfigure}
  \begin{subfigure}{.32\textwidth}
    \centering
    \includegraphics[scale=.2, page=1251]
    {diffusion/transient/continuous_2d_d00_p04/mixed_weak_cochain_brick_2d_5_forman_trapezoidal_0p001_2500_flow}
    \caption{Mixed weak flow, 1251}
  \end{subfigure}
  \begin{subfigure}{.32\textwidth}
    \centering
    \includegraphics[scale=.2, page=2501]
    {diffusion/transient/continuous_2d_d00_p04/mixed_weak_cochain_brick_2d_5_forman_trapezoidal_0p001_2500_flow}
    \caption{Mixed weak flow, 2501}
  \end{subfigure}
  \cprotect\caption{Potential and flow for the \verb|2d_d00_p04| problem}
  \label{figure:idec/diffusion/transient/continuous_2d_d00_p04/brick_2d_5_forman_trapezoidal_0p001_2500}
\end{figure}

\begin{example}
  Consider the steady-state continuous heat transport problem
  (\Cref{idec/diffusion/continuous/steady_state/primal_strong-formulation},
   \Cref{idec/diffusion/continuous/steady_state/primal_weak-formulation},
   \Cref{idec/diffusion/continuous/steady_state/mixed_weak-formulation})
  with input data \verb|2d_d00_p05| in the nomenclature of the C codebase.

  Concretely,
    $X = [0, 1]^2$,
    $\kappa_1 \equiv 1$,
    $f(x, y) = \sin(\pi x) \sin(\pi y) \, d x \wedge d y$,
    $G_D = \partial X$,
    $G_N = \emptyset$,
    $g_D \equiv 0$.

  This problem has the following exact solution:
  \begin{subequations}
    \begin{alignat}{3}
      & u(x, y) && = && \frac{\sin(\pi x) \sin(\pi y)}{2 \pi^2}, \\
      & q(x, y) && = &&
        \frac{1}{2 \pi}
        ((- \sin(\pi x) \cos(\pi y) \, d x + \sin(\pi y) \cos(\pi x)\, d y)).
    \end{alignat}
  \end{subequations}
  Consider a mesh $M$ for $X$ consisting of $5 \times 5$ squares (each axis is
  divided into $5$ segments) with Forman subdivision $K$
  ($10 \times 10$ squares).
  Its potential and flow on $K$ consisting of the exact solution and the $3$
  discussed cochain methods are shown on
  \Cref{figure:idec/diffusion/steady_state/continuous_2d_d00_p05/brick_2d_5_forman}.
\end{example}

\begin{figure}[!ht]
  \begin{subfigure}{.22\textwidth}
    \centering
    \includegraphics[scale=.2]
    {diffusion/steady_state/continuous_2d_d00_p05/exact_brick_2d_5_forman_potential}
    \caption{Exact potential}
  \end{subfigure}
  \begin{subfigure}{.22\textwidth}
    \centering
    \includegraphics[scale=.2]
    {diffusion/steady_state/continuous_2d_d00_p05/primal_strong_cochain_brick_2d_5_forman_potential}
    \caption{Primal strong potential}
  \end{subfigure}
  \begin{subfigure}{.22\textwidth}
    \centering
    \includegraphics[scale=.2]
    {diffusion/steady_state/continuous_2d_d00_p05/primal_weak_cochain_brick_2d_5_forman_potential}
    \caption{Primal weak potential}
  \end{subfigure}
  \begin{subfigure}{.22\textwidth}
    \centering
    \includegraphics[scale=.2]
    {diffusion/steady_state/continuous_2d_d00_p05/mixed_weak_cochain_brick_2d_5_forman_potential}
    \caption{Mixed weak potential}
  \end{subfigure}

  \begin{subfigure}{.22\textwidth}
    \centering
    \includegraphics[scale=.2]
    {diffusion/steady_state/continuous_2d_d00_p05/exact_brick_2d_5_forman_flow}
    \caption{Exact flow}
  \end{subfigure}
  \begin{subfigure}{.22\textwidth}
    \centering
    \includegraphics[scale=.2]
    {diffusion/steady_state/continuous_2d_d00_p05/primal_strong_cochain_brick_2d_5_forman_flow}
    \caption{Primal strong flow}
  \end{subfigure}
  \begin{subfigure}{.22\textwidth}
    \centering
    \includegraphics[scale=.2]
    {diffusion/steady_state/continuous_2d_d00_p05/primal_weak_cochain_brick_2d_5_forman_flow}
    \caption{Primal weak flow}
  \end{subfigure}
  \begin{subfigure}{.22\textwidth}
    \centering
    \includegraphics[scale=.2]
    {diffusion/steady_state/continuous_2d_d00_p05/mixed_weak_cochain_brick_2d_5_forman_flow}
    \caption{Mixed weak flow}
  \end{subfigure}
  \cprotect\caption{Potential and flow for the \verb|2d_d00_p05| problem}
  \label{figure:idec/diffusion/steady_state/continuous_2d_d00_p05/brick_2d_5_forman}
\end{figure}

\begin{example}
  \label{cmc/diffusion/continuous/steady_state/examples/2d_d01_p00-example}
  Consider the steady-state continuous heat transport problem
  (\Cref{cmc/diffusion/continuous/steady_state/primal_strong-formulation},
   \Cref{cmc/diffusion/continuous/steady_state/primal_weak-formulation},
   \Cref{cmc/diffusion/continuous/steady_state/mixed_weak-formulation})
  with input data \verb|2d_d01_p00| in the nomenclature of the C codebase.

  Concretely,
    $X = {\rm polygon}((-5, 0), (0, -5), (5, 0), (0, 5)$,
    $\kappa_1 \equiv 6$,
    $f \equiv 0$,
    $G_D = {\rm line}((-5, 0), (0, -5)) \cup {\rm line}((5, 0), (0, 5))$,
    $G_N = {\rm line}((0, -5), (5, 0)) \cup {\rm line}((0, 5), (-5, 0))$,
    $g_D(x, y) =
      \begin{cases}
        100, & (x, y) \in {\rm line}((-5, 0), (0, -5)) \\
        0, & (x, y) \in {\rm line}((5, 0), (0, 5))
      \end{cases}$,
    $g_N \equiv 0$.

  This problem has the following exact solution:
  \begin{subequations}
    \begin{alignat}{3}
      & u(x, y) && = && 50 (1 - (x + y) / 5), \\
      & q(x, y) && = && 60 (d x - d y).
    \end{alignat}
  \end{subequations}
  Consider a mesh $M$ for $X$ consisting of $4 \times 4$ squares (each axis is
  divided into $4$ segments) with Forman subdivision $K$ ($8 \times 8$ squares).
  Its potential and flow rate on $K$ consisting of the exact solution and the
  $3$ discussed cochain methods are shown on
  \Cref{figure:cmc/diffusion/steady_state/continuous_2d_d01_p00/square_8_potential}
  and
  \Cref{figure:cmc/diffusion/steady_state/continuous_2d_d01_p00/square_8_flow_rate}.
\end{example}

\begin{figure}[!ht]
  \begin{subfigure}{.22\textwidth}
    \centering
    \includegraphics[scale=.2]
    {diffusion/steady_state/continuous_2d_d01_p00/exact_square_8_potential}
    \caption{Exact potential}
  \end{subfigure}
  \begin{subfigure}{.22\textwidth}
    \centering
    \includegraphics[scale=.2]
    {diffusion/steady_state/continuous_2d_d01_p00/primal_strong_cochain_square_8_potential}
    \caption{Primal strong potential}
  \end{subfigure}
  \begin{subfigure}{.22\textwidth}
    \centering
    \includegraphics[scale=.2]
    {diffusion/steady_state/continuous_2d_d01_p00/primal_weak_cochain_square_8_potential}
    \caption{Primal weak potential}
  \end{subfigure}
  \begin{subfigure}{.22\textwidth}
    \centering
    \includegraphics[scale=.2]
    {diffusion/steady_state/continuous_2d_d01_p00/mixed_weak_cochain_square_8_potential}
    \caption{Mixed weak potential}
  \end{subfigure}

  \begin{subfigure}{.22\textwidth}
    \centering
    \includegraphics[scale=.2]
    {diffusion/steady_state/continuous_2d_d01_p00/exact_square_8_flow}
    \caption{Exact flow}
  \end{subfigure}
  \begin{subfigure}{.22\textwidth}
    \centering
    \includegraphics[scale=.2]
    {diffusion/steady_state/continuous_2d_d01_p00/primal_strong_cochain_square_8_flow}
    \caption{Primal strong flow}
  \end{subfigure}
  \begin{subfigure}{.22\textwidth}
    \centering
    \includegraphics[scale=.2]
    {diffusion/steady_state/continuous_2d_d01_p00/primal_weak_cochain_square_8_flow}
    \caption{Primal weak flow}
  \end{subfigure}
  \begin{subfigure}{.22\textwidth}
    \centering
    \includegraphics[scale=.2]
    {diffusion/steady_state/continuous_2d_d01_p00/mixed_weak_cochain_square_8_flow}
    \caption{Mixed weak flow}
  \end{subfigure}
  \cprotect\caption{Potential and flow for the \verb|2d_d01_p00| problem}
  \label{figure:idec/diffusion/steady_state/continuous_2d_d01_p00/square_8}
\end{figure}

\begin{example}
  Consider the steady-state continuous heat transport problem
  (\Cref{idec/diffusion/continuous/steady_state/primal_strong-formulation},
   \Cref{idec/diffusion/continuous/steady_state/primal_weak-formulation},
   \Cref{idec/diffusion/continuous/steady_state/mixed_weak-formulation})
  with input data \verb|2d_d02_p00| in the nomenclature of the C codebase.

  Concretely,
    $X = [0, 20] \times [0, 15]$,
    $\kappa_1 \equiv 6$,
    $f \equiv 0$,
    $G_D = \{0, 20\} \times [0, 15]$,
    $G_N = [0, 20] \times \{0, 15\}$,
    $g_D(x, y) = \begin{cases} 100, & x = 0 \\ 0, & x = 20 \end{cases}$,
    $g_N \equiv 0$.

  This problem has the following exact solution:
  \begin{subequations}
    \begin{alignat}{3}
      & u(x, y) && = && 5 (20 - x), \\
      & q(x, y) && = && -30 \, d y.
    \end{alignat}
  \end{subequations}
  For this problem I use a mesh $M$ generated by
  \href{https://neper.info/}{Neper} with Forman subdivision $K$.
  Its potential and flow on $K$ consisting of the exact solution and the $3$
  discussed cochain methods are shown on
  \Cref{figure:idec/diffusion/steady_state/continuous_2d_d02_p00/2d_10_grains_forman}.
\end{example}

\begin{figure}[!ht]
  \begin{subfigure}{.32\textwidth}
    \centering
    \includegraphics[scale=.2, page=1]
    {diffusion/transient/continuous_2d_d02_p00/primal_strong_cochain_2d_10_grains_forman_trapezoidal_0p05_1000_potential}
    \caption{Primal strong potential, 1}
  \end{subfigure}
  \begin{subfigure}{.32\textwidth}
    \centering
    \includegraphics[scale=.2, page=501]
    {diffusion/transient/continuous_2d_d02_p00/primal_strong_cochain_2d_10_grains_forman_trapezoidal_0p05_1000_potential}
    \caption{Primal strong potential, 501}
  \end{subfigure}
  \begin{subfigure}{.32\textwidth}
    \centering
    \includegraphics[scale=.2, page=1001]
    {diffusion/transient/continuous_2d_d02_p00/primal_strong_cochain_2d_10_grains_forman_trapezoidal_0p05_1000_potential}
    \caption{Primal strong potential, 1001}
  \end{subfigure}

  \begin{subfigure}{.32\textwidth}
    \centering
    \includegraphics[scale=.2, page=1]
    {diffusion/transient/continuous_2d_d02_p00/primal_weak_cochain_2d_10_grains_forman_trapezoidal_0p05_1000_potential}
    \caption{Primal weak potential, 1}
  \end{subfigure}
  \begin{subfigure}{.32\textwidth}
    \centering
    \includegraphics[scale=.2, page=501]
    {diffusion/transient/continuous_2d_d02_p00/primal_weak_cochain_2d_10_grains_forman_trapezoidal_0p05_1000_potential}
    \caption{Primal weak potential, 501}
  \end{subfigure}
  \begin{subfigure}{.32\textwidth}
    \centering
    \includegraphics[scale=.2, page=1001]
    {diffusion/transient/continuous_2d_d02_p00/primal_weak_cochain_2d_10_grains_forman_trapezoidal_0p05_1000_potential}
    \caption{Primal weak potential, 1001}
  \end{subfigure}

  \begin{subfigure}{.32\textwidth}
    \centering
    \includegraphics[scale=.2, page=1]
    {diffusion/transient/continuous_2d_d02_p00/mixed_weak_cochain_2d_10_grains_forman_trapezoidal_0p05_1000_potential}
    \caption{Mixed weak potential, 1}
  \end{subfigure}
  \begin{subfigure}{.32\textwidth}
    \centering
    \includegraphics[scale=.2, page=501]
    {diffusion/transient/continuous_2d_d02_p00/mixed_weak_cochain_2d_10_grains_forman_trapezoidal_0p05_1000_potential}
    \caption{Mixed weak potential, 501}
  \end{subfigure}
  \begin{subfigure}{.32\textwidth}
    \centering
    \includegraphics[scale=.2, page=1001]
    {diffusion/transient/continuous_2d_d02_p00/mixed_weak_cochain_2d_10_grains_forman_trapezoidal_0p05_1000_potential}
    \caption{Mixed weak potential, 1001}
  \end{subfigure}

  \begin{subfigure}{.32\textwidth}
    \centering
    \includegraphics[scale=.2, page=1]
    {diffusion/transient/continuous_2d_d02_p00/primal_strong_cochain_2d_10_grains_forman_trapezoidal_0p05_1000_flow}
    \caption{Primal strong flow, 1}
  \end{subfigure}
  \begin{subfigure}{.32\textwidth}
    \centering
    \includegraphics[scale=.2, page=501]
    {diffusion/transient/continuous_2d_d02_p00/primal_strong_cochain_2d_10_grains_forman_trapezoidal_0p05_1000_flow}
    \caption{Primal strong flow, 501}
  \end{subfigure}
  \begin{subfigure}{.32\textwidth}
    \centering
    \includegraphics[scale=.2, page=1001]
    {diffusion/transient/continuous_2d_d02_p00/primal_strong_cochain_2d_10_grains_forman_trapezoidal_0p05_1000_flow}
    \caption{Primal strong flow, 1001}
  \end{subfigure}

  \begin{subfigure}{.32\textwidth}
    \centering
    \includegraphics[scale=.2, page=1]
    {diffusion/transient/continuous_2d_d02_p00/primal_weak_cochain_2d_10_grains_forman_trapezoidal_0p05_1000_flow}
    \caption{Primal weak flow, 1}
  \end{subfigure}
  \begin{subfigure}{.32\textwidth}
    \centering
    \includegraphics[scale=.2, page=501]
    {diffusion/transient/continuous_2d_d02_p00/primal_weak_cochain_2d_10_grains_forman_trapezoidal_0p05_1000_flow}
    \caption{Primal weak flow, 501}
  \end{subfigure}
  \begin{subfigure}{.32\textwidth}
    \centering
    \includegraphics[scale=.2, page=1001]
    {diffusion/transient/continuous_2d_d02_p00/primal_weak_cochain_2d_10_grains_forman_trapezoidal_0p05_1000_flow}
    \caption{Primal weak flow, 1001}
  \end{subfigure}
  
  \begin{subfigure}{.32\textwidth}
    \centering
    \includegraphics[scale=.2, page=1]
    {diffusion/transient/continuous_2d_d02_p00/mixed_weak_cochain_2d_10_grains_forman_trapezoidal_0p05_1000_flow}
    \caption{Mixed weak flow, 1}
  \end{subfigure}
  \begin{subfigure}{.32\textwidth}
    \centering
    \includegraphics[scale=.2, page=501]
    {diffusion/transient/continuous_2d_d02_p00/mixed_weak_cochain_2d_10_grains_forman_trapezoidal_0p05_1000_flow}
    \caption{Mixed weak flow, 501}
  \end{subfigure}
  \begin{subfigure}{.32\textwidth}
    \centering
    \includegraphics[scale=.2, page=1001]
    {diffusion/transient/continuous_2d_d02_p00/mixed_weak_cochain_2d_10_grains_forman_trapezoidal_0p05_1000_flow}
    \caption{Mixed weak flow, 1001}
  \end{subfigure}
  \cprotect\caption{Potential and flow for the \verb|2d_d02_p00| problem}
  \label{figure:idec/diffusion/transient/continuous_2d_d02_p00/2d_10_grains_forman_trapezoidal_0p05_1000}
\end{figure}

\begin{example}
  \label{idec/diffusion/continuous/steady_state/examples/2d_d02_p01-example}
  Consider the steady-state continuous heat transport problem
  (\Cref{idec/diffusion/continuous/steady_state/primal_strong-formulation},
   \Cref{idec/diffusion/continuous/steady_state/primal_weak-formulation},
   \Cref{idec/diffusion/continuous/steady_state/mixed_weak-formulation})
  with input data \verb|2d_d02_p01| in the nomenclature of the C codebase.

  Concretely,
    $X = [0, 20] \times [0, 15]$,
    $\kappa_1 \equiv 6$,
    $f \equiv 0$,
    $G_D = \{0, 20\} \times [0, 15]$,
    $G_N = [0, 20] \times \{0, 15\}$,
    $g_D(x, y) = \begin{cases} 0, & x = 0 \\ 100, & x = 20 \end{cases}$,
    $g_N \equiv 0$.

  This problem has the following exact solution:
  \begin{subequations}
    \begin{alignat}{3}
      & u(x, y) && = && 5 x, \\
      & q(x, y) && = && 30 \, d y.
    \end{alignat}
  \end{subequations}
  For this problem I use a mesh $M$ generated by
  \href{https://neper.info/}{Neper} with Forman subdivision $K$.
  Its potential and flow on $K$ consisting of the exact solution and the $3$
  discussed cochain methods are shown on
  \Cref{figure:idec/diffusion/steady_state/continuous_2d_d02_p01/2d_10_grains_forman_potential}
  and
  \Cref{figure:idec/diffusion/steady_state/continuous_2d_d02_p01/2d_10_grains_forman_flow_rate}.
\end{example}

\begin{figure}[!ht]
  \begin{subfigure}{.24\textwidth}
    \centering
    \includegraphics[scale=.23]
    {diffusion/steady_state/continuous_2d_d02_p01/exact_2d_10_grains_forman_potential}
    \caption{Exact}
  \end{subfigure}
  \begin{subfigure}{.24\textwidth}
    \centering
    \includegraphics[scale=.23]
    {diffusion/steady_state/continuous_2d_d02_p01/primal_strong_cochain_2d_10_grains_forman_potential}
    \caption{Primal strong}
  \end{subfigure}
  \begin{subfigure}{.24\textwidth}
    \centering
    \includegraphics[scale=.23]
    {diffusion/steady_state/continuous_2d_d02_p01/primal_weak_cochain_2d_10_grains_forman_potential}
    \caption{Primal weak}
  \end{subfigure}
  \begin{subfigure}{.24\textwidth}
    \centering
    \includegraphics[scale=.23]
    {diffusion/steady_state/continuous_2d_d02_p01/mixed_weak_cochain_2d_10_grains_forman_potential}
    \caption{Mixed weak}
  \end{subfigure}
  \cprotect
  \caption{%
    \verb|diffusion/steady_state/continuous_2d_d02_p01|
    (\Cref{idec/diffusion/continuous/steady_state/examples/2d_d02_p01-example}):
    solutions for potential}
  \label{figure:idec/diffusion/steady_state/continuous_2d_d02_p01/2d_10_grains_forman_potential}
\end{figure}
\begin{figure}[!ht]
  \begin{subfigure}{.24\textwidth}
    \centering
    \includegraphics[scale=.23]
    {diffusion/steady_state/continuous_2d_d02_p01/exact_2d_10_grains_forman_flow_rate}
    \caption{Exact}
  \end{subfigure}
  \begin{subfigure}{.24\textwidth}
    \centering
    \includegraphics[scale=.23]
    {diffusion/steady_state/continuous_2d_d02_p01/primal_strong_cochain_2d_10_grains_forman_flow_rate}
    \caption{Primal strong}
  \end{subfigure}
  \begin{subfigure}{.24\textwidth}
    \centering
    \includegraphics[scale=.23]
    {diffusion/steady_state/continuous_2d_d02_p01/primal_weak_cochain_2d_10_grains_forman_flow_rate}
    \caption{Primal weak}
  \end{subfigure}
  \begin{subfigure}{.24\textwidth}
    \centering
    \includegraphics[scale=.23]
    {diffusion/steady_state/continuous_2d_d02_p01/mixed_weak_cochain_2d_10_grains_forman_flow_rate}
    \caption{Mixed weak}
  \end{subfigure}
  \cprotect
  \caption{%
    \verb|diffusion/steady_state/continuous_2d_d02_p01|
    (\Cref{idec/diffusion/continuous/steady_state/examples/2d_d02_p01-example}):
    solutions for flow rate}
  \label{figure:idec/diffusion/steady_state/continuous_2d_d02_p01/2d_10_grains_forman_flow_rate}
\end{figure}

\begin{example}
  \label{idec/diffusion/continuous/steady_state/examples/2d_d03_p00-example}
  Consider the steady-state continuous heat transport problem
  (\Cref{idec/diffusion/continuous/steady_state/primal_strong-formulation},
   \Cref{idec/diffusion/continuous/steady_state/primal_weak-formulation},
   \Cref{idec/diffusion/continuous/steady_state/mixed_weak-formulation})
  with input data \verb|2d_d03_p00| in the nomenclature of the C codebase.

  Concretely,
    $X = \set{(x, y) \in \R^2}{x^2 + y^2 \leq 1}$,
    $\kappa_1 \equiv 1$,
    $f = - 4 \, d x \wedge d y$,
    $G_D = \partial X$,
    $G_N = \emptyset$,
    $g_D \equiv 1$.

  This problem has the following exact solution:
  \begin{subequations}
    \begin{alignat}{3}
      & u(x, y) && = && x^2 + y^2, \\
      & q(x, y) && = && -2 y \, d x + 2 x \, d y.
    \end{alignat}
  \end{subequations}
  Consider a mesh $M$ for $X$ consisting of $n_a$ rays and $n_d$ disks
  with Forman subdivision $K$.
  Its potential and flow on $K$ consisting of the exact solution and the $2$
  of the discussed cochain methods (no primal strong) are shown on
  \Cref{figure:idec/diffusion/steady_state/continuous_2d_d03_p00/circular_4_3_forman_potential},
  \Cref{figure:idec/diffusion/steady_state/continuous_2d_d03_p00/circular_4_3_forman_flow_rate}
  ($(n_a, n_d) = (4, 3)$)
  and
  \Cref{figure:idec/diffusion/steady_state/continuous_2d_d03_p00/circular_18_10_forman_potential},
  \Cref{figure:idec/diffusion/steady_state/continuous_2d_d03_p00/circular_18_10_forman_flow_rate}
  ($(n_a, n_d) = (18, 10)$).
\end{example}

\begin{figure}[!ht]
  \begin{subfigure}{.32\textwidth}
    \centering
    \includegraphics[scale=.32]
    {diffusion/steady_state/continuous_2d_d03_p00/exact_circular_4_3_forman_potential}
    \caption{Exact}
  \end{subfigure}
  \begin{subfigure}{.32\textwidth}
    \centering
    \includegraphics[scale=.32]
    {diffusion/steady_state/continuous_2d_d03_p00/primal_weak_cochain_circular_4_3_forman_potential}
    \caption{Primal weak}
  \end{subfigure}
  \begin{subfigure}{.32\textwidth}
    \centering
    \includegraphics[scale=.32]
    {diffusion/steady_state/continuous_2d_d03_p00/mixed_weak_cochain_circular_4_3_forman_potential}
    \caption{Mixed weak}
  \end{subfigure}
  \cprotect
  \caption{%
    \verb|diffusion/steady_state/continuous_2d_d03_p00|
    (\Cref{idec/diffusion/continuous/steady_state/examples/2d_d03_p00-example}):
    solutions for potential on mesh \verb|circular_4_3_forman|}
  \label{figure:idec/diffusion/steady_state/continuous_2d_d03_p00/circular_4_3_forman_potential}
\end{figure}
\begin{figure}[!ht]
  \begin{subfigure}{.32\textwidth}
    \centering
    \includegraphics[scale=.32]
    {diffusion/steady_state/continuous_2d_d03_p00/exact_circular_4_3_forman_flow}
    \caption{Exact}
  \end{subfigure}
  \begin{subfigure}{.32\textwidth}
    \centering
    \includegraphics[scale=.32]
    {diffusion/steady_state/continuous_2d_d03_p00/primal_weak_cochain_circular_4_3_forman_flow}
    \caption{Primal weak}
  \end{subfigure}
  \begin{subfigure}{.32\textwidth}
    \centering
    \includegraphics[scale=.32]
    {diffusion/steady_state/continuous_2d_d03_p00/mixed_weak_cochain_circular_4_3_forman_flow}
    \caption{Mixed weak}
  \end{subfigure}
  \cprotect
  \caption{%
    \verb|diffusion/steady_state/continuous_2d_d03_p00|
    (\Cref{idec/diffusion/continuous/steady_state/examples/2d_d03_p00-example}):
    solutions for flow rate on mesh \verb|circular_4_3_forman|}
  \label{figure:idec/diffusion/steady_state/continuous_2d_d03_p00/circular_4_3_forman_flow_rate}
\end{figure}
\begin{figure}[!ht]
  \begin{subfigure}{.32\textwidth}
    \centering
    \includegraphics[scale=.32]
    {diffusion/steady_state/continuous_2d_d03_p00/exact_circular_18_10_forman_potential}
    \caption{Exact}
  \end{subfigure}
  \begin{subfigure}{.32\textwidth}
    \centering
    \includegraphics[scale=.32]
    {diffusion/steady_state/continuous_2d_d03_p00/primal_weak_cochain_circular_18_10_forman_potential}
    \caption{Primal weak}
  \end{subfigure}
  \begin{subfigure}{.32\textwidth}
    \centering
    \includegraphics[scale=.32]
    {diffusion/steady_state/continuous_2d_d03_p00/mixed_weak_cochain_circular_18_10_forman_potential}
    \caption{Mixed weak}
  \end{subfigure}
  \cprotect
  \caption{%
    \verb|diffusion/steady_state/continuous_2d_d03_p00|
    (\Cref{idec/diffusion/continuous/steady_state/examples/2d_d03_p00-example}):
    solutions for potential on mesh \verb|circular_18_10_forman|}
  \label{figure:idec/diffusion/steady_state/continuous_2d_d03_p00/circular_18_10_forman_potential}
\end{figure}
\begin{figure}[!ht]
  \begin{subfigure}{.32\textwidth}
    \centering
    \includegraphics[scale=.32]
    {diffusion/steady_state/continuous_2d_d03_p00/exact_circular_18_10_forman_flow}
    \caption{Exact}
  \end{subfigure}
  \begin{subfigure}{.32\textwidth}
    \centering
    \includegraphics[scale=.32]
    {diffusion/steady_state/continuous_2d_d03_p00/primal_weak_cochain_circular_18_10_forman_flow}
    \caption{Primal weak}
  \end{subfigure}
  \begin{subfigure}{.32\textwidth}
    \centering
    \includegraphics[scale=.32]
    {diffusion/steady_state/continuous_2d_d03_p00/mixed_weak_cochain_circular_18_10_forman_flow}
    \caption{Mixed weak}
  \end{subfigure}
  \cprotect
  \caption{%
    \verb|diffusion/steady_state/continuous_2d_d03_p00|
    (\Cref{idec/diffusion/continuous/steady_state/examples/2d_d03_p00-example}):
    solutions for flow rate on mesh \verb|circular_18_10_forman|}
  \label{figure:idec/diffusion/steady_state/continuous_2d_d03_p00/circular_18_10_forman_flow_rate}
\end{figure}

\begin{example}
  \label{idec/diffusion/continuous/transient/examples/2d_d03_p01-example}
  Consider the transient continuous heat transport problem
  (\Cref{idec/diffusion/continuous/transient/primal_strong-formulation},
   \Cref{idec/diffusion/continuous/transient/primal_weak-formulation},
   \Cref{idec/diffusion/continuous/transient/mixed_weak-formulation})
  with input data \verb|2d_d03_p01| in the nomenclature of the C codebase.

  Concretely,
    $X = \set{(x, y) \in \R^2}{x^2 + y^2 \leq 1}$,
    $\pi_0 \equiv 4$,
    $\kappa_1 \equiv 1$,
    $u_0(x, y) = 2 - (x^2 + y^2)$,
    $f \equiv -4\, d x \wedge d y$,
    $G_D = \set{(x, y) \in \partial X}{x \geq 0}$,
    $G_N = \set{(x, y) \in \partial X}{x \leq 0}$,
    $g_D \equiv 1$,
    $g_N(t) = 2 t \, d t$
    (with respect to the $(x, y) = (\cos(t), \sin(t))$ coordinates).

  This problem has the following exact solution in steady-state:
  \begin{subequations}
    \begin{alignat}{3}
      & u(x, y) && = && x^2 + y^2, \\
      & q(x, y) && = && -2 y\, d x + 2 x\, d y.
    \end{alignat}
  \end{subequations}
  Consider a mesh $M$ for $X$ consisting of $n_a$ rays and $n_d$ disks
  with Forman subdivision $K$.
  Its potential and flow rate on $K$ consisting of the $3$ discussed cochain
  methods for $(n_a, n_d) = (4, 3)$ are shown on
  \Cref{figure:idec/diffusion/transient/continuous_2d_d03_p01/disk_polar_4_3_forman_trapezoidal_0p05_1000_potential}
  and
  \Cref{figure:idec/diffusion/transient/continuous_2d_d03_p01/disk_polar_4_3_forman_trapezoidal_0p05_1000_flow_rate}.

  {\color{red} At the moment there are problems with the primal strong and mixed
  methods!!!}
\end{example}

\begin{figure}[!ht]
  \begin{subfigure}{.32\textwidth}
    \centering
    \includegraphics[scale=.2, page=1]
    {diffusion/transient/continuous_2d_d03_p01/primal_strong_cochain_circular_4_3_forman_trapezoidal_0p05_1000_potential}
    \caption{Primal strong potential, 1}
  \end{subfigure}
  \begin{subfigure}{.32\textwidth}
    \centering
    \includegraphics[scale=.2, page=101]
    {diffusion/transient/continuous_2d_d03_p01/primal_strong_cochain_circular_4_3_forman_trapezoidal_0p05_1000_potential}
    \caption{Primal strong potential, 101}
  \end{subfigure}
  \begin{subfigure}{.32\textwidth}
    \centering
    \includegraphics[scale=.2, page=201]
    {diffusion/transient/continuous_2d_d03_p01/primal_strong_cochain_circular_4_3_forman_trapezoidal_0p05_1000_potential}
    \caption{Primal strong potential, 201}
  \end{subfigure}

  \begin{subfigure}{.32\textwidth}
    \centering
    \includegraphics[scale=.2, page=1]
    {diffusion/transient/continuous_2d_d03_p01/primal_weak_cochain_circular_4_3_forman_trapezoidal_0p05_1000_potential}
    \caption{Primal weak potential, 1}
  \end{subfigure}
  \begin{subfigure}{.32\textwidth}
    \centering
    \includegraphics[scale=.2, page=101]
    {diffusion/transient/continuous_2d_d03_p01/primal_weak_cochain_circular_4_3_forman_trapezoidal_0p05_1000_potential}
    \caption{Primal weak potential, 101}
  \end{subfigure}
  \begin{subfigure}{.32\textwidth}
    \centering
    \includegraphics[scale=.2, page=201]
    {diffusion/transient/continuous_2d_d03_p01/primal_weak_cochain_circular_4_3_forman_trapezoidal_0p05_1000_potential}
    \caption{Primal weak potential, 201}
  \end{subfigure}

  \begin{subfigure}{.32\textwidth}
    \centering
    \includegraphics[scale=.2, page=1]
    {diffusion/transient/continuous_2d_d03_p01/mixed_weak_cochain_circular_4_3_forman_trapezoidal_0p05_1000_potential}
    \caption{Mixed weak potential, 1}
  \end{subfigure}
  \begin{subfigure}{.32\textwidth}
    \centering
    \includegraphics[scale=.2, page=101]
    {diffusion/transient/continuous_2d_d03_p01/mixed_weak_cochain_circular_4_3_forman_trapezoidal_0p05_1000_potential}
    \caption{Mixed weak potential, 101}
  \end{subfigure}
  \begin{subfigure}{.32\textwidth}
    \centering
    \includegraphics[scale=.2, page=201]
    {diffusion/transient/continuous_2d_d03_p01/mixed_weak_cochain_circular_4_3_forman_trapezoidal_0p05_1000_potential}
    \caption{Mixed weak potential, 201}
  \end{subfigure}

  \begin{subfigure}{.32\textwidth}
    \centering
    \includegraphics[scale=.2, page=1]
    {diffusion/transient/continuous_2d_d03_p01/primal_strong_cochain_circular_4_3_forman_trapezoidal_0p05_1000_flow}
    \caption{Primal strong flow, 1}
  \end{subfigure}
  \begin{subfigure}{.32\textwidth}
    \centering
    \includegraphics[scale=.2, page=101]
    {diffusion/transient/continuous_2d_d03_p01/primal_strong_cochain_circular_4_3_forman_trapezoidal_0p05_1000_flow}
    \caption{Primal strong flow, 101}
  \end{subfigure}
  \begin{subfigure}{.32\textwidth}
    \centering
    \includegraphics[scale=.2, page=201]
    {diffusion/transient/continuous_2d_d03_p01/primal_strong_cochain_circular_4_3_forman_trapezoidal_0p05_1000_flow}
    \caption{Primal strong flow, 201}
  \end{subfigure}

  \begin{subfigure}{.32\textwidth}
    \centering
    \includegraphics[scale=.2, page=1]
    {diffusion/transient/continuous_2d_d03_p01/primal_weak_cochain_circular_4_3_forman_trapezoidal_0p05_1000_flow}
    \caption{Primal weak flow, 1}
  \end{subfigure}
  \begin{subfigure}{.32\textwidth}
    \centering
    \includegraphics[scale=.2, page=101]
    {diffusion/transient/continuous_2d_d03_p01/primal_weak_cochain_circular_4_3_forman_trapezoidal_0p05_1000_flow}
    \caption{Primal weak flow, 101}
  \end{subfigure}
  \begin{subfigure}{.32\textwidth}
    \centering
    \includegraphics[scale=.2, page=201]
    {diffusion/transient/continuous_2d_d03_p01/primal_weak_cochain_circular_4_3_forman_trapezoidal_0p05_1000_flow}
    \caption{Primal weak flow, 201}
  \end{subfigure}
  
  \begin{subfigure}{.32\textwidth}
    \centering
    \includegraphics[scale=.2, page=1]
    {diffusion/transient/continuous_2d_d03_p01/mixed_weak_cochain_circular_4_3_forman_trapezoidal_0p05_1000_flow}
    \caption{Mixed weak flow, 1}
  \end{subfigure}
  \begin{subfigure}{.32\textwidth}
    \centering
    \includegraphics[scale=.2, page=101]
    {diffusion/transient/continuous_2d_d03_p01/mixed_weak_cochain_circular_4_3_forman_trapezoidal_0p05_1000_flow}
    \caption{Mixed weak flow, 101}
  \end{subfigure}
  \begin{subfigure}{.32\textwidth}
    \centering
    \includegraphics[scale=.2, page=201]
    {diffusion/transient/continuous_2d_d03_p01/mixed_weak_cochain_circular_4_3_forman_trapezoidal_0p05_1000_flow}
    \caption{Mixed weak flow, 201}
  \end{subfigure}
  \cprotect\caption{Potential and flow for the \verb|2d_d03_p01| problem}
  \label{figure:idec/diffusion/transient/continuous_2d_d03_p01/circular_4_3_forman_trapezoidal_0p05_1000}
\end{figure}

\subsection{Transient}
\begin{example}
  \label{idec/diffusion/continuous/transient/examples/2d_d00_p00-example}
  Consider the transient continuous heat transport problem
  (\Cref{idec/diffusion/continuous/transient/primal_strong-formulation},
   \Cref{idec/diffusion/continuous/transient/primal_weak-formulation},
   \Cref{idec/diffusion/continuous/transient/mixed_weak-formulation})
  with input data \verb|2d_d00_p00| in the nomenclature of the C codebase.

  Concretely,
    $X = [0, 1]^2$,
    $\pi_0 \equiv 1$,
    $\kappa_1 \equiv 1$,
    $u_0(x, y) =
      \begin{cases}
        100, & (x, y) = (0.5, 0.5) \\
        0, & \text{else}
      \end{cases}$,
    $f \equiv 0$,
    $G_D = \partial X$,
    $G_N = \emptyset$,
    $g_D(x, y) = 0$.

  This problem has the following exact solution in steady-state:
  \begin{subequations}
    \begin{alignat}{3}
      & u && \equiv && 0, \\
      & q && \equiv && 0.
    \end{alignat}
  \end{subequations}
  Consider a mesh $M$ for $X$ consisting of $2 \times 2$ squares (each axis is
  divided into $2$ segments) with Forman subdivision $K$ ($4 \times 4$ squares).
  Its potential and flow rate on $K$ consisting of the $3$ discussed cochain
  methods are shown on
  \Cref{figure:idec/diffusion/transient/continuous_2d_d00_p00/brick_2d_2_forman_trapezoidal_0p001_1000_potential}
  and
  \Cref{figure:idec/diffusion/transient/continuous_2d_d00_p00/brick_2d_2_forman_trapezoidal_0p001_1000_flow_rate}.
\end{example}

\begin{figure}[!ht]
  \begin{subfigure}{.32\textwidth}
    \centering
    \includegraphics[scale=.2, page=1]
    {diffusion/transient/continuous_2d_d00_p00/primal_strong_cochain_brick_2d_2_forman_trapezoidal_0p001_1000_potential}
    \caption{Primal strong potential, 1}
  \end{subfigure}
  \begin{subfigure}{.32\textwidth}
    \centering
    \includegraphics[scale=.2, page=151]
    {diffusion/transient/continuous_2d_d00_p00/primal_strong_cochain_brick_2d_2_forman_trapezoidal_0p001_1000_potential}
    \caption{Primal strong potential, 151}
  \end{subfigure}
  \begin{subfigure}{.32\textwidth}
    \centering
    \includegraphics[scale=.2, page=301]
    {diffusion/transient/continuous_2d_d00_p00/primal_strong_cochain_brick_2d_2_forman_trapezoidal_0p001_1000_potential}
    \caption{Primal strong potential, 301}
  \end{subfigure}

  \begin{subfigure}{.32\textwidth}
    \centering
    \includegraphics[scale=.2, page=1]
    {diffusion/transient/continuous_2d_d00_p00/primal_weak_cochain_brick_2d_2_forman_trapezoidal_0p001_1000_potential}
    \caption{Primal weak potential, 1}
  \end{subfigure}
  \begin{subfigure}{.32\textwidth}
    \centering
    \includegraphics[scale=.2, page=151]
    {diffusion/transient/continuous_2d_d00_p00/primal_weak_cochain_brick_2d_2_forman_trapezoidal_0p001_1000_potential}
    \caption{Primal weak potential, 151}
  \end{subfigure}
  \begin{subfigure}{.32\textwidth}
    \centering
    \includegraphics[scale=.2, page=301]
    {diffusion/transient/continuous_2d_d00_p00/primal_weak_cochain_brick_2d_2_forman_trapezoidal_0p001_1000_potential}
    \caption{Primal weak potential, 301}
  \end{subfigure}

  \begin{subfigure}{.32\textwidth}
    \centering
    \includegraphics[scale=.2, page=1]
    {diffusion/transient/continuous_2d_d00_p00/mixed_weak_cochain_brick_2d_2_forman_trapezoidal_0p001_1000_potential}
    \caption{Mixed weak potential, 1}
  \end{subfigure}
  \begin{subfigure}{.32\textwidth}
    \centering
    \includegraphics[scale=.2, page=151]
    {diffusion/transient/continuous_2d_d00_p00/mixed_weak_cochain_brick_2d_2_forman_trapezoidal_0p001_1000_potential}
    \caption{Mixed weak potential, 151}
  \end{subfigure}
  \begin{subfigure}{.32\textwidth}
    \centering
    \includegraphics[scale=.2, page=301]
    {diffusion/transient/continuous_2d_d00_p00/mixed_weak_cochain_brick_2d_2_forman_trapezoidal_0p001_1000_potential}
    \caption{Mixed weak potential, 301}
  \end{subfigure}

  \begin{subfigure}{.32\textwidth}
    \centering
    \includegraphics[scale=.2, page=1]
    {diffusion/transient/continuous_2d_d00_p00/primal_strong_cochain_brick_2d_2_forman_trapezoidal_0p001_1000_flow}
    \caption{Primal strong flow, 1}
  \end{subfigure}
  \begin{subfigure}{.32\textwidth}
    \centering
    \includegraphics[scale=.2, page=151]
    {diffusion/transient/continuous_2d_d00_p00/primal_strong_cochain_brick_2d_2_forman_trapezoidal_0p001_1000_flow}
    \caption{Primal strong flow, 151}
  \end{subfigure}
  \begin{subfigure}{.32\textwidth}
    \centering
    \includegraphics[scale=.2, page=301]
    {diffusion/transient/continuous_2d_d00_p00/primal_strong_cochain_brick_2d_2_forman_trapezoidal_0p001_1000_flow}
    \caption{Primal strong flow, 301}
  \end{subfigure}

  \begin{subfigure}{.32\textwidth}
    \centering
    \includegraphics[scale=.2, page=1]
    {diffusion/transient/continuous_2d_d00_p00/primal_weak_cochain_brick_2d_2_forman_trapezoidal_0p001_1000_flow}
    \caption{Primal weak flow, 1}
  \end{subfigure}
  \begin{subfigure}{.32\textwidth}
    \centering
    \includegraphics[scale=.2, page=151]
    {diffusion/transient/continuous_2d_d00_p00/primal_weak_cochain_brick_2d_2_forman_trapezoidal_0p001_1000_flow}
    \caption{Primal weak flow, 151}
  \end{subfigure}
  \begin{subfigure}{.32\textwidth}
    \centering
    \includegraphics[scale=.2, page=301]
    {diffusion/transient/continuous_2d_d00_p00/primal_weak_cochain_brick_2d_2_forman_trapezoidal_0p001_1000_flow}
    \caption{Primal weak flow, 301}
  \end{subfigure}
  
  \begin{subfigure}{.32\textwidth}
    \centering
    \includegraphics[scale=.2, page=1]
    {diffusion/transient/continuous_2d_d00_p00/mixed_weak_cochain_brick_2d_2_forman_trapezoidal_0p001_1000_flow}
    \caption{Mixed weak flow, 1}
  \end{subfigure}
  \begin{subfigure}{.32\textwidth}
    \centering
    \includegraphics[scale=.2, page=151]
    {diffusion/transient/continuous_2d_d00_p00/mixed_weak_cochain_brick_2d_2_forman_trapezoidal_0p001_1000_flow}
    \caption{Mixed weak flow, 151}
  \end{subfigure}
  \begin{subfigure}{.32\textwidth}
    \centering
    \includegraphics[scale=.2, page=301]
    {diffusion/transient/continuous_2d_d00_p00/mixed_weak_cochain_brick_2d_2_forman_trapezoidal_0p001_1000_flow}
    \caption{Mixed weak flow, 301}
  \end{subfigure}
  \cprotect\caption{Potential and flow for the \verb|2d_d00_p00| problem}
  \label{figure:idec/diffusion/transient/continuous_2d_d00_p00/cochain_brick_2d_2_forman_trapezoidal_0p001_1000}
\end{figure}

\begin{example}
  Consider the steady-state continuous heat transport problem
  (\Cref{idec/heat_transport/continuous/primal_strong_steady_state-formulation},
   \Cref{idec/heat_transport/continuous/primal_weak_steady_state-formulation},
   \Cref{idec/heat_transport/continuous/mixed_weak_steady_state-formulation})
  with input data \verb|2d_d00_p01| in the nomenclature of the C codebase.

  Concretely,
    $X = [0, 1]^2$,
    $\kappa_1 \equiv 1$,
    $f \equiv 0$,
    $G_D = \{0, 1\} \times [0, 1]$,
    $G_N = [0, 1] \times \{0, 1\}$,
    $g_D(x, y) = \begin{cases} -100, & x = 0 \\ 100, & x = 1 \end{cases}$,
    $g_N \equiv 0$.

  This problem has the following exact solution:
  \begin{subequations}
    \begin{alignat}{3}
      & u(x, y) && = && 100 (2 x - 1), \\
      & q(x, y) && = && 200\, d y.
    \end{alignat}
  \end{subequations}
  Consider a mesh $M$ for $X$ consisting of $2 \times 2$ squares (each axis is
  divided into $2$ segments) with Forman subdivision $K$ ($4 \times 4$ squares).
  Its potential and flow on $K$ consisting of the exact solution and the $3$
  discussed cochain methods are shown on
  \Cref{figure:idec/diffusion/steady_state/continuous_2d_d00_p01/brick_2d_2_forman}.
\end{example}

\begin{figure}[!ht]
  \begin{subfigure}{.24\textwidth}
    \centering
    \includegraphics[scale=.19, page=1]
    {diffusion/transient/continuous_2d_d00_p01/primal_strong_cochain_brick_2d_5_forman_trapezoidal_0p001_2500_potential}
    \caption{Primal strong, 1}
  \end{subfigure}
  \begin{subfigure}{.24\textwidth}
    \centering
    \includegraphics[scale=.19, page=501]
    {diffusion/transient/continuous_2d_d00_p01/primal_strong_cochain_brick_2d_5_forman_trapezoidal_0p001_2500_potential}
    \caption{Primal strong, 501}
  \end{subfigure}
  \begin{subfigure}{.24\textwidth}
    \centering
    \includegraphics[scale=.19, page=1001]
    {diffusion/transient/continuous_2d_d00_p01/primal_strong_cochain_brick_2d_5_forman_trapezoidal_0p001_2500_potential}
    \caption{Primal strong, 1001}
  \end{subfigure}
  \begin{subfigure}{.24\textwidth}
    \centering
    \includegraphics[scale=.19, page=1501]
    {diffusion/transient/continuous_2d_d00_p01/primal_strong_cochain_brick_2d_5_forman_trapezoidal_0p001_2500_potential}
    \caption{Primal strong, 1501}
  \end{subfigure}

  \begin{subfigure}{.24\textwidth}
    \centering
    \includegraphics[scale=.19, page=1]
    {diffusion/transient/continuous_2d_d00_p01/primal_weak_cochain_brick_2d_5_forman_trapezoidal_0p001_2500_potential}
    \caption{Primal weak, 1}
  \end{subfigure}
  \begin{subfigure}{.24\textwidth}
    \centering
    \includegraphics[scale=.19, page=501]
    {diffusion/transient/continuous_2d_d00_p01/primal_weak_cochain_brick_2d_5_forman_trapezoidal_0p001_2500_potential}
    \caption{Primal weak, 501}
  \end{subfigure}
  \begin{subfigure}{.24\textwidth}
    \centering
    \includegraphics[scale=.19, page=1001]
    {diffusion/transient/continuous_2d_d00_p01/primal_weak_cochain_brick_2d_5_forman_trapezoidal_0p001_2500_potential}
    \caption{Primal weak, 1001}
  \end{subfigure}
  \begin{subfigure}{.24\textwidth}
    \centering
    \includegraphics[scale=.19, page=1501]
    {diffusion/transient/continuous_2d_d00_p01/primal_weak_cochain_brick_2d_5_forman_trapezoidal_0p001_2500_potential}
    \caption{Primal weak, 1501}
  \end{subfigure}

  \begin{subfigure}{.24\textwidth}
    \centering
    \includegraphics[scale=.19, page=1]
    {diffusion/transient/continuous_2d_d00_p01/mixed_weak_cochain_brick_2d_5_forman_trapezoidal_0p001_2500_potential}
    \caption{Mixed weak, 1}
  \end{subfigure}
  \begin{subfigure}{.24\textwidth}
    \centering
    \includegraphics[scale=.19, page=501]
    {diffusion/transient/continuous_2d_d00_p01/mixed_weak_cochain_brick_2d_5_forman_trapezoidal_0p001_2500_potential}
    \caption{Mixed weak, 501}
  \end{subfigure}
  \begin{subfigure}{.24\textwidth}
    \centering
    \includegraphics[scale=.19, page=1001]
    {diffusion/transient/continuous_2d_d00_p01/mixed_weak_cochain_brick_2d_5_forman_trapezoidal_0p001_2500_potential}
    \caption{Mixed weak, 1001}
  \end{subfigure}
  \begin{subfigure}{.24\textwidth}
    \centering
    \includegraphics[scale=.19, page=1501]
    {diffusion/transient/continuous_2d_d00_p01/mixed_weak_cochain_brick_2d_5_forman_trapezoidal_0p001_2500_potential}
    \caption{Mixed weak, 1501}
  \end{subfigure}
  \cprotect
  \caption{%
    \verb|diffusion/transient/continuous_2d_d00_p01|
    (\Cref{cmc/diffusion/continuous/transient/examples/2d_d00_p01-example}):
    solutions for potential}
  \label{figure:cmc/diffusion/transient/continuous_2d_d00_p01/brick_2d_5_forman_trapezoidal_0p001_2500_potential}
\end{figure}
\begin{figure}[!ht]
  \begin{subfigure}{.24\textwidth}
    \centering
    \includegraphics[scale=.19, page=1]
    {diffusion/transient/continuous_2d_d00_p01/primal_strong_cochain_brick_2d_5_forman_trapezoidal_0p001_2500_flow_rate}
    \caption{Primal strong, 1}
  \end{subfigure}
  \begin{subfigure}{.24\textwidth}
    \centering
    \includegraphics[scale=.19, page=501]
    {diffusion/transient/continuous_2d_d00_p01/primal_strong_cochain_brick_2d_5_forman_trapezoidal_0p001_2500_flow_rate}
    \caption{Primal strong, 501}
  \end{subfigure}
  \begin{subfigure}{.24\textwidth}
    \centering
    \includegraphics[scale=.19, page=1001]
    {diffusion/transient/continuous_2d_d00_p01/primal_strong_cochain_brick_2d_5_forman_trapezoidal_0p001_2500_flow_rate}
    \caption{Primal strong, 1001}
  \end{subfigure}
  \begin{subfigure}{.24\textwidth}
    \centering
    \includegraphics[scale=.19, page=1501]
    {diffusion/transient/continuous_2d_d00_p01/primal_strong_cochain_brick_2d_5_forman_trapezoidal_0p001_2500_flow_rate}
    \caption{Primal strong, 1501}
  \end{subfigure}

  \begin{subfigure}{.24\textwidth}
    \centering
    \includegraphics[scale=.19, page=1]
    {diffusion/transient/continuous_2d_d00_p01/primal_weak_cochain_brick_2d_5_forman_trapezoidal_0p001_2500_flow_rate}
    \caption{Primal weak, 1}
  \end{subfigure}
  \begin{subfigure}{.24\textwidth}
    \centering
    \includegraphics[scale=.19, page=501]
    {diffusion/transient/continuous_2d_d00_p01/primal_weak_cochain_brick_2d_5_forman_trapezoidal_0p001_2500_flow_rate}
    \caption{Primal weak, 501}
  \end{subfigure}
  \begin{subfigure}{.24\textwidth}
    \centering
    \includegraphics[scale=.19, page=1001]
    {diffusion/transient/continuous_2d_d00_p01/primal_weak_cochain_brick_2d_5_forman_trapezoidal_0p001_2500_flow_rate}
    \caption{Primal weak, 1001}
  \end{subfigure}
  \begin{subfigure}{.24\textwidth}
    \centering
    \includegraphics[scale=.19, page=1501]
    {diffusion/transient/continuous_2d_d00_p01/primal_weak_cochain_brick_2d_5_forman_trapezoidal_0p001_2500_flow_rate}
    \caption{Primal weak, 1501}
  \end{subfigure}

  \begin{subfigure}{.24\textwidth}
    \centering
    \includegraphics[scale=.19, page=1]
    {diffusion/transient/continuous_2d_d00_p01/mixed_weak_cochain_brick_2d_5_forman_trapezoidal_0p001_2500_flow_rate}
    \caption{Mixed weak, 1}
  \end{subfigure}
  \begin{subfigure}{.24\textwidth}
    \centering
    \includegraphics[scale=.19, page=501]
    {diffusion/transient/continuous_2d_d00_p01/mixed_weak_cochain_brick_2d_5_forman_trapezoidal_0p001_2500_flow_rate}
    \caption{Mixed weak, 501}
  \end{subfigure}
  \begin{subfigure}{.24\textwidth}
    \centering
    \includegraphics[scale=.19, page=1001]
    {diffusion/transient/continuous_2d_d00_p01/mixed_weak_cochain_brick_2d_5_forman_trapezoidal_0p001_2500_flow_rate}
    \caption{Mixed weak, 1001}
  \end{subfigure}
  \begin{subfigure}{.24\textwidth}
    \centering
    \includegraphics[scale=.19, page=1501]
    {diffusion/transient/continuous_2d_d00_p01/mixed_weak_cochain_brick_2d_5_forman_trapezoidal_0p001_2500_flow_rate}
    \caption{Mixed weak, 1501}
  \end{subfigure}
  \cprotect
  \caption{%
    \verb|diffusion/transient/continuous_2d_d00_p01|
    (\Cref{cmc/diffusion/continuous/transient/examples/2d_d00_p01-example}):
    solutions for flow rate}
  \label{figure:cmc/diffusion/transient/continuous_2d_d00_p01/brick_2d_5_forman_trapezoidal_0p001_2500_flow_rate}
\end{figure}

\begin{example}
  Consider the transient continuous heat transport problem
  (\Cref{idec/diffusion/continuous/transient/primal_strong-formulation},
   \Cref{idec/diffusion/continuous/transient/primal_weak-formulation},
   \Cref{idec/diffusion/continuous/transient/mixed_weak-formulation})
  with input data \verb|2d_d00_p02| in the nomenclature of the C codebase.

  Concretely,
    $X = [0, 1]^2$,
    $\pi_0 \equiv 1$,
    $\kappa_1 \equiv 1$,
    $u_0(x, y) = x^2 + y^2$,
    $f \equiv - 4\, d x \wedge d y$,
    $G_D = \partial X$,
    $G_N = \emptyset$,
    $g_D(x, y) = x^2 + y^2$.

  This problem is in steady-state mode from the beginning
  with the following exact solution:
  \begin{subequations}
    \begin{alignat}{3}
      & u(x, y) && = && x^2 + y^2, \\
      & q(x, y) && = && 2 y\, d x - 2 x\, d y.
    \end{alignat}
  \end{subequations}
  Consider a mesh $M$ for $X$ consisting of $2 \times 2$ squares (each axis is
  divided into $2$ segments) with Forman subdivision $K$ ($4 \times 4$ squares).
  Its potential and flow on $K$ consisting of the exact solution and the $3$
  discussed cochain methods are shown on
  \Cref{figure:idec/diffusion/transient/continuous_2d_d00_p02/brick_2d_5_forman_trapezoidal_0p001_1000}.
\end{example}

\begin{figure}[!ht]
  \begin{subfigure}{.24\textwidth}
    \centering
    \includegraphics[scale=.19, page=1]
    {diffusion/transient/continuous_2d_d00_p02/primal_strong_cochain_brick_2d_5_forman_trapezoidal_0p001_1000_potential}
    \caption{Primal strong, 1}
  \end{subfigure}
  \begin{subfigure}{.24\textwidth}
    \centering
    \includegraphics[scale=.19, page=334]
    {diffusion/transient/continuous_2d_d00_p02/primal_strong_cochain_brick_2d_5_forman_trapezoidal_0p001_1000_potential}
    \caption{Primal strong, 334}
  \end{subfigure}
  \begin{subfigure}{.24\textwidth}
    \centering
    \includegraphics[scale=.19, page=667]
    {diffusion/transient/continuous_2d_d00_p02/primal_strong_cochain_brick_2d_5_forman_trapezoidal_0p001_1000_potential}
    \caption{Primal strong, 667}
  \end{subfigure}
  \begin{subfigure}{.24\textwidth}
    \centering
    \includegraphics[scale=.19, page=1001]
    {diffusion/transient/continuous_2d_d00_p02/primal_strong_cochain_brick_2d_5_forman_trapezoidal_0p001_1000_potential}
    \caption{Primal strong, 1001}
  \end{subfigure}

  \begin{subfigure}{.24\textwidth}
    \centering
    \includegraphics[scale=.19, page=1]
    {diffusion/transient/continuous_2d_d00_p02/primal_weak_cochain_brick_2d_5_forman_trapezoidal_0p001_1000_potential}
    \caption{Primal weak, 1}
  \end{subfigure}
  \begin{subfigure}{.24\textwidth}
    \centering
    \includegraphics[scale=.19, page=334]
    {diffusion/transient/continuous_2d_d00_p02/primal_weak_cochain_brick_2d_5_forman_trapezoidal_0p001_1000_potential}
    \caption{Primal weak, 334}
  \end{subfigure}
  \begin{subfigure}{.24\textwidth}
    \centering
    \includegraphics[scale=.19, page=667]
    {diffusion/transient/continuous_2d_d00_p02/primal_weak_cochain_brick_2d_5_forman_trapezoidal_0p001_1000_potential}
    \caption{Primal weak, 667}
  \end{subfigure}
  \begin{subfigure}{.24\textwidth}
    \centering
    \includegraphics[scale=.19, page=1001]
    {diffusion/transient/continuous_2d_d00_p02/primal_weak_cochain_brick_2d_5_forman_trapezoidal_0p001_1000_potential}
    \caption{Primal weak, 1001}
  \end{subfigure}

  \begin{subfigure}{.24\textwidth}
    \centering
    \includegraphics[scale=.19, page=1]
    {diffusion/transient/continuous_2d_d00_p02/mixed_weak_cochain_brick_2d_5_forman_trapezoidal_0p001_1000_potential}
    \caption{Mixed weak, 1}
  \end{subfigure}
  \begin{subfigure}{.24\textwidth}
    \centering
    \includegraphics[scale=.19, page=334]
    {diffusion/transient/continuous_2d_d00_p02/mixed_weak_cochain_brick_2d_5_forman_trapezoidal_0p001_1000_potential}
    \caption{Mixed weak, 334}
  \end{subfigure}
  \begin{subfigure}{.24\textwidth}
    \centering
    \includegraphics[scale=.19, page=667]
    {diffusion/transient/continuous_2d_d00_p02/mixed_weak_cochain_brick_2d_5_forman_trapezoidal_0p001_1000_potential}
    \caption{Mixed weak, 667}
  \end{subfigure}
  \begin{subfigure}{.24\textwidth}
    \centering
    \includegraphics[scale=.19, page=1001]
    {diffusion/transient/continuous_2d_d00_p02/mixed_weak_cochain_brick_2d_5_forman_trapezoidal_0p001_1000_potential}
    \caption{Mixed weak, 1001}
  \end{subfigure}
  \cprotect
  \caption{%
    \verb|diffusion/transient/continuous_2d_d00_p02|
    (\Cref{cmc/diffusion/continuous/transient/examples/2d_d00_p02-example}):
    solutions for potential}
  \label{figure:cmc/diffusion/transient/continuous_2d_d00_p02/brick_2d_5_forman_trapezoidal_0p001_1000_potential}
\end{figure}
\begin{figure}[!ht]
  \begin{subfigure}{.24\textwidth}
    \centering
    \includegraphics[scale=.19, page=1]
    {diffusion/transient/continuous_2d_d00_p02/primal_strong_cochain_brick_2d_5_forman_trapezoidal_0p001_1000_flow_rate}
    \caption{Primal strong, 1}
  \end{subfigure}
  \begin{subfigure}{.24\textwidth}
    \centering
    \includegraphics[scale=.19, page=334]
    {diffusion/transient/continuous_2d_d00_p02/primal_strong_cochain_brick_2d_5_forman_trapezoidal_0p001_1000_flow_rate}
    \caption{Primal strong, 334}
  \end{subfigure}
  \begin{subfigure}{.24\textwidth}
    \centering
    \includegraphics[scale=.19, page=667]
    {diffusion/transient/continuous_2d_d00_p02/primal_strong_cochain_brick_2d_5_forman_trapezoidal_0p001_1000_flow_rate}
    \caption{Primal strong, 667}
  \end{subfigure}
  \begin{subfigure}{.24\textwidth}
    \centering
    \includegraphics[scale=.19, page=1001]
    {diffusion/transient/continuous_2d_d00_p02/primal_strong_cochain_brick_2d_5_forman_trapezoidal_0p001_1000_flow_rate}
    \caption{Primal strong, 1001}
  \end{subfigure}

  \begin{subfigure}{.24\textwidth}
    \centering
    \includegraphics[scale=.19, page=1]
    {diffusion/transient/continuous_2d_d00_p02/primal_weak_cochain_brick_2d_5_forman_trapezoidal_0p001_1000_flow_rate}
    \caption{Primal weak, 1}
  \end{subfigure}
  \begin{subfigure}{.24\textwidth}
    \centering
    \includegraphics[scale=.19, page=334]
    {diffusion/transient/continuous_2d_d00_p02/primal_weak_cochain_brick_2d_5_forman_trapezoidal_0p001_1000_flow_rate}
    \caption{Primal weak, 334}
  \end{subfigure}
  \begin{subfigure}{.24\textwidth}
    \centering
    \includegraphics[scale=.19, page=667]
    {diffusion/transient/continuous_2d_d00_p02/primal_weak_cochain_brick_2d_5_forman_trapezoidal_0p001_1000_flow_rate}
    \caption{Primal weak, 667}
  \end{subfigure}
  \begin{subfigure}{.24\textwidth}
    \centering
    \includegraphics[scale=.19, page=1001]
    {diffusion/transient/continuous_2d_d00_p02/primal_weak_cochain_brick_2d_5_forman_trapezoidal_0p001_1000_flow_rate}
    \caption{Primal weak, 1001}
  \end{subfigure}

  \begin{subfigure}{.24\textwidth}
    \centering
    \includegraphics[scale=.19, page=1]
    {diffusion/transient/continuous_2d_d00_p02/mixed_weak_cochain_brick_2d_5_forman_trapezoidal_0p001_1000_flow_rate}
    \caption{Mixed weak, 1}
  \end{subfigure}
  \begin{subfigure}{.24\textwidth}
    \centering
    \includegraphics[scale=.19, page=334]
    {diffusion/transient/continuous_2d_d00_p02/mixed_weak_cochain_brick_2d_5_forman_trapezoidal_0p001_1000_flow_rate}
    \caption{Mixed weak, 334}
  \end{subfigure}
  \begin{subfigure}{.24\textwidth}
    \centering
    \includegraphics[scale=.19, page=667]
    {diffusion/transient/continuous_2d_d00_p02/mixed_weak_cochain_brick_2d_5_forman_trapezoidal_0p001_1000_flow_rate}
    \caption{Mixed weak, 667}
  \end{subfigure}
  \begin{subfigure}{.24\textwidth}
    \centering
    \includegraphics[scale=.19, page=1001]
    {diffusion/transient/continuous_2d_d00_p02/mixed_weak_cochain_brick_2d_5_forman_trapezoidal_0p001_1000_flow_rate}
    \caption{Mixed weak, 1001}
  \end{subfigure}
  \cprotect
  \caption{%
    \verb|diffusion/transient/continuous_2d_d00_p02|
    (\Cref{cmc/diffusion/continuous/transient/examples/2d_d00_p02-example}):
    solutions for flow rate}
  \label{figure:cmc/diffusion/transient/continuous_2d_d00_p02/brick_2d_5_forman_trapezoidal_0p001_1000_flow_rate}
\end{figure}

\begin{example}
  \label{idec/diffusion/continuous/steady_state/examples/2d_d00_p03-example}
  Consider the steady-state continuous heat transport problem
  (\Cref{idec/diffusion/continuous/steady_state/primal_strong-formulation},
   \Cref{idec/diffusion/continuous/steady_state/primal_weak-formulation},
   \Cref{idec/diffusion/continuous/steady_state/mixed_weak-formulation})
  with input data \verb|2d_d00_p03| in the nomenclature of the C codebase.

  Concretely,
    $X = [0, 1]^2$,
    $\kappa_1 \equiv 1$,
    $f = -2 \, d x \wedge d y$,
    $G_D = \{0, 1\} \times [0, 1]$,
    $G_N = [0, 1] \times \{0, 1\}$,
    $g_D \equiv 0$,
    $g_N \equiv 0$.

  This problem has the following exact solution:
  \begin{subequations}
    \begin{alignat}{3}
      & u(x, y) && = && x (x - 1), \\
      & q(x, y) && = && (2 x - 1)\, d y.
    \end{alignat}
  \end{subequations}
  Consider a mesh $M$ for $X$ consisting of $2 \times 2$ squares (each axis is
  divided into $2$ segments) with Forman subdivision $K$ ($4 \times 4$ squares).
  Its potential and flow on $K$ consisting of the exact solution and the $3$
  discussed cochain methods are shown on
  \Cref{figure:idec/diffusion/steady_state/continuous_2d_d00_p03/brick_2d_2_forman_potential}
  and
  \Cref{figure:idec/diffusion/steady_state/continuous_2d_d00_p03/brick_2d_2_forman_flow_rate}.
\end{example}

\begin{figure}[!ht]
  \begin{subfigure}{.24\textwidth}
    \centering
    \includegraphics[scale=.23]
    {diffusion/steady_state/continuous_2d_d00_p03/exact_brick_2d_2_forman_potential}
    \caption{Exact}
  \end{subfigure}
  \begin{subfigure}{.24\textwidth}
    \centering
    \includegraphics[scale=.23]
    {diffusion/steady_state/continuous_2d_d00_p03/primal_strong_cochain_brick_2d_2_forman_potential}
    \caption{Primal strong}
  \end{subfigure}
  \begin{subfigure}{.24\textwidth}
    \centering
    \includegraphics[scale=.23]
    {diffusion/steady_state/continuous_2d_d00_p03/primal_weak_cochain_brick_2d_2_forman_potential}
    \caption{Primal weak}
  \end{subfigure}
  \begin{subfigure}{.24\textwidth}
    \centering
    \includegraphics[scale=.23]
    {diffusion/steady_state/continuous_2d_d00_p03/mixed_weak_cochain_brick_2d_2_forman_potential}
    \caption{Mixed weak}
  \end{subfigure}
  \cprotect
  \caption{%
    \verb|diffusion/steady_state/continuous_2d_d00_p03|
    (\Cref{cmc/diffusion/continuous/steady_state/examples/2d_d00_p03-example}):
    solutions for potential}
  \label{figure:cmc/diffusion/steady_state/continuous_2d_d00_p03/brick_2d_2_forman_potential}
\end{figure}
\begin{figure}[!ht]
  \begin{subfigure}{.24\textwidth}
    \centering
    \includegraphics[scale=.23]
    {diffusion/steady_state/continuous_2d_d00_p03/exact_brick_2d_2_forman_flow_rate}
    \caption{Exact}
  \end{subfigure}
  \begin{subfigure}{.24\textwidth}
    \centering
    \includegraphics[scale=.23]
    {diffusion/steady_state/continuous_2d_d00_p03/primal_strong_cochain_brick_2d_2_forman_flow_rate}
    \caption{Primal strong}
  \end{subfigure}
  \begin{subfigure}{.24\textwidth}
    \centering
    \includegraphics[scale=.23]
    {diffusion/steady_state/continuous_2d_d00_p03/primal_weak_cochain_brick_2d_2_forman_flow_rate}
    \caption{Primal weak}
  \end{subfigure}
  \begin{subfigure}{.24\textwidth}
    \centering
    \includegraphics[scale=.23]
    {diffusion/steady_state/continuous_2d_d00_p03/mixed_weak_cochain_brick_2d_2_forman_flow_rate}
    \caption{Mixed weak}
  \end{subfigure}
  \cprotect
  \caption{%
    \verb|diffusion/steady_state/continuous_2d_d00_p03|
    (\Cref{cmc/diffusion/continuous/steady_state/examples/2d_d00_p03-example}):
    solutions for flow rate}
  \label{figure:cmc/diffusion/steady_state/continuous_2d_d00_p03/brick_2d_2_forman_flow_rate}
\end{figure}

\begin{example}
  \label{idec/diffusion/continuous/steady_state/examples/2d_d00_p04-example}
  Consider the steady-state continuous heat transport problem
  (\Cref{idec/diffusion/continuous/steady_state/primal_strong-formulation},
   \Cref{idec/diffusion/continuous/steady_state/primal_weak-formulation},
   \Cref{idec/diffusion/continuous/steady_state/mixed_weak-formulation})
  with input data \verb|2d_d00_p04| in the nomenclature of the C codebase.

  Concretely,
    $X = [0, 1]^2$,
    $\kappa_1 \equiv 1$,
    $f = -4 \, d x \wedge d y$,
    $G_D = \{0, 1\} \times [0, 1]$,
    $G_N = [0, 1] \times \{0, 1\}$,
    $g_D(x, y) = y (y - 1)$,
    $g_N(x, y) = (1 - 2 x) \, d x$.

  This problem has the following exact solution:
  \begin{subequations}
    \begin{alignat}{3}
      & u(x, y) && = && x (x - 1) + y (y - 1), \\
      & q(x, y) && = && (1 - 2 y)\, d x + (2 x - 1)\, d y.
    \end{alignat}
  \end{subequations}
  Consider a mesh $M$ for $X$ consisting of $5 \times 5$ squares (each axis is
  divided into $5$ segments) with Forman subdivision $K$
  ($10 \times 10$ squares).
  Its potential and flow on $K$ consisting of the exact solution and the $3$
  discussed cochain methods are shown on
  \Cref{figure:idec/diffusion/steady_state/continuous_2d_d00_p04/brick_2d_5_forman_potential}
  and
  \Cref{figure:idec/diffusion/steady_state/continuous_2d_d00_p04/brick_2d_5_forman_flow_rate}.
\end{example}

\begin{figure}[!ht]
  \begin{subfigure}{.32\textwidth}
    \centering
    \includegraphics[scale=.2, page=1]
    {diffusion/transient/continuous_2d_d00_p04/primal_strong_cochain_brick_2d_5_forman_trapezoidal_0p001_2500_potential}
    \caption{Primal strong potential, 1}
  \end{subfigure}
  \begin{subfigure}{.32\textwidth}
    \centering
    \includegraphics[scale=.2, page=1251]
    {diffusion/transient/continuous_2d_d00_p04/primal_strong_cochain_brick_2d_5_forman_trapezoidal_0p001_2500_potential}
    \caption{Primal strong potential, 1251}
  \end{subfigure}
  \begin{subfigure}{.32\textwidth}
    \centering
    \includegraphics[scale=.2, page=2501]
    {diffusion/transient/continuous_2d_d00_p04/primal_strong_cochain_brick_2d_5_forman_trapezoidal_0p001_2500_potential}
    \caption{Primal strong potential, 2501}
  \end{subfigure}

  \begin{subfigure}{.32\textwidth}
    \centering
    \includegraphics[scale=.2, page=1]
    {diffusion/transient/continuous_2d_d00_p04/mixed_weak_cochain_brick_2d_5_forman_trapezoidal_0p001_2500_potential}
    \caption{Mixed weak potential, 1}
  \end{subfigure}
  \begin{subfigure}{.32\textwidth}
    \centering
    \includegraphics[scale=.2, page=1251]
    {diffusion/transient/continuous_2d_d00_p04/mixed_weak_cochain_brick_2d_5_forman_trapezoidal_0p001_2500_potential}
    \caption{Mixed weak potential, 1251}
  \end{subfigure}
  \begin{subfigure}{.32\textwidth}
    \centering
    \includegraphics[scale=.2, page=2501]
    {diffusion/transient/continuous_2d_d00_p04/mixed_weak_cochain_brick_2d_5_forman_trapezoidal_0p001_2500_potential}
    \caption{Mixed weak potential, 2501}
  \end{subfigure}

  \begin{subfigure}{.32\textwidth}
    \centering
    \includegraphics[scale=.2, page=1]
    {diffusion/transient/continuous_2d_d00_p04/primal_strong_cochain_brick_2d_5_forman_trapezoidal_0p001_2500_flow}
    \caption{Primal strong flow, 1}
  \end{subfigure}
  \begin{subfigure}{.32\textwidth}
    \centering
    \includegraphics[scale=.2, page=1251]
    {diffusion/transient/continuous_2d_d00_p04/primal_strong_cochain_brick_2d_5_forman_trapezoidal_0p001_2500_flow}
    \caption{Primal strong flow, 1251}
  \end{subfigure}
  \begin{subfigure}{.32\textwidth}
    \centering
    \includegraphics[scale=.2, page=2501]
    {diffusion/transient/continuous_2d_d00_p04/primal_strong_cochain_brick_2d_5_forman_trapezoidal_0p001_2500_flow}
    \caption{Primal strong flow, 2501}
  \end{subfigure}
  
  \begin{subfigure}{.32\textwidth}
    \centering
    \includegraphics[scale=.2, page=1]
    {diffusion/transient/continuous_2d_d00_p04/mixed_weak_cochain_brick_2d_5_forman_trapezoidal_0p001_2500_flow}
    \caption{Mixed weak flow, 1}
  \end{subfigure}
  \begin{subfigure}{.32\textwidth}
    \centering
    \includegraphics[scale=.2, page=1251]
    {diffusion/transient/continuous_2d_d00_p04/mixed_weak_cochain_brick_2d_5_forman_trapezoidal_0p001_2500_flow}
    \caption{Mixed weak flow, 1251}
  \end{subfigure}
  \begin{subfigure}{.32\textwidth}
    \centering
    \includegraphics[scale=.2, page=2501]
    {diffusion/transient/continuous_2d_d00_p04/mixed_weak_cochain_brick_2d_5_forman_trapezoidal_0p001_2500_flow}
    \caption{Mixed weak flow, 2501}
  \end{subfigure}
  \cprotect\caption{Potential and flow for the \verb|2d_d00_p04| problem}
  \label{figure:idec/diffusion/transient/continuous_2d_d00_p04/brick_2d_5_forman_trapezoidal_0p001_2500}
\end{figure}

\begin{example}
  Consider the steady-state continuous heat transport problem
  (\Cref{idec/diffusion/continuous/steady_state/primal_strong-formulation},
   \Cref{idec/diffusion/continuous/steady_state/primal_weak-formulation},
   \Cref{idec/diffusion/continuous/steady_state/mixed_weak-formulation})
  with input data \verb|2d_d00_p05| in the nomenclature of the C codebase.

  Concretely,
    $X = [0, 1]^2$,
    $\kappa_1 \equiv 1$,
    $f(x, y) = \sin(\pi x) \sin(\pi y) \, d x \wedge d y$,
    $G_D = \partial X$,
    $G_N = \emptyset$,
    $g_D \equiv 0$.

  This problem has the following exact solution:
  \begin{subequations}
    \begin{alignat}{3}
      & u(x, y) && = && \frac{\sin(\pi x) \sin(\pi y)}{2 \pi^2}, \\
      & q(x, y) && = &&
        \frac{1}{2 \pi}
        ((- \sin(\pi x) \cos(\pi y) \, d x + \sin(\pi y) \cos(\pi x)\, d y)).
    \end{alignat}
  \end{subequations}
  Consider a mesh $M$ for $X$ consisting of $5 \times 5$ squares (each axis is
  divided into $5$ segments) with Forman subdivision $K$
  ($10 \times 10$ squares).
  Its potential and flow on $K$ consisting of the exact solution and the $3$
  discussed cochain methods are shown on
  \Cref{figure:idec/diffusion/steady_state/continuous_2d_d00_p05/brick_2d_5_forman}.
\end{example}

\begin{figure}[!ht]
  \begin{subfigure}{.22\textwidth}
    \centering
    \includegraphics[scale=.2]
    {diffusion/steady_state/continuous_2d_d00_p05/exact_brick_2d_5_forman_potential}
    \caption{Exact potential}
  \end{subfigure}
  \begin{subfigure}{.22\textwidth}
    \centering
    \includegraphics[scale=.2]
    {diffusion/steady_state/continuous_2d_d00_p05/primal_strong_cochain_brick_2d_5_forman_potential}
    \caption{Primal strong potential}
  \end{subfigure}
  \begin{subfigure}{.22\textwidth}
    \centering
    \includegraphics[scale=.2]
    {diffusion/steady_state/continuous_2d_d00_p05/primal_weak_cochain_brick_2d_5_forman_potential}
    \caption{Primal weak potential}
  \end{subfigure}
  \begin{subfigure}{.22\textwidth}
    \centering
    \includegraphics[scale=.2]
    {diffusion/steady_state/continuous_2d_d00_p05/mixed_weak_cochain_brick_2d_5_forman_potential}
    \caption{Mixed weak potential}
  \end{subfigure}

  \begin{subfigure}{.22\textwidth}
    \centering
    \includegraphics[scale=.2]
    {diffusion/steady_state/continuous_2d_d00_p05/exact_brick_2d_5_forman_flow}
    \caption{Exact flow}
  \end{subfigure}
  \begin{subfigure}{.22\textwidth}
    \centering
    \includegraphics[scale=.2]
    {diffusion/steady_state/continuous_2d_d00_p05/primal_strong_cochain_brick_2d_5_forman_flow}
    \caption{Primal strong flow}
  \end{subfigure}
  \begin{subfigure}{.22\textwidth}
    \centering
    \includegraphics[scale=.2]
    {diffusion/steady_state/continuous_2d_d00_p05/primal_weak_cochain_brick_2d_5_forman_flow}
    \caption{Primal weak flow}
  \end{subfigure}
  \begin{subfigure}{.22\textwidth}
    \centering
    \includegraphics[scale=.2]
    {diffusion/steady_state/continuous_2d_d00_p05/mixed_weak_cochain_brick_2d_5_forman_flow}
    \caption{Mixed weak flow}
  \end{subfigure}
  \cprotect\caption{Potential and flow for the \verb|2d_d00_p05| problem}
  \label{figure:idec/diffusion/steady_state/continuous_2d_d00_p05/brick_2d_5_forman}
\end{figure}

\begin{example}
  \label{cmc/diffusion/continuous/steady_state/examples/2d_d01_p00-example}
  Consider the steady-state continuous heat transport problem
  (\Cref{cmc/diffusion/continuous/steady_state/primal_strong-formulation},
   \Cref{cmc/diffusion/continuous/steady_state/primal_weak-formulation},
   \Cref{cmc/diffusion/continuous/steady_state/mixed_weak-formulation})
  with input data \verb|2d_d01_p00| in the nomenclature of the C codebase.

  Concretely,
    $X = {\rm polygon}((-5, 0), (0, -5), (5, 0), (0, 5)$,
    $\kappa_1 \equiv 6$,
    $f \equiv 0$,
    $G_D = {\rm line}((-5, 0), (0, -5)) \cup {\rm line}((5, 0), (0, 5))$,
    $G_N = {\rm line}((0, -5), (5, 0)) \cup {\rm line}((0, 5), (-5, 0))$,
    $g_D(x, y) =
      \begin{cases}
        100, & (x, y) \in {\rm line}((-5, 0), (0, -5)) \\
        0, & (x, y) \in {\rm line}((5, 0), (0, 5))
      \end{cases}$,
    $g_N \equiv 0$.

  This problem has the following exact solution:
  \begin{subequations}
    \begin{alignat}{3}
      & u(x, y) && = && 50 (1 - (x + y) / 5), \\
      & q(x, y) && = && 60 (d x - d y).
    \end{alignat}
  \end{subequations}
  Consider a mesh $M$ for $X$ consisting of $4 \times 4$ squares (each axis is
  divided into $4$ segments) with Forman subdivision $K$ ($8 \times 8$ squares).
  Its potential and flow rate on $K$ consisting of the exact solution and the
  $3$ discussed cochain methods are shown on
  \Cref{figure:cmc/diffusion/steady_state/continuous_2d_d01_p00/square_8_potential}
  and
  \Cref{figure:cmc/diffusion/steady_state/continuous_2d_d01_p00/square_8_flow_rate}.
\end{example}

\begin{figure}[!ht]
  \begin{subfigure}{.22\textwidth}
    \centering
    \includegraphics[scale=.2]
    {diffusion/steady_state/continuous_2d_d01_p00/exact_square_8_potential}
    \caption{Exact potential}
  \end{subfigure}
  \begin{subfigure}{.22\textwidth}
    \centering
    \includegraphics[scale=.2]
    {diffusion/steady_state/continuous_2d_d01_p00/primal_strong_cochain_square_8_potential}
    \caption{Primal strong potential}
  \end{subfigure}
  \begin{subfigure}{.22\textwidth}
    \centering
    \includegraphics[scale=.2]
    {diffusion/steady_state/continuous_2d_d01_p00/primal_weak_cochain_square_8_potential}
    \caption{Primal weak potential}
  \end{subfigure}
  \begin{subfigure}{.22\textwidth}
    \centering
    \includegraphics[scale=.2]
    {diffusion/steady_state/continuous_2d_d01_p00/mixed_weak_cochain_square_8_potential}
    \caption{Mixed weak potential}
  \end{subfigure}

  \begin{subfigure}{.22\textwidth}
    \centering
    \includegraphics[scale=.2]
    {diffusion/steady_state/continuous_2d_d01_p00/exact_square_8_flow}
    \caption{Exact flow}
  \end{subfigure}
  \begin{subfigure}{.22\textwidth}
    \centering
    \includegraphics[scale=.2]
    {diffusion/steady_state/continuous_2d_d01_p00/primal_strong_cochain_square_8_flow}
    \caption{Primal strong flow}
  \end{subfigure}
  \begin{subfigure}{.22\textwidth}
    \centering
    \includegraphics[scale=.2]
    {diffusion/steady_state/continuous_2d_d01_p00/primal_weak_cochain_square_8_flow}
    \caption{Primal weak flow}
  \end{subfigure}
  \begin{subfigure}{.22\textwidth}
    \centering
    \includegraphics[scale=.2]
    {diffusion/steady_state/continuous_2d_d01_p00/mixed_weak_cochain_square_8_flow}
    \caption{Mixed weak flow}
  \end{subfigure}
  \cprotect\caption{Potential and flow for the \verb|2d_d01_p00| problem}
  \label{figure:idec/diffusion/steady_state/continuous_2d_d01_p00/square_8}
\end{figure}

\begin{example}
  Consider the steady-state continuous heat transport problem
  (\Cref{idec/diffusion/continuous/steady_state/primal_strong-formulation},
   \Cref{idec/diffusion/continuous/steady_state/primal_weak-formulation},
   \Cref{idec/diffusion/continuous/steady_state/mixed_weak-formulation})
  with input data \verb|2d_d02_p00| in the nomenclature of the C codebase.

  Concretely,
    $X = [0, 20] \times [0, 15]$,
    $\kappa_1 \equiv 6$,
    $f \equiv 0$,
    $G_D = \{0, 20\} \times [0, 15]$,
    $G_N = [0, 20] \times \{0, 15\}$,
    $g_D(x, y) = \begin{cases} 100, & x = 0 \\ 0, & x = 20 \end{cases}$,
    $g_N \equiv 0$.

  This problem has the following exact solution:
  \begin{subequations}
    \begin{alignat}{3}
      & u(x, y) && = && 5 (20 - x), \\
      & q(x, y) && = && -30 \, d y.
    \end{alignat}
  \end{subequations}
  For this problem I use a mesh $M$ generated by
  \href{https://neper.info/}{Neper} with Forman subdivision $K$.
  Its potential and flow on $K$ consisting of the exact solution and the $3$
  discussed cochain methods are shown on
  \Cref{figure:idec/diffusion/steady_state/continuous_2d_d02_p00/2d_10_grains_forman}.
\end{example}

\begin{figure}[!ht]
  \begin{subfigure}{.32\textwidth}
    \centering
    \includegraphics[scale=.2, page=1]
    {diffusion/transient/continuous_2d_d02_p00/primal_strong_cochain_2d_10_grains_forman_trapezoidal_0p05_1000_potential}
    \caption{Primal strong potential, 1}
  \end{subfigure}
  \begin{subfigure}{.32\textwidth}
    \centering
    \includegraphics[scale=.2, page=501]
    {diffusion/transient/continuous_2d_d02_p00/primal_strong_cochain_2d_10_grains_forman_trapezoidal_0p05_1000_potential}
    \caption{Primal strong potential, 501}
  \end{subfigure}
  \begin{subfigure}{.32\textwidth}
    \centering
    \includegraphics[scale=.2, page=1001]
    {diffusion/transient/continuous_2d_d02_p00/primal_strong_cochain_2d_10_grains_forman_trapezoidal_0p05_1000_potential}
    \caption{Primal strong potential, 1001}
  \end{subfigure}

  \begin{subfigure}{.32\textwidth}
    \centering
    \includegraphics[scale=.2, page=1]
    {diffusion/transient/continuous_2d_d02_p00/primal_weak_cochain_2d_10_grains_forman_trapezoidal_0p05_1000_potential}
    \caption{Primal weak potential, 1}
  \end{subfigure}
  \begin{subfigure}{.32\textwidth}
    \centering
    \includegraphics[scale=.2, page=501]
    {diffusion/transient/continuous_2d_d02_p00/primal_weak_cochain_2d_10_grains_forman_trapezoidal_0p05_1000_potential}
    \caption{Primal weak potential, 501}
  \end{subfigure}
  \begin{subfigure}{.32\textwidth}
    \centering
    \includegraphics[scale=.2, page=1001]
    {diffusion/transient/continuous_2d_d02_p00/primal_weak_cochain_2d_10_grains_forman_trapezoidal_0p05_1000_potential}
    \caption{Primal weak potential, 1001}
  \end{subfigure}

  \begin{subfigure}{.32\textwidth}
    \centering
    \includegraphics[scale=.2, page=1]
    {diffusion/transient/continuous_2d_d02_p00/mixed_weak_cochain_2d_10_grains_forman_trapezoidal_0p05_1000_potential}
    \caption{Mixed weak potential, 1}
  \end{subfigure}
  \begin{subfigure}{.32\textwidth}
    \centering
    \includegraphics[scale=.2, page=501]
    {diffusion/transient/continuous_2d_d02_p00/mixed_weak_cochain_2d_10_grains_forman_trapezoidal_0p05_1000_potential}
    \caption{Mixed weak potential, 501}
  \end{subfigure}
  \begin{subfigure}{.32\textwidth}
    \centering
    \includegraphics[scale=.2, page=1001]
    {diffusion/transient/continuous_2d_d02_p00/mixed_weak_cochain_2d_10_grains_forman_trapezoidal_0p05_1000_potential}
    \caption{Mixed weak potential, 1001}
  \end{subfigure}

  \begin{subfigure}{.32\textwidth}
    \centering
    \includegraphics[scale=.2, page=1]
    {diffusion/transient/continuous_2d_d02_p00/primal_strong_cochain_2d_10_grains_forman_trapezoidal_0p05_1000_flow}
    \caption{Primal strong flow, 1}
  \end{subfigure}
  \begin{subfigure}{.32\textwidth}
    \centering
    \includegraphics[scale=.2, page=501]
    {diffusion/transient/continuous_2d_d02_p00/primal_strong_cochain_2d_10_grains_forman_trapezoidal_0p05_1000_flow}
    \caption{Primal strong flow, 501}
  \end{subfigure}
  \begin{subfigure}{.32\textwidth}
    \centering
    \includegraphics[scale=.2, page=1001]
    {diffusion/transient/continuous_2d_d02_p00/primal_strong_cochain_2d_10_grains_forman_trapezoidal_0p05_1000_flow}
    \caption{Primal strong flow, 1001}
  \end{subfigure}

  \begin{subfigure}{.32\textwidth}
    \centering
    \includegraphics[scale=.2, page=1]
    {diffusion/transient/continuous_2d_d02_p00/primal_weak_cochain_2d_10_grains_forman_trapezoidal_0p05_1000_flow}
    \caption{Primal weak flow, 1}
  \end{subfigure}
  \begin{subfigure}{.32\textwidth}
    \centering
    \includegraphics[scale=.2, page=501]
    {diffusion/transient/continuous_2d_d02_p00/primal_weak_cochain_2d_10_grains_forman_trapezoidal_0p05_1000_flow}
    \caption{Primal weak flow, 501}
  \end{subfigure}
  \begin{subfigure}{.32\textwidth}
    \centering
    \includegraphics[scale=.2, page=1001]
    {diffusion/transient/continuous_2d_d02_p00/primal_weak_cochain_2d_10_grains_forman_trapezoidal_0p05_1000_flow}
    \caption{Primal weak flow, 1001}
  \end{subfigure}
  
  \begin{subfigure}{.32\textwidth}
    \centering
    \includegraphics[scale=.2, page=1]
    {diffusion/transient/continuous_2d_d02_p00/mixed_weak_cochain_2d_10_grains_forman_trapezoidal_0p05_1000_flow}
    \caption{Mixed weak flow, 1}
  \end{subfigure}
  \begin{subfigure}{.32\textwidth}
    \centering
    \includegraphics[scale=.2, page=501]
    {diffusion/transient/continuous_2d_d02_p00/mixed_weak_cochain_2d_10_grains_forman_trapezoidal_0p05_1000_flow}
    \caption{Mixed weak flow, 501}
  \end{subfigure}
  \begin{subfigure}{.32\textwidth}
    \centering
    \includegraphics[scale=.2, page=1001]
    {diffusion/transient/continuous_2d_d02_p00/mixed_weak_cochain_2d_10_grains_forman_trapezoidal_0p05_1000_flow}
    \caption{Mixed weak flow, 1001}
  \end{subfigure}
  \cprotect\caption{Potential and flow for the \verb|2d_d02_p00| problem}
  \label{figure:idec/diffusion/transient/continuous_2d_d02_p00/2d_10_grains_forman_trapezoidal_0p05_1000}
\end{figure}

\begin{example}
  \label{idec/diffusion/continuous/steady_state/examples/2d_d02_p01-example}
  Consider the steady-state continuous heat transport problem
  (\Cref{idec/diffusion/continuous/steady_state/primal_strong-formulation},
   \Cref{idec/diffusion/continuous/steady_state/primal_weak-formulation},
   \Cref{idec/diffusion/continuous/steady_state/mixed_weak-formulation})
  with input data \verb|2d_d02_p01| in the nomenclature of the C codebase.

  Concretely,
    $X = [0, 20] \times [0, 15]$,
    $\kappa_1 \equiv 6$,
    $f \equiv 0$,
    $G_D = \{0, 20\} \times [0, 15]$,
    $G_N = [0, 20] \times \{0, 15\}$,
    $g_D(x, y) = \begin{cases} 0, & x = 0 \\ 100, & x = 20 \end{cases}$,
    $g_N \equiv 0$.

  This problem has the following exact solution:
  \begin{subequations}
    \begin{alignat}{3}
      & u(x, y) && = && 5 x, \\
      & q(x, y) && = && 30 \, d y.
    \end{alignat}
  \end{subequations}
  For this problem I use a mesh $M$ generated by
  \href{https://neper.info/}{Neper} with Forman subdivision $K$.
  Its potential and flow on $K$ consisting of the exact solution and the $3$
  discussed cochain methods are shown on
  \Cref{figure:idec/diffusion/steady_state/continuous_2d_d02_p01/2d_10_grains_forman_potential}
  and
  \Cref{figure:idec/diffusion/steady_state/continuous_2d_d02_p01/2d_10_grains_forman_flow_rate}.
\end{example}

\begin{figure}[!ht]
  \begin{subfigure}{.24\textwidth}
    \centering
    \includegraphics[scale=.23]
    {diffusion/steady_state/continuous_2d_d02_p01/exact_2d_10_grains_forman_potential}
    \caption{Exact}
  \end{subfigure}
  \begin{subfigure}{.24\textwidth}
    \centering
    \includegraphics[scale=.23]
    {diffusion/steady_state/continuous_2d_d02_p01/primal_strong_cochain_2d_10_grains_forman_potential}
    \caption{Primal strong}
  \end{subfigure}
  \begin{subfigure}{.24\textwidth}
    \centering
    \includegraphics[scale=.23]
    {diffusion/steady_state/continuous_2d_d02_p01/primal_weak_cochain_2d_10_grains_forman_potential}
    \caption{Primal weak}
  \end{subfigure}
  \begin{subfigure}{.24\textwidth}
    \centering
    \includegraphics[scale=.23]
    {diffusion/steady_state/continuous_2d_d02_p01/mixed_weak_cochain_2d_10_grains_forman_potential}
    \caption{Mixed weak}
  \end{subfigure}
  \cprotect
  \caption{%
    \verb|diffusion/steady_state/continuous_2d_d02_p01|
    (\Cref{idec/diffusion/continuous/steady_state/examples/2d_d02_p01-example}):
    solutions for potential}
  \label{figure:idec/diffusion/steady_state/continuous_2d_d02_p01/2d_10_grains_forman_potential}
\end{figure}
\begin{figure}[!ht]
  \begin{subfigure}{.24\textwidth}
    \centering
    \includegraphics[scale=.23]
    {diffusion/steady_state/continuous_2d_d02_p01/exact_2d_10_grains_forman_flow_rate}
    \caption{Exact}
  \end{subfigure}
  \begin{subfigure}{.24\textwidth}
    \centering
    \includegraphics[scale=.23]
    {diffusion/steady_state/continuous_2d_d02_p01/primal_strong_cochain_2d_10_grains_forman_flow_rate}
    \caption{Primal strong}
  \end{subfigure}
  \begin{subfigure}{.24\textwidth}
    \centering
    \includegraphics[scale=.23]
    {diffusion/steady_state/continuous_2d_d02_p01/primal_weak_cochain_2d_10_grains_forman_flow_rate}
    \caption{Primal weak}
  \end{subfigure}
  \begin{subfigure}{.24\textwidth}
    \centering
    \includegraphics[scale=.23]
    {diffusion/steady_state/continuous_2d_d02_p01/mixed_weak_cochain_2d_10_grains_forman_flow_rate}
    \caption{Mixed weak}
  \end{subfigure}
  \cprotect
  \caption{%
    \verb|diffusion/steady_state/continuous_2d_d02_p01|
    (\Cref{idec/diffusion/continuous/steady_state/examples/2d_d02_p01-example}):
    solutions for flow rate}
  \label{figure:idec/diffusion/steady_state/continuous_2d_d02_p01/2d_10_grains_forman_flow_rate}
\end{figure}

\begin{example}
  \label{idec/diffusion/continuous/steady_state/examples/2d_d03_p00-example}
  Consider the steady-state continuous heat transport problem
  (\Cref{idec/diffusion/continuous/steady_state/primal_strong-formulation},
   \Cref{idec/diffusion/continuous/steady_state/primal_weak-formulation},
   \Cref{idec/diffusion/continuous/steady_state/mixed_weak-formulation})
  with input data \verb|2d_d03_p00| in the nomenclature of the C codebase.

  Concretely,
    $X = \set{(x, y) \in \R^2}{x^2 + y^2 \leq 1}$,
    $\kappa_1 \equiv 1$,
    $f = - 4 \, d x \wedge d y$,
    $G_D = \partial X$,
    $G_N = \emptyset$,
    $g_D \equiv 1$.

  This problem has the following exact solution:
  \begin{subequations}
    \begin{alignat}{3}
      & u(x, y) && = && x^2 + y^2, \\
      & q(x, y) && = && -2 y \, d x + 2 x \, d y.
    \end{alignat}
  \end{subequations}
  Consider a mesh $M$ for $X$ consisting of $n_a$ rays and $n_d$ disks
  with Forman subdivision $K$.
  Its potential and flow on $K$ consisting of the exact solution and the $2$
  of the discussed cochain methods (no primal strong) are shown on
  \Cref{figure:idec/diffusion/steady_state/continuous_2d_d03_p00/circular_4_3_forman_potential},
  \Cref{figure:idec/diffusion/steady_state/continuous_2d_d03_p00/circular_4_3_forman_flow_rate}
  ($(n_a, n_d) = (4, 3)$)
  and
  \Cref{figure:idec/diffusion/steady_state/continuous_2d_d03_p00/circular_18_10_forman_potential},
  \Cref{figure:idec/diffusion/steady_state/continuous_2d_d03_p00/circular_18_10_forman_flow_rate}
  ($(n_a, n_d) = (18, 10)$).
\end{example}

\begin{figure}[!ht]
  \begin{subfigure}{.32\textwidth}
    \centering
    \includegraphics[scale=.32]
    {diffusion/steady_state/continuous_2d_d03_p00/exact_circular_4_3_forman_potential}
    \caption{Exact}
  \end{subfigure}
  \begin{subfigure}{.32\textwidth}
    \centering
    \includegraphics[scale=.32]
    {diffusion/steady_state/continuous_2d_d03_p00/primal_weak_cochain_circular_4_3_forman_potential}
    \caption{Primal weak}
  \end{subfigure}
  \begin{subfigure}{.32\textwidth}
    \centering
    \includegraphics[scale=.32]
    {diffusion/steady_state/continuous_2d_d03_p00/mixed_weak_cochain_circular_4_3_forman_potential}
    \caption{Mixed weak}
  \end{subfigure}
  \cprotect
  \caption{%
    \verb|diffusion/steady_state/continuous_2d_d03_p00|
    (\Cref{idec/diffusion/continuous/steady_state/examples/2d_d03_p00-example}):
    solutions for potential on mesh \verb|circular_4_3_forman|}
  \label{figure:idec/diffusion/steady_state/continuous_2d_d03_p00/circular_4_3_forman_potential}
\end{figure}
\begin{figure}[!ht]
  \begin{subfigure}{.32\textwidth}
    \centering
    \includegraphics[scale=.32]
    {diffusion/steady_state/continuous_2d_d03_p00/exact_circular_4_3_forman_flow}
    \caption{Exact}
  \end{subfigure}
  \begin{subfigure}{.32\textwidth}
    \centering
    \includegraphics[scale=.32]
    {diffusion/steady_state/continuous_2d_d03_p00/primal_weak_cochain_circular_4_3_forman_flow}
    \caption{Primal weak}
  \end{subfigure}
  \begin{subfigure}{.32\textwidth}
    \centering
    \includegraphics[scale=.32]
    {diffusion/steady_state/continuous_2d_d03_p00/mixed_weak_cochain_circular_4_3_forman_flow}
    \caption{Mixed weak}
  \end{subfigure}
  \cprotect
  \caption{%
    \verb|diffusion/steady_state/continuous_2d_d03_p00|
    (\Cref{idec/diffusion/continuous/steady_state/examples/2d_d03_p00-example}):
    solutions for flow rate on mesh \verb|circular_4_3_forman|}
  \label{figure:idec/diffusion/steady_state/continuous_2d_d03_p00/circular_4_3_forman_flow_rate}
\end{figure}
\begin{figure}[!ht]
  \begin{subfigure}{.32\textwidth}
    \centering
    \includegraphics[scale=.32]
    {diffusion/steady_state/continuous_2d_d03_p00/exact_circular_18_10_forman_potential}
    \caption{Exact}
  \end{subfigure}
  \begin{subfigure}{.32\textwidth}
    \centering
    \includegraphics[scale=.32]
    {diffusion/steady_state/continuous_2d_d03_p00/primal_weak_cochain_circular_18_10_forman_potential}
    \caption{Primal weak}
  \end{subfigure}
  \begin{subfigure}{.32\textwidth}
    \centering
    \includegraphics[scale=.32]
    {diffusion/steady_state/continuous_2d_d03_p00/mixed_weak_cochain_circular_18_10_forman_potential}
    \caption{Mixed weak}
  \end{subfigure}
  \cprotect
  \caption{%
    \verb|diffusion/steady_state/continuous_2d_d03_p00|
    (\Cref{idec/diffusion/continuous/steady_state/examples/2d_d03_p00-example}):
    solutions for potential on mesh \verb|circular_18_10_forman|}
  \label{figure:idec/diffusion/steady_state/continuous_2d_d03_p00/circular_18_10_forman_potential}
\end{figure}
\begin{figure}[!ht]
  \begin{subfigure}{.32\textwidth}
    \centering
    \includegraphics[scale=.32]
    {diffusion/steady_state/continuous_2d_d03_p00/exact_circular_18_10_forman_flow}
    \caption{Exact}
  \end{subfigure}
  \begin{subfigure}{.32\textwidth}
    \centering
    \includegraphics[scale=.32]
    {diffusion/steady_state/continuous_2d_d03_p00/primal_weak_cochain_circular_18_10_forman_flow}
    \caption{Primal weak}
  \end{subfigure}
  \begin{subfigure}{.32\textwidth}
    \centering
    \includegraphics[scale=.32]
    {diffusion/steady_state/continuous_2d_d03_p00/mixed_weak_cochain_circular_18_10_forman_flow}
    \caption{Mixed weak}
  \end{subfigure}
  \cprotect
  \caption{%
    \verb|diffusion/steady_state/continuous_2d_d03_p00|
    (\Cref{idec/diffusion/continuous/steady_state/examples/2d_d03_p00-example}):
    solutions for flow rate on mesh \verb|circular_18_10_forman|}
  \label{figure:idec/diffusion/steady_state/continuous_2d_d03_p00/circular_18_10_forman_flow_rate}
\end{figure}

\begin{example}
  \label{idec/diffusion/continuous/transient/examples/2d_d03_p01-example}
  Consider the transient continuous heat transport problem
  (\Cref{idec/diffusion/continuous/transient/primal_strong-formulation},
   \Cref{idec/diffusion/continuous/transient/primal_weak-formulation},
   \Cref{idec/diffusion/continuous/transient/mixed_weak-formulation})
  with input data \verb|2d_d03_p01| in the nomenclature of the C codebase.

  Concretely,
    $X = \set{(x, y) \in \R^2}{x^2 + y^2 \leq 1}$,
    $\pi_0 \equiv 4$,
    $\kappa_1 \equiv 1$,
    $u_0(x, y) = 2 - (x^2 + y^2)$,
    $f \equiv -4\, d x \wedge d y$,
    $G_D = \set{(x, y) \in \partial X}{x \geq 0}$,
    $G_N = \set{(x, y) \in \partial X}{x \leq 0}$,
    $g_D \equiv 1$,
    $g_N(t) = 2 t \, d t$
    (with respect to the $(x, y) = (\cos(t), \sin(t))$ coordinates).

  This problem has the following exact solution in steady-state:
  \begin{subequations}
    \begin{alignat}{3}
      & u(x, y) && = && x^2 + y^2, \\
      & q(x, y) && = && -2 y\, d x + 2 x\, d y.
    \end{alignat}
  \end{subequations}
  Consider a mesh $M$ for $X$ consisting of $n_a$ rays and $n_d$ disks
  with Forman subdivision $K$.
  Its potential and flow rate on $K$ consisting of the $3$ discussed cochain
  methods for $(n_a, n_d) = (4, 3)$ are shown on
  \Cref{figure:idec/diffusion/transient/continuous_2d_d03_p01/disk_polar_4_3_forman_trapezoidal_0p05_1000_potential}
  and
  \Cref{figure:idec/diffusion/transient/continuous_2d_d03_p01/disk_polar_4_3_forman_trapezoidal_0p05_1000_flow_rate}.

  {\color{red} At the moment there are problems with the primal strong and mixed
  methods!!!}
\end{example}

\begin{figure}[!ht]
  \begin{subfigure}{.32\textwidth}
    \centering
    \includegraphics[scale=.2, page=1]
    {diffusion/transient/continuous_2d_d03_p01/primal_strong_cochain_circular_4_3_forman_trapezoidal_0p05_1000_potential}
    \caption{Primal strong potential, 1}
  \end{subfigure}
  \begin{subfigure}{.32\textwidth}
    \centering
    \includegraphics[scale=.2, page=101]
    {diffusion/transient/continuous_2d_d03_p01/primal_strong_cochain_circular_4_3_forman_trapezoidal_0p05_1000_potential}
    \caption{Primal strong potential, 101}
  \end{subfigure}
  \begin{subfigure}{.32\textwidth}
    \centering
    \includegraphics[scale=.2, page=201]
    {diffusion/transient/continuous_2d_d03_p01/primal_strong_cochain_circular_4_3_forman_trapezoidal_0p05_1000_potential}
    \caption{Primal strong potential, 201}
  \end{subfigure}

  \begin{subfigure}{.32\textwidth}
    \centering
    \includegraphics[scale=.2, page=1]
    {diffusion/transient/continuous_2d_d03_p01/primal_weak_cochain_circular_4_3_forman_trapezoidal_0p05_1000_potential}
    \caption{Primal weak potential, 1}
  \end{subfigure}
  \begin{subfigure}{.32\textwidth}
    \centering
    \includegraphics[scale=.2, page=101]
    {diffusion/transient/continuous_2d_d03_p01/primal_weak_cochain_circular_4_3_forman_trapezoidal_0p05_1000_potential}
    \caption{Primal weak potential, 101}
  \end{subfigure}
  \begin{subfigure}{.32\textwidth}
    \centering
    \includegraphics[scale=.2, page=201]
    {diffusion/transient/continuous_2d_d03_p01/primal_weak_cochain_circular_4_3_forman_trapezoidal_0p05_1000_potential}
    \caption{Primal weak potential, 201}
  \end{subfigure}

  \begin{subfigure}{.32\textwidth}
    \centering
    \includegraphics[scale=.2, page=1]
    {diffusion/transient/continuous_2d_d03_p01/mixed_weak_cochain_circular_4_3_forman_trapezoidal_0p05_1000_potential}
    \caption{Mixed weak potential, 1}
  \end{subfigure}
  \begin{subfigure}{.32\textwidth}
    \centering
    \includegraphics[scale=.2, page=101]
    {diffusion/transient/continuous_2d_d03_p01/mixed_weak_cochain_circular_4_3_forman_trapezoidal_0p05_1000_potential}
    \caption{Mixed weak potential, 101}
  \end{subfigure}
  \begin{subfigure}{.32\textwidth}
    \centering
    \includegraphics[scale=.2, page=201]
    {diffusion/transient/continuous_2d_d03_p01/mixed_weak_cochain_circular_4_3_forman_trapezoidal_0p05_1000_potential}
    \caption{Mixed weak potential, 201}
  \end{subfigure}

  \begin{subfigure}{.32\textwidth}
    \centering
    \includegraphics[scale=.2, page=1]
    {diffusion/transient/continuous_2d_d03_p01/primal_strong_cochain_circular_4_3_forman_trapezoidal_0p05_1000_flow}
    \caption{Primal strong flow, 1}
  \end{subfigure}
  \begin{subfigure}{.32\textwidth}
    \centering
    \includegraphics[scale=.2, page=101]
    {diffusion/transient/continuous_2d_d03_p01/primal_strong_cochain_circular_4_3_forman_trapezoidal_0p05_1000_flow}
    \caption{Primal strong flow, 101}
  \end{subfigure}
  \begin{subfigure}{.32\textwidth}
    \centering
    \includegraphics[scale=.2, page=201]
    {diffusion/transient/continuous_2d_d03_p01/primal_strong_cochain_circular_4_3_forman_trapezoidal_0p05_1000_flow}
    \caption{Primal strong flow, 201}
  \end{subfigure}

  \begin{subfigure}{.32\textwidth}
    \centering
    \includegraphics[scale=.2, page=1]
    {diffusion/transient/continuous_2d_d03_p01/primal_weak_cochain_circular_4_3_forman_trapezoidal_0p05_1000_flow}
    \caption{Primal weak flow, 1}
  \end{subfigure}
  \begin{subfigure}{.32\textwidth}
    \centering
    \includegraphics[scale=.2, page=101]
    {diffusion/transient/continuous_2d_d03_p01/primal_weak_cochain_circular_4_3_forman_trapezoidal_0p05_1000_flow}
    \caption{Primal weak flow, 101}
  \end{subfigure}
  \begin{subfigure}{.32\textwidth}
    \centering
    \includegraphics[scale=.2, page=201]
    {diffusion/transient/continuous_2d_d03_p01/primal_weak_cochain_circular_4_3_forman_trapezoidal_0p05_1000_flow}
    \caption{Primal weak flow, 201}
  \end{subfigure}
  
  \begin{subfigure}{.32\textwidth}
    \centering
    \includegraphics[scale=.2, page=1]
    {diffusion/transient/continuous_2d_d03_p01/mixed_weak_cochain_circular_4_3_forman_trapezoidal_0p05_1000_flow}
    \caption{Mixed weak flow, 1}
  \end{subfigure}
  \begin{subfigure}{.32\textwidth}
    \centering
    \includegraphics[scale=.2, page=101]
    {diffusion/transient/continuous_2d_d03_p01/mixed_weak_cochain_circular_4_3_forman_trapezoidal_0p05_1000_flow}
    \caption{Mixed weak flow, 101}
  \end{subfigure}
  \begin{subfigure}{.32\textwidth}
    \centering
    \includegraphics[scale=.2, page=201]
    {diffusion/transient/continuous_2d_d03_p01/mixed_weak_cochain_circular_4_3_forman_trapezoidal_0p05_1000_flow}
    \caption{Mixed weak flow, 201}
  \end{subfigure}
  \cprotect\caption{Potential and flow for the \verb|2d_d03_p01| problem}
  \label{figure:idec/diffusion/transient/continuous_2d_d03_p01/circular_4_3_forman_trapezoidal_0p05_1000}
\end{figure}


\section{Continuous electromagnetism}
\label{section:continuous_electromagnetism}
\phantom{T}
\begin{discussion}
  The quantities participating in the classical electromagnetism are given in
  \Cref{table:electromagnetism/continuous/quantities}.
  The Maxwell's equations and the Poynting's theorem are given in
  \Cref{table:electromagnetism/continuous/laws}.
  The linear constitutive laws in the macroscopic formulation are given in
  \Cref{table:electromagnetism/continuous/constitutive_relations}.
  (Note that $\mu = \mu_0$ and $\varepsilon = \varepsilon_0$ in the microscopic
  formulation.
  Conductivity is not used there since $J$ is given.)
\end{discussion}

\begin{table}[!ht]
  \caption{Quantities in electromagnetism with forms}
  \label{table:electromagnetism/continuous/quantities}
  \centering
  \begin{tabular}{|l|l|l|l|l|}
    \hline
    Quantity
    & Variable
    & Spatial domain
    & Definition
    & Dimension \topStrut \\[2pt]
    \hline
    \hline
    Electric charge
    & $Q$
    & $\Omega^0 M$
    & given
    & $\charge$ \topStrut \\[2pt]
    \hline
    Electric charge
    & $J$
    & $\Omega^2 M$
    & given (or via constitutive law)
    & $\time^{-1} \charge$ \topStrut \\[2pt]
    \hline
    Electric potential
    & $\varphi$
    & $\Omega^0 M$
    & unknown (gauge freedom)
    & $\mass \length^2 \time^{-2} \charge^{-1}$ \topStrut \\[2pt]
    \hline
    Magnetic potential
    & $A$
    & $\Omega^1 M$
    & unknown (gauge freedom)
    & $\mass \length^2 \time^{-1} \charge^{-1}$ \topStrut \\[2pt]
    \hline
    Electric field
    & $E$
    & $\Omega^1 M$
    & $- d \varphi - \frac{\partial A}{\partial t}$
    & $\mass \length^2 \time^{-2} \charge^{-1}$ \topStrut \\[2pt]
    \hline
    Electric displacement
    & $D$
    & $\Omega^2 M$
    & via constitutive law
    & $\charge$ \topStrut \\[2pt]
    \hline
    Magnetic field
    & $B$
    & $\Omega^2 M$
    & $d_1 A$
    & $\mass \length^2 \time^{-1} \charge^{-1}$ \topStrut \\[2pt]
    \hline
    Magnetisation
    & $H$
    & $\Omega^1 M$
    & via constitutive law
    & $\time^{-1} \charge$ \topStrut \\[2pt]
    \hline
    Poynting form
    & $S$
    & $\Omega^2 M$
    & $E \wedge H$
    & $\mass \length^2 \time^{-3}$ \topStrut \\[2pt]
    \hline
    Electric energy form
    & $u_\mathcal{E}$
    & $\Omega^3 M$
    & $(E \wedge D) / 2$
    & $\mass \length^2 \time^{-2}$ \topStrut \\[2pt]
    \hline
    Magnetic energy form
    & $u_\mathcal{M}$
    & $\Omega^3 M$
    & $(B \wedge H) / 2$
    & $\mass \length^2 \time^{-2}$ \topStrut \\[2pt]
    \hline
    Electromagnetic energy form
    & $u$
    & $\Omega^3 M$
    & $u_\mathcal{E} + u_\mathcal{M}$
    & $\mass \length^2 \time^{-2}$ \topStrut \\[2pt]
    \hline
    Lorentz force form
    & F
    & $\Omega^2 M$
    & $\star_1 (\star_3 Q \wedge E) + \star_2 J \wedge \star_2 B$
    & $\mass \time^{-2}$ \topStrut \\[2pt]
    \hline
    Permittivity
    & $\varepsilon$
    & $\Omega^2 M \to \Omega^2 M$
    & material parameter
    & $\mass^{-1} \length^{-3} \time^2 \charge^2$ \topStrut \\[2pt]
    \hline
    Permeability
    & $\mu$
    & $\Omega^2 M \to \Omega^2 M$
    & material parameter
    & $\mass \length \charge^{-2}$ \topStrut \\[2pt]
    \hline
    Conductivity
    & $\sigma$
    & $\Omega^2 M \to \Omega^2 M$
    & material parameter
    & $\mass^{-1} \length^{-3} \time \charge^2$ \topStrut \\[2pt]
    \hline
  \end{tabular}
\end{table}

\begin{table}[!ht]
  \caption{Laws of electromagnetism with forms}
  \label{table:electromagnetism/continuous/laws}
  \centering
  \begin{tabular}{|l|l|l|l|}
    \hline
    Name
    & Equation
    & Domain
    & Dimension \topStrut \\[2pt]
    \hline
    \hline
    Gauss's law for electricity
    & $d_2 D = Q$
    & $\Omega^3 M$
    & $\charge$ \topStrut \\[2pt]
    \hline
    Gauss's law for magnetism
    & $d_2 B = 0$
    & $\Omega^3 M$
    & $\mass \length^2 \time^{-1} \charge ^{-1}$ \topStrut \\[2pt]
    \hline
    Faradey's law of induction
    & $\frac{\partial B}{\partial t} = - d_1 E$
    & $\Omega^2 M$
    & $\mass \length^2 \time^{-2} \charge^{-1}$ \topStrut \\[2pt]
    \hline
    Ampere's circuital law
    & $\frac{\partial D}{\partial t} = d_1 H - J$
    & $\Omega^2 M$
    & $\time^{-1} \charge$ \topStrut \\[2pt]
    \hline
    Poynting's theorem
    & $\frac{\partial u}{\partial t} = - d_2 S - E \wedge J$
    & $\Omega^3 M$
    & $\mass \length^2 \time^{-3}$ \topStrut \\[2pt]
    \hline
  \end{tabular}
\end{table}

\begin{table}[!ht]
  \caption{Linear constitutive relations in elecromagnetism}
  \label{table:electromagnetism/continuous/constitutive_relations}
  \centering
  \begin{tabular}{|l|l|l|l|}
    \hline
    Name
    & Equation
    & Domain
    & Dimension \topStrut \\[2pt]
    \hline
    \hline
    Polarization relation
    & $D = \varepsilon \star_1 E$
    & $\Omega^2 M$
    & $\charge$ \topStrut \\[2pt]
    \hline
    Magnetization relation
    & $B = \mu \star_1 H$
    & $\Omega^2 M$
    & $\mass \length^2 \time^{-1} \charge ^{-1}$ \topStrut \\[2pt]
    \hline
    Ohm's relation
    & $J = \sigma \star_1 E$
    & $\Omega^2 M$
    & $\time^{-1} \charge$ \topStrut \\[2pt]
    \hline
  \end{tabular}
\end{table}


\section{Discrete elasticity}
\label{section:discrete_elasticity}
\phantom{T}
\begin{discussion}[Discrete elasticity]
  Let $M$ be a mesh of dimension $3$, $K$ be the Forman subdivision of $M$.
  Let $L$ and $F$ denote length and force measures respectively.

  Discrete displacement is represented by
  \begin{equation}
    \eta^1 [L^2] \in C^1 K.
  \end{equation}
  Displacement gradient is represented by
  \begin{equation}
    \epsilon^0 [1] \in C^0 K,\ \omega^2 [L^2] \in C^2 K.
  \end{equation}
  Stress (force) is represented by
  \begin{equation}
    \tau^0 [F] \in C^0 K,\ \tau^2 [F L^2] \in C^2 K.
  \end{equation}
  Body force is represented by
  \begin{equation}
    \mathfrak{b}^1 [F] \in C^1 K.
  \end{equation}
  Let $\lambda, \mu [F] \in \R$ be the Lam{\'e} parameters.
  Our model is the following.
  \begin{subequations}
    \begin{align}
      & \epsilon^0 = \delta_1^\star \eta^1
      & (\text{volumetric displacement gradient}), \\
      & \omega^2 = \delta_1 \eta^1
      & (\text{deviatoric displacement gradient}), \\
      & \tau^0 = \lambda \epsilon^0
      & (\text{hydrostatic force}), \\
      & \tau^2 = \mu \omega^2
      & (\text{deviatoric force}), \\
      & \delta_0 \tau^0 + \delta_2^\star \tau^2 + \mathfrak{b}^1 = 0
      & (\text{conservation of linear momentum}).
    \end{align}
  \end{subequations}
\end{discussion}

\begin{example}
  Consider the problem of a twist of a cylindrical bar described
  in Section 9.1 of
  (\href
    {https://www.sciencedirect.com/science/article/pii/S0020768311001727}
    {Hadjesfandiari 2011}).
  Let
    $\theta$ be the constant angle of twist per unit length,
    $\lambda$ and $\mu$ be the Lame parameters.
  Let
    ${\bf u}$ be the displacement vector,
    $\boldsymbol{\epsilon}$ be the strain tensor,
    $\boldsymbol{\omega}$ be the rotation tensor,
    $\boldsymbol{\sigma}$ be the stress tensor.
  Then at any point $x = (x_1, x_2, x_3)$ we have:
  \begin{equation}
    {\bf u} =
    \begin{pmatrix}
      -\theta x_2 x_3 \\
      \theta x_1 x_3 \\
      0
    \end{pmatrix},\
    \boldsymbol{\epsilon} =
      \frac{\theta}{2}
      \begin{pmatrix}
        0 & 0 & - x_2  \\
        0 & 0 & x_1 \\
        -x_2 & x_1 & 0
      \end{pmatrix},\
    \boldsymbol{\omega} =
      \frac{\theta}{2}
      \begin{pmatrix}
        0 & -2 x_3 & - x_2  \\
        2 x_3 & 0 & x_1 \\
        x_2 & -x_1 & 0
      \end{pmatrix},\
    \boldsymbol{\sigma} =
      \mu \theta
      \begin{pmatrix}
        0 & 0 & - x_2  \\
        0 & 0 & x_1 \\
        -x_2 & x_1 & 0
      \end{pmatrix}.
  \end{equation}
  Note that the skew-symmetric matrix $\boldsymbol{\omega}$ corresponds to the
  vector
  \begin{equation}
    \label{cmc/discrete_elasticity/example_9_1:equation:rotation_vector}
    \boldsymbol{\omega} \mapsto \frac{\theta}{2} (-x_1, -x_2, 2 x_3)^T.
  \end{equation}
  Let $h \in \R^+$ and consider a $3$D regular grid $K$ of size $h$.
  For integers $i, j, k$, nodes in $K$ have coordinates
  \begin{equation}
    {\bf x}_{(i, j, k)} := (i h, j h, k h).
  \end{equation}
  Nodes in $K$ will be denoted by $\mathcal{N}_{(i, j, k)}$.

  There are three type of edges in $K$ (parallel to the $3$ axes)
  constructed as follows.
  Let $p \in \{1, 2, 3\}$ and $e_p$ be the $p$-th unit vector.
  Denote by $\mathcal{E}^{(p)}_{(i, j, k)}$ the edge starting at
  $\mathcal{N}_{(i, j, k)}$ and ending at $\mathcal{N}_{(i, j, k) + e_p}$.
  In particular, the oriented boundary of $\mathcal{E}^{(1)}_{(i, j, k)}$ is
  \begin{equation}
    \partial_1 \mathcal{E}^{(1)}_{(i, j, k)} =
    - \mathcal{N}_{(i, j, k)} + \mathcal{N}_{(i + 1, j, k)}.
  \end{equation}
  Similar computation holds for $p = 2$ and $p = 3$.

  For $p, q \in \{1, 2, 3\},\ p < q$ faces in $K$ are denoted by
  $\mathcal{F}^{(p, q)}_{(i, j, k)}$ and represent squares starting at
  $\mathcal{N}_{(i, j, k)}$ with basis vectors going in directions
  $e_p$ and $e_q$.
  Also use the identification of cochains
  \begin{equation}
    \mathcal{F}^{(q, p)}_{(i, j, k)} := -\mathcal{F}^{(p, q)}_{(i, j, k)}.
  \end{equation}
  For instance, the oriented boundary of $\mathcal{F}^{(1, 2}_{(i, j, k)}$ is
  \begin{equation}
    \partial_2 \mathcal{F}^{(1, 2)}_{(i, j, k)} =
    - \mathcal{E}^{(1)}_{(i, j + 1, k)}
    + \mathcal{E}^{(1)}_{(i, j, k)}
    + \mathcal{E}^{(2)}_{(i + 1, j, k)}
    - \mathcal{E}^{(2)}_{(i, j, k)}.
  \end{equation}
  We will work with the approximation $\eta^1 := \underline{u}$.

  Let $p \in \{1, 2, 3\}$.
  Then for $\epsilon^0 := \delta_1^\star \eta^1$, using the fact that
  \begin{equation}
    \eta^1 \mathcal{E}^{(p)}_{(i, j, k) + e_p} =
    \eta^1 \mathcal{E}^{(p)}_{(i, j, k)},
  \end{equation}
  we calculate:
  \begin{equation}
    \epsilon^0 \mathcal{N}_{(i, j, k)}
    =
      \frac{1}{h^2}
      \sum_{p = 1}^3 (
          \eta^1 \mathcal{E}^{(p)}_{(i, j, k)}
        - \eta^1 \mathcal{E}^{(p)}_{(i, j, k) + e_p}
      )
    = 0.
  \end{equation}
  Hence,
  \begin{equation}
    \tau^0 = \lambda \epsilon^0 = 0.
  \end{equation}
  Using the computation from
  \Cref{cmc/vector_field_to_1_cochain/1d_example:exact_value}
  in each direction, we get
  \begin{subequations}
    \begin{alignat}{2}
      & \eta^1 \mathcal{E}^{(1)}_{(i, j, k)} &&
        = \frac{h}{2} \theta (-x_2 x_3 - x_2 x_3)
        = - \theta h x_2 x_3
        = - \theta j k h^3, \\
      & \eta^1 \mathcal{E}^{(2)}_{(i, j, k)} &&
        = \frac{h}{2} \theta (x_1 x_3 + x_1 x_3)
        = \theta h x_1 x_3
        = \theta i k h^3, \\
      & \eta^1 \mathcal{E}^{(3)}_{(i, j, k)} &&
        = \frac{h}{2} \theta (0 + 0)
        = 0.
    \end{alignat}
  \end{subequations}
  For $\omega^2 := \delta_1 \eta^1$, using that
  $\epsilon(c_2) = \eta^1(\partial c_2)$, we get:
  \begin{subequations}
    \begin{alignat}{3}
      & \omega^2 \mathcal{F}^{(2, 3)}_{(i, j, k)}
      && =
        ( - \eta^1 \mathcal{E}^{(2)}_{(i, j, k + 1)}
          + \eta^1 \mathcal{E}^{(2)}_{(i, j, k)}
        )
      + ( \eta^1 \mathcal{E}^{(3)}_{(i, j + 1, k)}
          - \eta^1 \mathcal{E}^{(3)}_{(i, j, k)}
        )
      = - \theta i h^3 + 0
      && = - \theta i h^3, \\
%
      & \omega^2 \mathcal{F}^{(3, 1)}_{(i, j, k)}
      && =
        ( - \eta^1 \mathcal{E}^{(3)}_{(i + 1, j, k)}
          + \eta^1 \mathcal{E}^{(3)}_{(i, j, k)}
        )
      + ( \eta^1 \mathcal{E}^{(1)}_{(i, j, k + 1)}
          + \eta^1 \mathcal{E}^{(1)}_{(i, j, k)}
        )
      = 0 - \theta j h^3
      && = - \theta j h^3, \\
%
      & \omega^2 \mathcal{F}^{(1, 2)}_{(i, j, k)}
      && =
        ( - \eta^1 \mathcal{E}^{(1)}_{(i, j + 1, k)}
          + \eta^1 \mathcal{E}^{(1)}_{(i, j, k)}
        )
      + ( \eta^1 \mathcal{E}^{(2)}_{(i + 1, j, k)}
          - \eta^1 \mathcal{E}^{(2)}_{(i, j, k)}
        )
      = \theta k h^3 + \theta k h^3
      && = 2 \theta k h^3.
    \end{alignat}
  \end{subequations}
  We see the clear correspondence (by a factor of $2 h^2$) to the ``flattened''
  version of $\boldsymbol{\omega}$,
  \Cref{cmc/discrete_elasticity/example_9_1:equation:rotation_vector}.

  We have $\tau^2 = \mu \omega^2$ and hence
  $\delta_2^\star \tau^2 = \mu \delta_2^\star \omega^2$.
  Then
  \begin{subequations}
    \begin{alignat}{3}
      & (\delta_2^\star \tau^2) \mathcal{E}^{(1)}_{(i, j, k)}
      && =
        \frac{\mu}{h^2}
        (
          ( \omega^2 \mathcal{F}^{(1, 2)}_{(i, j, k)}
            - \omega^2 \mathcal{F}^{(1, 2)}_{(i, j - 1, k)}
          )
          -
          ( \omega^2 \mathcal{F}^{(1, 3)}_{(i, j, k)}
            - \omega^2 \mathcal{F}^{(1, 3)}_{(i, j, k - 1)}
          )
        )
      = \frac{\mu}{h^2} (0 - 0)
      && = 0, \\
%
      & (\delta_2^\star \tau^2) \mathcal{E}^{(2)}_{(i, j, k)}
      && =
        \frac{\mu}{h^2}
        (
          ( \omega^2 \mathcal{F}^{(1, 2)}_{(i - 1, j, k)}
            - \omega^2 \mathcal{F}^{(1, 2)}_{(i, j, k)}
          )
          -
          ( \omega^2 \mathcal{F}^{(2, 3)}_{(i, j, k)}
            - \omega^2 \mathcal{F}^{(2, 3)}_{(i, j, k - 1)}
          )
        )
      = \frac{\mu}{h^2} (0 - 0)
      &&  = 0, \\
%
      & (\delta_2^\star \tau^2) \mathcal{E}^{(3)}_{(i, j, k)}
      && =
        \frac{\mu}{h^2}
        (
          ( \omega^2 \mathcal{F}^{(1, 3)}_{(i - 1, j, k)}
            - \omega^2 \mathcal{F}^{(1, 3)}_{(i, j, k)}
          )
          -
          ( \omega^2 \mathcal{F}^{(2, 3)}_{(i, j - 1, k)}
            - \omega^2 \mathcal{F}^{(2, 3)}_{(i, j, k)}
          )
        )
      = \frac{\mu}{h^2} (0 - 0)
      && = 0.
    \end{alignat}
  \end{subequations}
  Hence, $\delta_2^\star \tau^2 = 0$.
  With zero body force $\mathfrak{b}^1$, we get:
  \begin{equation}
    \delta_0 \tau^0 + \delta_2^\star \tau^2 + \mathfrak{b}^1 = 0 + 0 + 0 = 0.
  \end{equation}
\end{example}


\section{Discrete vector bundles and covariant exterior derivative}
\phantom{T}
\label{section:discrete_vector_bundles_and_covariant_exterior_derivative}
\begin{definition}
  Let
    $d \in \N$,
    $K$ be a quasi-cubical mesh of dimension $d$,
    $\partial$ be a boundary operator on $K$,
    $p \in \{1, ..., d\}$.
  The \textbf{adjoint coboundary operator on $p$-cochain}
  $\delta^\star_p \colon C^p K \to C^{p - 1} K$
  is defined as the adjoint of $\delta_{p - 1}$
  with respect to the inner product of cochains.
  In other words, for any $\pi^p \in C^p K,\ \rho^{p - 1} \in C^{p - 1}$,
  \begin{equation}
    \inner{\delta_p^\star \pi^p}{\rho^{p - 1}}_{p - 1}
    = \inner{\pi^p}{\delta_{p - 1} \rho^{p - 1}}_p.
  \end{equation}
  The operator $\delta^\star_p$ has physical dimension $[L^{-2}]$.
\end{definition}

\begin{definition}
  Let
    $K$ be a quasi-cubical mesh,
    $V$ be a vector space,
    $p \in \N$,
    $q \in \N$.
  Define the \textbf{cup product of a vector-valued cochain with a cochain}
  \begin{equation}
    \usmile \colon C^p(K, V) \times C^q K \to C^{p + q}(K, V)
  \end{equation}
  as follows: for any $v \in V$, $\tau^p \in C^p K$, $\sigma^q \in C^q K$,
  \begin{equation}
    (v \otimes \tau^p) \usmile \sigma^q := v \otimes (\tau^p \smile \sigma^q),
  \end{equation}
  and extend it by linearity on $C^p(K, V) \times C^q K$.
\end{definition}

\begin{remark}
  Let
    $K$ be a quasi-cubical mesh,
    $V$ be a vector space,
    $v \in V$,
    $p \in \N$,
    $\sigma^p \in C^p K$.
  Denote by $1$ the identity zero-cochain on $K$.
  Then
  \begin{equation}
  \label{cmc:tensor_to_cup:equation}
    v \otimes \sigma^p
    = v \otimes (1 \smile \sigma^p)
    = (v \otimes 1) \usmile \sigma^p.
  \end{equation}
  Abuse the notation and identify $v \in V$ with $v \otimes 1 \in C^0(K, V)$.
  Then \Cref{cmc:tensor_to_cup:equation} reads as
  \begin{equation}
    v \otimes \sigma^p = v \usmile \sigma^p.
  \end{equation}
\end{remark}

\begin{definition}
  Let
    $d \in \N$,
    $K$ be a quasi-cubical mesh of dimension $d$,
    $\partial$ be a boundary operator on $K$,
    $p \in \{1, ..., d\}$.
  The \textbf{adjoint coboundary operator on $p$-cochain}
  $\delta^\star_p \colon C^p K \to C^{p - 1} K$
  is defined as the adjoint of $\delta_{p - 1}$
  with respect to the inner product of cochains.
  In other words, for any $\pi^p \in C^p K,\ \rho^{p - 1} \in C^{p - 1}$,
  \begin{equation}
    \inner{\delta_p^\star \pi^p}{\rho^{p - 1}}_{p - 1}
    = \inner{\pi^p}{\delta_{p - 1} \rho^{p - 1}}_p.
  \end{equation}
  The operator $\delta^\star_p$ has physical dimension $[L^{-2}]$.
\end{definition}

\begin{proposition}
  Let
    $K$ be a quasi-cubical mesh,
    $V$ be a vector space,
    $v \in V$,
    $p \in \N$,
    $\sigma^p \in C^p K$.
  Then
  \begin{equation}
    \nabla_p(v \usmile \sigma^p)
    = \nabla_0 v \usmile \sigma^p + v \usmile \delta_p \sigma^p.
  \end{equation}
\end{proposition}

\begin{proof}
  \begin{equation}
    \nabla_p(v \usmile \sigma^p)
    := \nabla_p(v \otimes \sigma^p)
    = \nabla_p((v \otimes 1) \usmile \sigma^p)
    = \nabla_0(v \otimes 1) \usmile \sigma^p
      + (v \otimes 1) \usmile \delta_p \sigma^p
    =: \nabla_0 v \usmile \sigma^p + v \usmile \delta_p \sigma^p.
  \end{equation}
\end{proof}

\begin{proposition}
  Let
    $K$ be a quasi-cubical mesh,
    $V$ be a vector space,
    $\nabla \colon C^\bullet(K, V) \to C^\bullet(K, V)$ be
      a discrete exterior covariant derivative.
  Then $\nabla$ satisfies the graded Leibniz rule:
  for any $p, q \in \N$, $\sigma_V^p \in C^p(K, V)$, $\tau^q \in C^q K$,
  \begin{equation}
    \nabla_{p + q}(\sigma_V^p \usmile \tau^q)
    = \nabla_p \sigma_V^p \usmile \tau^q
      + (-1)^p \sigma_V^p \usmile \delta_q \tau^q.
  \end{equation}
\end{proposition}

\begin{proof}
  It is enough to prove the proposition for a product element $\sigma_V^p$.
  Let $v \in V$, $\theta^p \in C^p K$ and $\sigma_V^p = v \usmile \theta^p$.
  Then
  \begin{equation}
    \begin{split}
      \nabla_{p + q}(\sigma_V^p \usmile \tau^q)
      & = \nabla_{p + q}(v \usmile (\theta^p \smile \tau^q)) \\
      & = \nabla_0 v \usmile (\theta^p \smile \tau^q)
          + v \usmile (\delta_p \theta^p \smile \tau^q)
          + (-1)^p v \usmile (\theta^p \smile \delta_q \tau^q) \\
      & = \nabla_p(v \usmile \theta^p) \usmile \tau^q
          + (-1)^p (v \usmile \theta^p) \usmile \delta_q \tau^q \\
      & = \nabla_p \sigma_V^p \usmile \tau^q
          + (-1)^p \sigma_V^p \usmile \delta_q \tau^q.
    \end{split}
  \end{equation}
\end{proof}


\end{document}
