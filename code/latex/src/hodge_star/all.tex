\begin{definition}
  Let $V$ be a vector space over $\R$.
  An \textbf{inner product} on $V$ is a function
  $g \colon V \times V \to \R$ that is:
  \begin{enumerate}
    \item
      bilinear, i.e., for all $\lambda, \mu \in \R$, $u, v, w \in V$:
      \begin{equation}
        g(\lambda u + \mu v, w) = \lambda g(u, w) + \mu g(v, w),\
        g(w, \lambda u + \mu v) = \lambda g(w, u) + \mu g(w, v);
      \end{equation}
    \item
      symmetric, i.e., for all $u, v \in V$,
      \begin{equation}
        g(u, v) = g(v, u);
      \end{equation}
    \item
      positive definite: for all $v \in V \setminus \{0\}$,
      \begin{equation}
        g(v, v) > 0.
      \end{equation}
  \end{enumerate}
  If $g$ is an inner product on $V$, the pair $(V, g)$ is called an
  \textbf{real inner product space}.
\end{definition}
\begin{proposition}
  Let $(V, g)$ be a finite-dimensional real inner product space.
  Define the map $\tilde{g} \colon V \to V^*$ as follows: for any $v \in V$,
  \begin{equation}
    \tilde{g}(v) := (w \in V \mapsto g(v, w)).
  \end{equation}
  Then $\tilde{g}$ is an isomorphism.
\end{proposition}
\begin{definition}
  Let
    $d \in \N$,
    $(V, g)$ be a real inner product space of dimension $d$,
    $e = \{e_0, ..., e_{d - 1}\}$ be a basis of $V$.
  We say that $e$ is an \textbf{orthogonal basis} if any two disjoint elements
  are orthogonal, i.e., for all $i, j \in \{0, ..., d - 1\}$,
  if $i \neq j$, then
  \begin{equation}
    g(e_i, e_j) = 0.
  \end{equation}
  The basis $e$ is called \textbf{orthonormal}, if it is orthogonal and
  for any $i \in \{0, ..., d - 1\}$,
  \begin{equation}
     g(e_i, e_i) = 1.
  \end{equation}
  An equivalent way of saying that $e$ is orthonormal is by using the Kronecker
  delta symbol: for all $i, j \in \{0, ..., d - 1\}$,
  \begin{equation}
    g(e_i, e_j) = \delta_{i, j} :=
    \begin{cases}
      0, & \text{if $i \neq j$} \\
      1, & \text{if $i = j$}
    \end{cases}.
  \end{equation}
\end{definition}
\begin{definition}
  Let
    $d \in \N$,
    $(V, g)$ be a real inner product space of dimension $d$,
    $p \in \{0, ..., d\}$.
  Define the \textbf{inner product on $p$-vectors}
  $\Lambda^p g \colon \Lambda^p V \times \Lambda^p V \to \R$ as follows:
  for any $v_0, ..., v_{p - 1},\ w_0, ..., w_{p - 1} \in V$,
  \begin{equation}
    (\Lambda^p g)
    (v_0 \wedge ... \wedge v_{p - 1}, w_0 \wedge ... \wedge w_{p - 1})
    := \det (g(v_i, w_j))_{i, j = 0}^{p - 1}
  \end{equation}
  (in other words, on simple $p$-vectors $\Lambda^p g$ is constructed using
  the Gram determinant).
  For arbitrary $p$-vectors, expand the above definition by bilinearity.

  We define the \textbf{exterior algebra inner product}
  \begin{equation}
    \Lambda^\bullet g \colon \Lambda^\bullet V \times \Lambda^\bullet V \to \R
  \end{equation}
  as follows:
  for $p, q \in \N,\ \omega_p \in \Lambda^p V,\ \eta_q \in \Lambda_q V$,
  \begin{equation}
    (\Lambda g)(\omega_p, \eta_q) :=
    \begin{cases}
      (\Lambda^p g)(\omega_p, \eta_q), & p = q \\
      0, & p \neq q
    \end{cases}.
  \end{equation}
\end{definition}
\begin{definition}
  Let
    $d \in \N$,
    $(V, g)$ be an oriented real inner product space of dimension $d$.
  The \textbf{volume $d$-vector} on $V$ is the unique element of $\Lambda^d V$
  which has norm $1$ and is in the same oriented class as the chosen
  orientation.
\end{definition}
\begin{remark}
  On a non-oriented real inner product space $V$ of dimension $d \in \N$
  there are exactly two elements $\omega, \eta \in\Lambda^d V^*$ that have norm
  $1$ (they are opposite to one another, i.e., $\omega = -\eta$) and so are
  candidates for a volume $d$-vector.
  Choosing one of them is equivalent to a choice of orientation on $V$.
\end{remark}
\begin{definition}
  Let
    $d \in \N$,
    $(V, g)$ be an oriented real inner product space of dimension $d$,
    $\vol$ be the volume $d$-vector on $\Lambda^d V$,
    $p \in \N$.
  The \textbf{Hodge star operator on $p$-vectors} $\star_p$
  is defined as the unique operator
  \begin{equation}
    \star_p \colon \Lambda^p V \to \Lambda^{d - p} V
  \end{equation}
  such that for any
  $\omega_p \in \Lambda^p V,\ \eta_{d - p} \in \Lambda^{d - p} V$,
  \begin{equation}
    (\Lambda^{d - p} g)(\omega_{d - p}, \star_p \eta_p) \vol
    = \omega_{d - p} \wedge \eta_p.
  \end{equation}
  The \textbf{Hodge star operator} $\star$ is the direct sum of all $\star_p$
  for $p = 0, ..., d$.
\end{definition}
\begin{example}
  Consider
    $V = \R^2$ with the standard basis $e = (e_0, e_1)$,
    inner product $g$ making $e$ orthonormal,
    volume $2$-vector $\vol := e_0 \wedge e_1$.
  Then
  \begin{subequations}
    \begin{align}
      & \star_0 1 = e_0 \wedge e_1 = \vol, \\
      & \star_1 e_0 = - e_1, \\
      & \star_1 e_1 = e_0, \\
      & \star_2 e_0 \wedge e_1 = \star_2 \vol = 1.
    \end{align}
  \end{subequations}
\end{example}
\begin{remark}
  Note that in other sources $\star_p$ is defined as satisfying
  \begin{equation}
    \omega_p \wedge (\star_p \eta_p) = (\Lambda^p g)(\omega_p, \eta_p) \vol.
  \end{equation}
  In the previous example this would lead to change of signs:
  \begin{subequations}
    \begin{align}
      & \star_1 e_0 = e_1 \\
      & \star_1 e_1 = - e_0.
    \end{align}
  \end{subequations}
  More precisely, define the $4$ operators
  $\star^{(i)}_p \colon \Lambda^p V \to \Lambda^{d - p} V$ ($i = 0, 1, 2, 3$)
  as unique solutions to the following equations
  (we denote $g_p := \Lambda^p g,\ q := d - p$):
  \begin{subequations}
    \begin{align}
      & g_q(\omega_q, \star^{(0)}_p \eta_p) \vol = \omega_q \wedge \eta_p, \\
      & g_q(\star^{(1)}_p \omega_p, \eta_q) \vol = \omega_p \wedge \eta_q, \\
      & \omega_p \wedge (\star^{(2)}_p \eta_p) = g_p(\omega_p, \eta_p) \vol, \\
      & (\star^{(3)}_p \omega_p) \wedge \eta_p = g_p(\omega_p, \eta_p) \vol.
    \end{align}
  \end{subequations}
  It can be shown that
  \begin{equation}
    \star^{(0)}_p
    = \star^{(3)}_p
    = (-1)^{p (d - p)} \star^{(1)}_p
    = (-1)^{p (d - p)} \star^{(2)}_p.
  \end{equation}
  We will choose $\star^{(0)}$ because it (along with $\star^{(1)}$) allows us
  to use alike operators in mesh calculus.
  We choose $\star^{(0)}$ over $\star^{(1)}$ because in mesh calculus its
  analogue is in widespread usage.
\end{remark}
\begin{proposition}
  Let
    $d \in \N$
    $(V, g)$ be an oriented real inner product space of dimension $d$,
    $\vol$ be the volume $d$-vector on $\Lambda^d V$,
    $p \in \N$,
    $\star$ be the Hodge star operator on $\Lambda^\bullet V$.
  Then
  \begin{equation}
    \star_{d - p} \circ \star_p = (-1)^{p (d - p)} \id_{\Lambda^p V}.
  \end{equation}
\end{proposition}
