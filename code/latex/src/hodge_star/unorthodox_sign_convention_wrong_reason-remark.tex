\begin{remark}
  Note that in other sources $\star_p$ is defined as satisfying
  \begin{equation}
    \omega_p \wedge (\star_p \eta_p) = (\Lambda^p g)(\omega_p, \eta_p) \vol.
  \end{equation}
  In the previous example this would lead to change of signs:
  \begin{subequations}
    \begin{align}
      & \star_1 e_0 = e_1 \\
      & \star_1 e_1 = - e_0.
    \end{align}
  \end{subequations}
  More precisely, define the $4$ operators
  $\star^{(i)}_p \colon \Lambda^p V \to \Lambda^{d - p} V$ ($i = 0, 1, 2, 3$)
  as unique solutions to the following equations
  (we denote $g_p := \Lambda^p g,\ q := d - p$):
  \begin{subequations}
    \begin{align}
      & g_q(\omega_q, \star^{(0)}_p \eta_p) \vol = \omega_q \wedge \eta_p, \\
      & g_q(\star^{(1)}_p \omega_p, \eta_q) \vol = \omega_p \wedge \eta_q, \\
      & \omega_p \wedge (\star^{(2)}_p \eta_p) = g_p(\omega_p, \eta_p) \vol, \\
      & (\star^{(3)}_p \omega_p) \wedge \eta_p = g_p(\omega_p, \eta_p) \vol.
    \end{align}
  \end{subequations}
  It can be shown that
  \begin{equation}
    \star^{(0)}_p
    = \star^{(3)}_p
    = (-1)^{p (d - p)} \star^{(1)}_p
    = (-1)^{p (d - p)} \star^{(2)}_p.
  \end{equation}
  We will choose $\star^{(0)}$ because it (along with $\star^{(1)}$) allows us
  to use alike operators in mesh calculus.
  We choose $\star^{(0)}$ over $\star^{(1)}$ because in mesh calculus its
  analogue is in widespread usage.
\end{remark}
