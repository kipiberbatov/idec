\begin{remark}
  If the matrix $A$ is of full rank there exists a closed formula for $A^*$.
  \begin{enumerate}
    \item
      If $A$ is a square matrix with full rank, i.e., an invertible one, then
      $A^* = A^{-1}$.
    \item
      If $m > n$ and $A$ is an $m \times n$ matrix of full rank, then its
      columns are linearly independent which means that $A^T A$ is
      symmetric and positive definite and hence invertible.
      (Its inverse $(A^T A)^{-1}$ is also symmetric and positive definite.)
      It is then easy to check that
      \begin{equation}
        B := (A^T A)^{-1} A^T
      \end{equation}
      is the Moore-Penrose inverse of $A$.
      Indeed, obviously $B$ is a left inverse of $A$, and
      \begin{subequations}
        \begin{alignat}{2}
          & A B A && = A (B A) = A I_n = A, \\
          & B A B && = (B A) B = I_n B = B, \\
          & (A B)^T && = (A (A^T A)^{-1} A^T)^T
            = (A^T)^T ((A^T A)^{-1})^T A^T = A (A^T A)^{-1} A^T = A B, \\
          & (B A)^T && = I_n^T = I_n = B A.
        \end{alignat}
      \end{subequations}
    \item
      If $m < n$ and $A$ is an $m \times n$ matrix of full rank, then an
      analogous reasoning to the previous point shows that
      \begin{equation}
        A^* = A^T (A A^T)^{-1}.
      \end{equation}
  \end{enumerate}
\end{remark}
