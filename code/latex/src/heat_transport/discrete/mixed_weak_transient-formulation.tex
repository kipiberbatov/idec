\begin{formulation}
  \label{idec/heat_transport/discrete/mixed_weak_transient-formulation}
  [Mixed weak formulation for the discrete transient heat equation]
  The following formulation is a discrete version of
  \Cref{idec/heat_transport/continuous/mixed_weak_transient-formulation}.
  Let:
  \begin{itemize}
    \item
      $d$ be a positive integer (space dimension);
    \item
      $K$ be an oriented quasi-cubical \hyperref[idec:mesh:definition]{mesh} of
      dimension $d$ representing the material body;
    \item
      $[K]$ be the fundamental class of $K$;
    \item
      $t_0 \in \R$ be the initial time;
    \item
      $I = [t_0, \infty)$ be the time-interval where the process occurs;
    \item
      $f [E T^{-1}] \colon I \to C^d K$ be the heat source;
    \item
      $u_0 [\Theta] \in C^0 K$ be the initial temperature;
    \item
      $\kappa_{d - 1} [E L^{2 - d} T^{-1} \Theta^{-1}]
      \colon C^{d - 1} K \to C^{d - 1} K$
      be the thermal conductivity of the material;
    \item
      $\tilde{\pi}_d [E L^{-d} \Theta^{-1}] \colon C^d K \to C^d K$
      be the heat capacity of the material;
    \item
      $\partial K = \Gamma_D \cup \Gamma_N$ be the partition of the boundary of
      $K$ into Dirichlet ($\Gamma_D$) and Neumann ($\Gamma_N$) regions;
    \item
      $[\Gamma_D]$ be the fundamental class of $\Gamma_D$, where $\Gamma_D$
      has the boundary orientation induced from $K$;
    \item
      $g_D^0 [\Theta] \colon I \to C^0 \Gamma_D$
      be the prescribed temperature on the Dirichlet boundary;
    \item
      $g_N^{d - 1} [E T^{-1}] \colon I \to C^{d - 1} \Gamma_N$
      be the prescribed flow on the Neumann boundary.
  \end{itemize}
  define the following operators:
  \begin{subequations}
    \begin{alignat}{3}
      & A \colon C^{d - 1} K \times (I \to C^{d - 1} K) \to \R,
        \enspace
      && A(r^{d - 1}, s^{d - 1})
        := \inner{r^{d - 1}}{\kappa_{d - 1}^{-1} s^{d - 1}}_{K, d - 1} \enspace
      && [E^{-1} T \Theta], \\
%
      & B \colon C^d K \times (I \to C^{d - 1} K) \to \R, \enspace
      && B(v^d, r^{d - 1})
        := \inner{\delta_{d - 1} r^{d - 1}}{v^d}_{K, d} \enspace
      && [L^{-d}], \\
%
      & C \colon C^d K \times (I \to C^d K) \to \R, \enspace
      && C(v^d, w^d) := \inner{\tilde{\pi}_d w^d}{v^d}_{K, d} \enspace
      && [E L^{-2 d} \Theta^{-1}], \\
%
      & G \colon C^{d - 1} K \to \R, \enspace
      && G(r^{d - 1}) :=(\tr_{\Gamma_D, d - 1} r^{d - 1} \smile g_D^0)[\Gamma_D]
        \enspace
      && [\Theta], \\
%
      & F \colon C^d K \to \R, \enspace
      && F(v^d) := \inner{f^d}{v^d}_{K, d} \enspace
      && [E T^{-1} L^{-d}].
    \end{alignat}
  \end{subequations}
  Our unknowns are:
  \begin{itemize}
    \item
      $q^{d - 1} [E T^{-1}] \colon I \to C^{d - 1} K$ (heat flow);
    \item
      $\tilde{u}^d [\Theta L^d] \colon I \to C^d K$ (dual temperature).
  \end{itemize}
  We are solving the following problem for $q^{d - 1}$ and $\tilde{u}^d$:
  \begin{subequations}
    \begin{alignat}{4}
      & \forall r^{d - 1} [E T^{-1}] \in \Ker \tr_{\Gamma_N, d - 1}, \enspace
      && A(r^{d - 1}, q^{d - 1}) + B^T(r^{d - 1}, \tilde{u}^d)
      && = G(r^{d - 1}) \enspace
      && [E T^{-1} \Theta], \\
%
      & \forall v^d [\Theta L^d] \in C^d K, \enspace
      && B(v^d, q^{d - 1}) - C(v^d, \frac{\partial \tilde{u}^d}{\partial t})
      && = - F(v^d) \enspace
      && [E T^{-1} \Theta], \\
%
      &
      && \tr_{\Gamma_N, d - 1} q^{d - 1}
      && = g_N^{d - 1} \enspace
      && [E T^{-1}], \\
%
      &
      && \tilde{u}^d(t_0)
      && = \star_{K, 0} u_0 \enspace
      && [\Theta L^d].
    \end{alignat}
  \end{subequations}
  The temperature $u^0 [\Theta] \colon I \to C^0 K$ is calculated in the
  post-processing phase by the formula
  \begin{equation}
    u^0(t, c_0) :=
    \begin{cases}
      u_0(c_0), & t = t_0 \\
      (\star_d \tilde{u}^d)(t, x),
        & t > t_0\ \text{and}\ c_0 \notin (\Gamma_D)_0 \\
      g_D^0(t, c_0), & t_0 > 0\ \text{and}\ c_0 \in (\Gamma_D)_0
    \end{cases}.
  \end{equation}
\end{formulation}
