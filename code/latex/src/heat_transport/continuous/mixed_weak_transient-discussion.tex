\begin{discussion}
  We are going to formulate the \textbf{mixed weak formulation for continuous
  heat transport with differential forms}.
  Consider the model
  \Cref{idec/heat_transport/continuous/model_with_differential_forms-discussion}
  with the same domains and variable names.
  Let $v \in \Omega^0 X$ and $r \in \Ker \tr_{\Gamma_N, 2}$ be test functions.
  Define
  \begin{equation}
    \pi_2 :=
    \star_2^{-1} \circ \pi_1 \circ \star_2 \colon \Omega^2 X \to \Omega^2 X.
  \end{equation}
  Then
  \begin{equation}
    \pi_2^{-1} q = \star_2^{-1} (d_0 u),
  \end{equation}
  and therefore
  \begin{equation}
    \begin{split}
      \inner{r}{\pi_2^{-1} q}
      & = \int_X (r \wedge \star_2 (\pi_2^{-1} q)) \\
      & = \int_X (r \wedge d_0 u) \\
      & = \int_{\partial X} (\tr_{X, 2} r \wedge \tr_{X, 0} u)
        - \int_{X} (d_2 r \wedge u) \\
      & = \int_{\Gamma_D} (\tr_{\Gamma_D, 2} r \wedge g_D)
        - \int_{X} (d_2 r \wedge u).
    \end{split}
  \end{equation}
  Multiplying the balance equation with $v$ and integrating over $X$ gives
  \begin{equation}
    \frac{d}{d t} \int_X (v \wedge \star_0 (\pi_0 u))
    = \int_X v \wedge d_2 q + \int_X (v \wedge f).
  \end{equation}
  We get the following formulation.
\end{discussion}
