\begin{discussion}
  In this section we will consider the heat transport phenomenon in both
  transient and steady-state form.
  Our formulation will be represented in the language of differential forms
  because they better represent the meaning of physical quantities.
  Various (weak) reformulations will be presented -- those reformulations will
  give us hints on how to construct purely discrete formulations.

  We will formulate our phenomenon in arbitrary dimensions, although our model
  problem is in physical $3$-dimensional space.
  The reason is that we conduct tests in different dimensions and, also,
  transport phenomena can be applied in domains with different dimensions.
\end{discussion}
