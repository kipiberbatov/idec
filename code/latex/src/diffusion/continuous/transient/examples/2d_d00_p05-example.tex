\begin{example}
  Consider the transient continuous heat transport problem
  (\Cref{idec/diffusion/continuous/transient/primal_strong-formulation},
   \Cref{idec/diffusion/continuous/transient/primal_weak-formulation},
   \Cref{idec/diffusion/continuous/transient/mixed_weak-formulation})
  with input data \verb|2d_d00_p05| in the nomenclature of the C codebase.

  Concretely,
    $X = [0, 1]^2$,
    $\pi_0 \equiv 0$,
    $\kappa_1 \equiv 1$,
    $u_0(x, y) = \sin(\pi x) \sin(\pi y)$,
    $f \equiv 0$,
    $G_D = \partial X$,
    $G_N = \emptyset$,
    $g_D(x, y) = 0$.

  This problem has the following exact solution in steady-state:
  \begin{subequations}
    \begin{alignat}{3}
      & u(x, y) && = && ?, \\
      & q(x, y) && = && ?.
    \end{alignat}
  \end{subequations}
  Consider a mesh $M$ for $X$ consisting of $5 \times 5$ squares (each axis is
  divided into $5$ segments) with Forman subdivision $K$
  ($10 \times 10$ squares).
  Its potential and flow on $K$ consisting of the exact solution and the $2$ of
  the discussed cochain methods (primal strong and mixed weak) are shown on
  \Cref{figure:idec/diffusion/transient/continuous_2d_d00_p05/brick_2d_5_forman_trapezoidal_0p001_2500}.
\end{example}
