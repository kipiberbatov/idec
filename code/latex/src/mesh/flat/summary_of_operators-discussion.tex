\begin{discussion}
  Let me summarize the operations relating cochains and embedding.
  We will use notation coming from dependent type theory for functions whose
  codomain depends on the domain.
  Namely, if
    $X$ is a type (set),
    $\{Y(x)\}_{x \in X}$ is a family of sets and
    $\{f(x) \in Y(x)\}_{x \in X}$,
  we will write
  \begin{equation}
    f \colon \prod_{x \in X} Y(x).
  \end{equation}
  Let
    $d \in \N$,
    $K$ be a quasi-cubical flat mesh of dimension $d$,
    $X$ be the manifold it encompasses.
  We define the following data:
  \begin{itemize}
    \item
      $n \colon K_0 \to \N$ denotes the number of node neighbors of a $0$-cell;
    \item
      $\widehat{\phantom{T}} \colon C^1 K \to
        \displaystyle \prod_{x_0 \in K_0} \R^{n(x_0)}$
      denotes the neighbor representation of a $1$-cochain,
      $\widehat{\phantom{T}}\ [1]$;
    \item
      $\displaystyle
        \mathcal{L} \colon \prod_{x_0 \in K_0} M_{n(x_0), d}(\R)$
      denotes the node neighbors matrix,
      $\mathcal{L}\ [L]$;
    \item
      $\displaystyle
        \star \colon \prod_{(m, n) \in \N^2} M_{m, n}(\R) \to M_{n, m}(\R)$
      denotes the Moore-Penrose inverse of a rectangular matrix,
      ($\star$ reverses physical dimensions);
    \item
      $\overline{\phantom{T}} \colon C^1 K \to \Hom_\R(C^0 K, \R^d)$
      denotes the approximation of a $1$-cochain as a Euclidean vector-valued
      $0$-cochain,
      \begin{equation}
        \overline{\pi^1} c_0 :=
        (\mathcal{L}_{c_0})^\star \cdot (\widehat{\pi^1})_{c_0},
      \end{equation}
      $\overline{\phantom{T}}\ [L^{-1}]$;
    \item
      $\underline{\phantom{T}} \colon \chi X \to C^1 K$
      denotes the discretization of a continuum vector field as a $1$-cochain,
      $\underline{\phantom{T}}\ [L]$;
  \end{itemize}
\end{discussion}
