\begin{example}
  With the mesh of the previous example, we have
  \begin{equation}
    \label{idec/vector_field_to_1_cochain/1d_example:exact_value}
    \underline{u}(\mathcal{E}_i)
    = \frac{1}{2} (h u(x_i) + h u(x_{i + 1}))
    = h \frac{u(x_i) + u(x_{i + 1})}{2}.
  \end{equation}
  Let's calculate the consecutive application of approximation and embedding
  (and vice versa).
  \begin{equation}
    \underline{\left(\overline{\pi^1}\right)}(\mathcal{E}_i)
    = h
      \left(
        \frac{\overline{\pi^1}(x_i) + \overline{\pi^1}(x_{i + 1})}{2}
      \right)
    = \frac{h}{2}
      \frac{1}{2 h}
      ((\pi^1 \mathcal{E}_{i - 1} + \pi^1 \mathcal{E}_i)
       + (\pi^1 \mathcal{E}_i + \pi^1 \mathcal{E}_{i + 1}))
    = \frac
    {\pi^1 \mathcal{E}_{i - 1} + 2 \pi^1 \mathcal{E}_i
      + \pi^1 \mathcal{E}_{i + 1}}
    {4}.
  \end{equation}
  \begin{equation}
    \overline{\left(\underline{u}\right)}(x_i)
    = \frac{1}{2 h}
      \left(
        \underline{u} \mathcal{E}_{i - 1} + \underline{u} \mathcal{E}_i
      \right)
    = \frac{1}{2 h}
      \frac{h}{2}
      ((u(x_{i - 1}) + u(x_i)) + (u(x_i) + u(x_{i + 1})))
    = \frac{u(x_{i - 1}) + 2 u(x_i) + u(x_{i + 1})}{4}.
  \end{equation}
  In both cases of composition of embedding and approximation the final result
  is the identity operator when $\pi^1$ (respectively $u$) is linear with
  respect to the index $i$.
\end{example}
