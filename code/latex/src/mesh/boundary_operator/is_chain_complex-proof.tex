\begin{proof}
  It is enough to prove that for any $p \in \{0, ..., d\}$, $c_p \in M_d$,
  \begin{equation}
    \partial^2 c_p = 0.
  \end{equation}
  The proposition is trivially true for $p = 0$ and $p = 1$
  because $\partial_0 = 0$.
  Assume that $p \geq 2$.
  Then
  \begin{equation}
    \begin{split}
      \partial^2 a_p
      & = \partial_{p - 1} (\partial_p a_p) \\
      & = \partial_{p - 1}
      \left(
        \sum_{b_{p - 1} \prec a_p} \varepsilon(a_p, b_{p - 1}) b_{p - 1}
      \right) \\
      & =
      \sum_{b_{p - 1} \prec a_p}
        \sum_{c_{p - 2} \prec b_{p - 1}}
            \varepsilon(a_p, b_{p - 1})
            \varepsilon(b_{p - 1}, c_{p - 2})
            c_{p - 2} \\
      & =
      \sum_{c_{p - 2} \prec a_p}
        \left(
          \sum _{b_{p - 1} \in (c_{p - 2}, a_p)}
            \varepsilon(a_p, b_{p - 1}) \varepsilon(b_{p - 1}, c_{p - 2})
        \right)
        c_{p - 2} \\
      & = 0
    \end{split}
  \end{equation}
  (the last equation follows from the last condition in the definition of
  \hyperref[cmc:relative_orientation:definition]{relative orientation}).
\end{proof}
