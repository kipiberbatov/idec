\begin{definition}
  A \hyperref[cmc:mesh:definition]{mesh}
  is represented in memory by its topology and geometry.
  The topology is represented using:
  \begin{itemize}
    \item
      integer \emph{dim} ($d$) storing mesh's dimension;
    \item
      array \emph{cn} storing the number of $p$-cells for $p = 0, ..., d$;
    \item
      a jagged array \emph{cf} of order $4$ storing its topology.
      More precisely, if $0 < p \leq d$, $0 \leq q < p$,
      $0 \leq i < {\rm cn}[p]$, then ${\rm cf}[p][q][i]$ stores the indices of
      all $q$-dimensional subfaces of the $i$-th $p$-cell.
  \end{itemize}
  For a mesh of flat polytopes, the topology is represented by:
  \begin{itemize}
    \item
      integer \emph{dim\_embedded} ($d$) storing mesh's embedding dimension;
    \item
      array of floating point numbers \emph{coord} of length
      ${\rm cn}[0] * \verb|dim_embedded|$ storing coordinates of the nodes
      (${\rm cn}[0]$ nodes of dimension $\R^{{\rm cn}[0]}$).
  \end{itemize}
\end{definition}
