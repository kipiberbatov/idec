\begin{notation}
  Let
    $d \in \N$,
    $K$ be a flat mesh of dimension $d$,
    $c_0 \in C_0(\partial K)$ with corresponding point $x \in \R^d$.
  By ${\bf n}_{c_0}$ we will denote the exterior unit normal at $x$ to $K$.
  When $c_{0}$ has more than one non-parallel adjacent hyperfaces
  (for instance, in $3$D, it can lie on an edge or at a corner), we will take
  some average of the normals to those faces.
  The easiest one is to sum all exterior unit normals and divide by the length
  of the sum.
  This is the approach taken in the software implementation.

  When $\Gamma \subseteq (\partial K)_0$, we will understand ${\bf n}$ as
  a function from $\Gamma$ to $\R^d$ or as a linear map in
  $\Hom({\rm Free}_\R(\Gamma), \R^d)$.
\end{notation}
