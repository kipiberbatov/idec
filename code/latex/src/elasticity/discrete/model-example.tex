\begin{example}
  Consider the problem of a twist of a cylindrical bar described
  in Section 9.1 of
  (\href
    {https://www.sciencedirect.com/science/article/pii/S0020768311001727}
    {Hadjesfandiari 2011}).
  Let
    $\theta$ be the constant angle of twist per unit length,
    $\lambda$ and $\mu$ be the Lame parameters.
  Let
    ${\bf u}$ be the displacement vector,
    $\boldsymbol{\epsilon}$ be the strain tensor,
    $\boldsymbol{\omega}$ be the rotation tensor,
    $\boldsymbol{\sigma}$ be the stress tensor.
  Then at any point $x = (x_1, x_2, x_3)$ we have:
  \begin{equation}
    {\bf u} =
    \begin{pmatrix}
      -\theta x_2 x_3 \\
      \theta x_1 x_3 \\
      0
    \end{pmatrix},\
    \boldsymbol{\epsilon} =
      \frac{\theta}{2}
      \begin{pmatrix}
        0 & 0 & - x_2  \\
        0 & 0 & x_1 \\
        -x_2 & x_1 & 0
      \end{pmatrix},\
    \boldsymbol{\omega} =
      \frac{\theta}{2}
      \begin{pmatrix}
        0 & -2 x_3 & - x_2  \\
        2 x_3 & 0 & x_1 \\
        x_2 & -x_1 & 0
      \end{pmatrix},\
    \boldsymbol{\sigma} =
      \mu \theta
      \begin{pmatrix}
        0 & 0 & - x_2  \\
        0 & 0 & x_1 \\
        -x_2 & x_1 & 0
      \end{pmatrix}.
  \end{equation}
  Note that the skew-symmetric matrix $\boldsymbol{\omega}$ corresponds to the
  vector
  \begin{equation}
    \label{cmc/discrete_elasticity/example_9_1:equation:rotation_vector}
    \boldsymbol{\omega} \mapsto \frac{\theta}{2} (-x_1, -x_2, 2 x_3)^T.
  \end{equation}
  Let $h \in \R^+$ and consider a $3$D regular grid $K$ of size $h$.
  For integers $i, j, k$, nodes in $K$ have coordinates
  \begin{equation}
    {\bf x}_{(i, j, k)} := (i h, j h, k h).
  \end{equation}
  Nodes in $K$ will be denoted by $\mathcal{N}_{(i, j, k)}$.

  There are three type of edges in $K$ (parallel to the $3$ axes)
  constructed as follows.
  Let $p \in \{1, 2, 3\}$ and $e_p$ be the $p$-th unit vector.
  Denote by $\mathcal{E}^{(p)}_{(i, j, k)}$ the edge starting at
  $\mathcal{N}_{(i, j, k)}$ and ending at $\mathcal{N}_{(i, j, k) + e_p}$.
  In particular, the oriented boundary of $\mathcal{E}^{(1)}_{(i, j, k)}$ is
  \begin{equation}
    \partial_1 \mathcal{E}^{(1)}_{(i, j, k)} =
    - \mathcal{N}_{(i, j, k)} + \mathcal{N}_{(i + 1, j, k)}.
  \end{equation}
  Similar computation holds for $p = 2$ and $p = 3$.

  For $p, q \in \{1, 2, 3\},\ p < q$ faces in $K$ are denoted by
  $\mathcal{F}^{(p, q)}_{(i, j, k)}$ and represent squares starting at
  $\mathcal{N}_{(i, j, k)}$ with basis vectors going in directions
  $e_p$ and $e_q$.
  Also use the identification of cochains
  \begin{equation}
    \mathcal{F}^{(q, p)}_{(i, j, k)} := -\mathcal{F}^{(p, q)}_{(i, j, k)}.
  \end{equation}
  For instance, the oriented boundary of $\mathcal{F}^{(1, 2}_{(i, j, k)}$ is
  \begin{equation}
    \partial_2 \mathcal{F}^{(1, 2)}_{(i, j, k)} =
    - \mathcal{E}^{(1)}_{(i, j + 1, k)}
    + \mathcal{E}^{(1)}_{(i, j, k)}
    + \mathcal{E}^{(2)}_{(i + 1, j, k)}
    - \mathcal{E}^{(2)}_{(i, j, k)}.
  \end{equation}
  We will work with the approximation $\eta^1 := \underline{u}$.

  Let $p \in \{1, 2, 3\}$.
  Then for $\epsilon^0 := \delta_1^\star \eta^1$, using the fact that
  \begin{equation}
    \eta^1 \mathcal{E}^{(p)}_{(i, j, k) + e_p} =
    \eta^1 \mathcal{E}^{(p)}_{(i, j, k)},
  \end{equation}
  we calculate:
  \begin{equation}
    \epsilon^0 \mathcal{N}_{(i, j, k)}
    =
      \frac{1}{h^2}
      \sum_{p = 1}^3 (
          \eta^1 \mathcal{E}^{(p)}_{(i, j, k)}
        - \eta^1 \mathcal{E}^{(p)}_{(i, j, k) + e_p}
      )
    = 0.
  \end{equation}
  Hence,
  \begin{equation}
    \tau^0 = \lambda \epsilon^0 = 0.
  \end{equation}
  Using the computation from
  \Cref{cmc/vector_field_to_1_cochain/1d_example:exact_value}
  in each direction, we get
  \begin{subequations}
    \begin{alignat}{2}
      & \eta^1 \mathcal{E}^{(1)}_{(i, j, k)} &&
        = \frac{h}{2} \theta (-x_2 x_3 - x_2 x_3)
        = - \theta h x_2 x_3
        = - \theta j k h^3, \\
      & \eta^1 \mathcal{E}^{(2)}_{(i, j, k)} &&
        = \frac{h}{2} \theta (x_1 x_3 + x_1 x_3)
        = \theta h x_1 x_3
        = \theta i k h^3, \\
      & \eta^1 \mathcal{E}^{(3)}_{(i, j, k)} &&
        = \frac{h}{2} \theta (0 + 0)
        = 0.
    \end{alignat}
  \end{subequations}
  For $\omega^2 := \delta_1 \eta^1$, using that
  $\epsilon(c_2) = \eta^1(\partial c_2)$, we get:
  \begin{subequations}
    \begin{alignat}{3}
      & \omega^2 \mathcal{F}^{(2, 3)}_{(i, j, k)}
      && =
        ( - \eta^1 \mathcal{E}^{(2)}_{(i, j, k + 1)}
          + \eta^1 \mathcal{E}^{(2)}_{(i, j, k)}
        )
      + ( \eta^1 \mathcal{E}^{(3)}_{(i, j + 1, k)}
          - \eta^1 \mathcal{E}^{(3)}_{(i, j, k)}
        )
      = - \theta i h^3 + 0
      && = - \theta i h^3, \\
%
      & \omega^2 \mathcal{F}^{(3, 1)}_{(i, j, k)}
      && =
        ( - \eta^1 \mathcal{E}^{(3)}_{(i + 1, j, k)}
          + \eta^1 \mathcal{E}^{(3)}_{(i, j, k)}
        )
      + ( \eta^1 \mathcal{E}^{(1)}_{(i, j, k + 1)}
          + \eta^1 \mathcal{E}^{(1)}_{(i, j, k)}
        )
      = 0 - \theta j h^3
      && = - \theta j h^3, \\
%
      & \omega^2 \mathcal{F}^{(1, 2)}_{(i, j, k)}
      && =
        ( - \eta^1 \mathcal{E}^{(1)}_{(i, j + 1, k)}
          + \eta^1 \mathcal{E}^{(1)}_{(i, j, k)}
        )
      + ( \eta^1 \mathcal{E}^{(2)}_{(i + 1, j, k)}
          - \eta^1 \mathcal{E}^{(2)}_{(i, j, k)}
        )
      = \theta k h^3 + \theta k h^3
      && = 2 \theta k h^3.
    \end{alignat}
  \end{subequations}
  We see the clear correspondence (by a factor of $2 h^2$) to the ``flattened''
  version of $\boldsymbol{\omega}$,
  \Cref{cmc/discrete_elasticity/example_9_1:equation:rotation_vector}.

  We have $\tau^2 = \mu \omega^2$ and hence
  $\delta_2^\star \tau^2 = \mu \delta_2^\star \omega^2$.
  Then
  \begin{subequations}
    \begin{alignat}{3}
      & (\delta_2^\star \tau^2) \mathcal{E}^{(1)}_{(i, j, k)}
      && =
        \frac{\mu}{h^2}
        (
          ( \omega^2 \mathcal{F}^{(1, 2)}_{(i, j, k)}
            - \omega^2 \mathcal{F}^{(1, 2)}_{(i, j - 1, k)}
          )
          -
          ( \omega^2 \mathcal{F}^{(1, 3)}_{(i, j, k)}
            - \omega^2 \mathcal{F}^{(1, 3)}_{(i, j, k - 1)}
          )
        )
      = \frac{\mu}{h^2} (0 - 0)
      && = 0, \\
%
      & (\delta_2^\star \tau^2) \mathcal{E}^{(2)}_{(i, j, k)}
      && =
        \frac{\mu}{h^2}
        (
          ( \omega^2 \mathcal{F}^{(1, 2)}_{(i - 1, j, k)}
            - \omega^2 \mathcal{F}^{(1, 2)}_{(i, j, k)}
          )
          -
          ( \omega^2 \mathcal{F}^{(2, 3)}_{(i, j, k)}
            - \omega^2 \mathcal{F}^{(2, 3)}_{(i, j, k - 1)}
          )
        )
      = \frac{\mu}{h^2} (0 - 0)
      &&  = 0, \\
%
      & (\delta_2^\star \tau^2) \mathcal{E}^{(3)}_{(i, j, k)}
      && =
        \frac{\mu}{h^2}
        (
          ( \omega^2 \mathcal{F}^{(1, 3)}_{(i - 1, j, k)}
            - \omega^2 \mathcal{F}^{(1, 3)}_{(i, j, k)}
          )
          -
          ( \omega^2 \mathcal{F}^{(2, 3)}_{(i, j - 1, k)}
            - \omega^2 \mathcal{F}^{(2, 3)}_{(i, j, k)}
          )
        )
      = \frac{\mu}{h^2} (0 - 0)
      && = 0.
    \end{alignat}
  \end{subequations}
  Hence, $\delta_2^\star \tau^2 = 0$.
  With zero body force $\mathfrak{b}^1$, we get:
  \begin{equation}
    \delta_0 \tau^0 + \delta_2^\star \tau^2 + \mathfrak{b}^1 = 0 + 0 + 0 = 0.
  \end{equation}
\end{example}
