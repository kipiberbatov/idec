\documentclass[fleqn]{article}
\usepackage[margin = 0.6in, paper = a4paper]{geometry}
\usepackage[multiple]{footmisc}
\usepackage{authblk}
\usepackage{xcolor}
\usepackage[fleqn]{amsmath}
\usepackage{amssymb}
\usepackage{amsthm}
\usepackage{bookmark}
\usepackage{hyperref}
\hypersetup{
  pdfauthor = {Kiprian Berbatov},
  pdftitle = {IDEC Documentation},
  pdfsubject = {Discrete calculus},
  pdfkeywords = {discrete, microstructure, calculus, geometry},
  pdfcreator = {pdflatex},
  pdfproducer = {Latex2e with hyperref},
  colorlinks = true,
  linkcolor = blue,
  citecolor = green,
  urlcolor = cyan,
  bookmarksnumbered = true
}
\usepackage[nameinlink]{cleveref}

\counterwithin{equation}{section}
\theoremstyle{definition}
\newtheorem{theorem}{Theorem}[section]
\newtheorem{proposition}[theorem]{Proposition}
\newtheorem{lemma}[theorem]{Lemma}
\newtheorem{corollary}[theorem]{Corollary}
\newtheorem{hypothesis}[theorem]{Hypothesis}
\newtheorem{notation}[theorem]{Notation}
\newtheorem{definition}[theorem]{Definition}
\newtheorem{discussion}[theorem]{}
\newtheorem{example}[theorem]{Example}
\newtheorem{remark}[theorem]{Remark}

\renewcommand{\thefootnote}{\arabic{footnote}}

\title{IDEC documentation}
\author{Kiprian Berbatov}
\date{June 11, 2024}

\begin{document}

\maketitle

\section{Jagged arrays}

\begin{definition}
  Let $n$ be a positive integer and $T$ be a type (e.g., integers)
  A jagged array of order $n$ is defined recursively as follows:
  \begin{enumerate}
    \item
      jagged array of order $0$ is an object of type $T$;
    \item
      for $n > 0$, jagged array of order $n$ is a finite (possibly empty)
      sequence of jagged arrays of order $n - 1$.
  \end{enumerate}
\end{definition}

\begin{example}
  \begin{itemize}
    \item
      $()$ are $(0, 1, 2)$ jagged arrays of order $1$ (ordinary arrays);
    \item
      $((0, 1, 2), (3), ())$ is a jagged array of order $2$
      (note that its last elements is the empty array of order $1$);
    \item
      $(((0, 1, 2), (3)),())$ is a jagged array of order $3$
      (note that its last elements is the empty array of order $2$);
  \end{itemize}
\end{example}

\begin{definition}
  Let $n > 0$.
  A jagged array of order $n$ is represented in memory by a structure of $n + 1$
  fields as follows: an integer $a_0$, $n - 1$ pointers to (arrays of) integers
  $a_1, ..., a_{n - 1}$, a pointer $a_n$ to an array of type $T$.
  
  The fields $a_0, ..., a_{n - 1}$ represent the internal structure,
  while $a_n$ holds the values.
  More precisely, let $x$ be a jagged array of order $n$.
  $a_0$ is the number of its elements of order $n - 1$.
  For $i = 0, ..., a_0 - 1$,
  $a_1[i]$ represents the number of elements of order $n - 2$ in $x[i]$.
  There are totally $b_2 = a_1[0] + ... a_1[a_0 - 1]$ elements of order $n - 2$.
  For $i = 0, ..., a_0 - 1$, $j  = 0, ..., a_1[i] - 1$,
  $a_2[a_1[0] + ... + a_1[i - 1] + j]$
  is the number of its elements of order $n - 3$ in $x[i][j]$.
  Continue in the same manner, $a[n]$ represents the flattened version of $x$.
\end{definition}

\begin{example}
  Consider the jagged array of order $3$
  \begin{equation}
    x = (((0, 1, 2), (3, 4)), (), (()), ((5, 6), ())).
  \end{equation}
  Then $x$ has the following representation:
  \begin{equation}
    \begin{split}
      & x.a0 = 4 \\
      & x.a1 = (2, 0, 1, 2) \\
      & x.a2 = (3, 2, 0, 2, 0) \\
      & x.a3 = (0, 1, 2, 3, 4 , 5, 6)
    \end{split}
  \end{equation}
\end{example}

\end{document}
