\documentclass[fleqn]{article}
\usepackage[margin = 0.6in, paper = a4paper]{geometry}
\usepackage[multiple]{footmisc}
\usepackage{authblk}
\usepackage{xcolor}
\usepackage[fleqn]{amsmath}
\usepackage{amssymb}
\usepackage{amsthm}
\usepackage{bookmark}
\usepackage{hyperref}
\hypersetup{
  pdfauthor = {Kiprian Berbatov},
  pdftitle = {IDEC Documentation},
  pdfsubject = {Discrete calculus},
  pdfkeywords = {discrete, microstructure, calculus, geometry},
  pdfcreator = {pdflatex},
  pdfproducer = {Latex2e with hyperref},
  colorlinks = true,
  linkcolor = blue,
  citecolor = green,
  urlcolor = cyan,
  bookmarksnumbered = true
}
\usepackage[nameinlink]{cleveref}

\counterwithin{equation}{section}
\theoremstyle{definition}
\newtheorem{theorem}{Theorem}[section]
\newtheorem{proposition}[theorem]{Proposition}
\newtheorem{lemma}[theorem]{Lemma}
\newtheorem{corollary}[theorem]{Corollary}
\newtheorem{hypothesis}[theorem]{Hypothesis}
\newtheorem{notation}[theorem]{Notation}
\newtheorem{definition}[theorem]{Definition}
\newtheorem{discussion}[theorem]{}
\newtheorem{example}[theorem]{Example}
\newtheorem{remark}[theorem]{Remark}

\renewcommand{\thefootnote}{\arabic{footnote}}

\title{IDEC documentation}
\author{Kiprian Berbatov}
\date{June 11, 2024}

\begin{document}

\maketitle

\section{Jagged arrays}

\begin{definition}
  Let $n$ be a positive integer and $T$ be a type (e.g., integers)
  A \textbf{jagged} array of order $n$ is defined recursively as follows:
  \begin{enumerate}
    \item
      jagged array of order $0$ is an object of type $T$;
    \item
      for $n > 0$, jagged array of order $n$ is a finite (possibly empty)
      sequence of jagged arrays of order $n - 1$.
  \end{enumerate}
\end{definition}

\begin{example}
  \begin{itemize}
    \item
      $()$ are $(0, 1, 2)$ jagged arrays of order $1$ (ordinary arrays);
    \item
      $((0, 1, 2), (3), ())$ is a jagged array of order $2$
      (note that its last element is the empty array of order $1$);
    \item
      $(((0, 1, 2), (3)),())$ is a jagged array of order $3$
      (note that its last element is the empty array of order $2$);
  \end{itemize}
\end{example}

\begin{definition}
  Let $n > 0$.
  A jagged array of order $n$ is represented in memory by a structure of $n + 1$
  fields as follows:
  an integer $a_0$,
  $n - 1$ pointers $a_1, ..., a_{n - 1}$ to integer
  (any of them points to the  $0$-th element of an array of integers),
  a pointer $a_n$ to the $0$-th element of an array of type $T$.
  
  The fields $a_0, ..., a_{n - 1}$ represent the internal structure,
  while $a_n$ holds the values.
  More precisely, let $x$ be a jagged array of order $n$.
  $a_0$ is the number of its elements of order $n - 1$.
  For $i = 0, ..., a_0 - 1$,
  $a_1[i]$ represents the number of elements of order $n - 2$ in $x[i]$.
  There are totally $b_2 = a_1[0] + ... a_1[a_0 - 1]$ elements of order $n - 2$.
  For $i = 0, ..., a_0 - 1$, $j  = 0, ..., a_1[i] - 1$,
  $a_2[a_1[0] + ... + a_1[i - 1] + j]$
  is the number of its elements of order $n - 3$ in $x[i][j]$.
  Continue in the same manner for higher orders.
  Finally, $a_n$ represents the flattened version of $x$.
\end{definition}

\begin{example}
  Consider the jagged array of order $3$
  \begin{equation}
    x = (((0, 1, 2), (3, 4)), (), (()), ((5, 6), ())).
  \end{equation}
  Then $x$ has the following representation:
  \begin{equation}
    \begin{split}
      & x.a0 = 4 \\
      & x.a1 = (2, 0, 1, 2) \\
      & x.a2 = (3, 2, 0, 2, 0) \\
      & x.a3 = (0, 1, 2, 3, 4 , 5, 6)
    \end{split}
  \end{equation}
\end{example}

\section{Meshes}

\begin{definition}
  Let $d$ be a natural number.
  A \textbf{mesh} of dimension $d$ is a finite set of polytopes of dimension at
  most $d$ such that:
  \begin{itemize}
    \item
      if $X$ is an element of $M$, then all subfaces of $X$ are also in $M$;
    \item
      the intersection of elements of $M$ is a finite (possibly empty) union of
      elements of $M$.
  \end{itemize}
  For an integer $p \in [0, d]$, the set of elements (polytopes) of dimension
  $p$ in $M$ is denoted by $M_p$.
\end{definition}

\begin{definition}
  A mesh is represented in memory by its topology and geometry.
  The topology is represented using:
  \begin{itemize}
    \item
      integer \emph{dim} ($d$) storing mesh's dimension;
    \item
      array \emph{cn} storing the number of $p$-cells for $p = 0, ..., d$; 
    \item
      a jagged array \emph{cf} of order $4$ storing its topology.
      More precisely, if $0 < p \leq d$, $0 \leq q < p$,
      $0 \leq i < {\rm cn}[p]$, then ${\rm cf}[p][q][i]$ stores the indices of
      all $q$-dimensional subfaces of the $i$-th $p$-cell.
  \end{itemize}
  For a mesh of flat polytopes, the topology is represented by:
  \begin{itemize}
    \item
      integer \emph{dim\_embedded} ($d$) storing mesh's embedding dimension;
    \item
      array of floating point numbers \emph{coord} of length
      ${\rm cn}[0] * {\rm dim\_embedded}$ storing coordinates of the nodes
      (${\rm cn}[0]$ nodes of dimension $\mathbb{R}^{{\rm cn}[0]}$).
  \end{itemize}
\end{definition}

\begin{example}
  Consider two triangles dividing a square.
  Representation of the topology: ${\rm dim} = 2$, ${\rm cn} = \{4, 5, 2\}$,
  \begin{equation}
    {\rm cf} =
    (
      (
        ((0, 1), (1, 2), (2, 3), (3, 0), (0, 2)),
      ),
      (
        ((0, 1, 2), (0, 1, 3)),
        ((0, 1, 4), (2, 3, 4))
      )
    ).
  \end{equation}
  Representation of the geometry: ${\rm dim\_embedded} = 2$,
  ${\rm coord} = (-1, -1, 1, -1, 1, 1, 1, -1)$
  (the latter represents the nodes with coordinates
  $(-1, -1), (1, -1), (1, 1), (1, -1)$ respectively).
\end{example}

\section{Combinatorial differential forms and Forman decomposition}

\begin{definition}
  Let $d \in \mathbb{N}$, $M$ be a mesh of dimension $d$, $0 \leq p_f \leq d$.
  A \textbf{combinatorial differential form} of dimension $p_f$-form on $M$
  is a linear map
  \begin{equation}
    \omega \colon C_\bullet M \to C_\bullet M
  \end{equation}
  such that for any $p \in [p_f, d]$, $c_p \in M_p$, $\omega(c_p)$ is a linear
  combination of the $(p - p_f)$-subfaces of $M$.

  The space of all $p_f$-forms on $M$ is denoted by $\Omega^{p_f} M$.
  The space of all combinatorial differential forms is the direct sum
  \begin{equation}
    \Omega^\bullet M := \bigoplus_{p_f = 0}^{d} \Omega^{p_f} M.
  \end{equation}
\end{definition}

\begin{definition}
  Let $d \in \mathbb{N}$, $M$ be a mesh of dimension $d$ with boundary operator
  $\partial$.
  The \textbf{discrete differential} on $M$ is the linear map
  \begin{equation}
    D \colon \Omega^\bullet \to \Omega^\bullet
  \end{equation}
  which maps a $p$-form $\omega$ to a $(p + 1)$-form by the formula
  \begin{equation}
    D \omega := \omega \circ \partial - (-1)^p \partial \circ \omega.
  \end{equation}
\end{definition}

\begin{proposition}
  Let $d \in \mathbb{N}$, $M$ be a mesh of dimension $d$ with boundary operator
  $\partial$, $D$ is the discrete differential on $M$.
  Then $(\Omega^\bullet M, D)$ is a cochain complex, i.e.,
  \begin{equation}
    D^2 = 0.
  \end{equation}
\end{proposition}

\begin{proof}
  A straightforward computation using the fact that $\partial^2 = 0$.
  Indeed, let $\omega \in \Omega^p M$.
  Then
  \begin{equation}
    D^2(\omega)
    = D(D \omega)
    = D(\omega \circ \partial - (-1)^p \partial \circ \omega)
    = \omega \circ \partial \circ \partial
      - (-1)^{p + 1} \partial \circ \omega \circ \partial
      - (-1)^p \partial \circ \omega \circ \partial
      - (-1)^p (-1)^{p + 1} \partial \circ \partial \circ \omega
    = 0.
  \end{equation}
\end{proof}

\begin{definition}
  Let $d \in \mathbb{N}$, $M$ be a mesh of dimension $d$.
  Consider a mesh $K$ constructed as follows.
  The nodes of $K$ are the centroids of the cells of $M$.
  (In general, the topology of $K$ can always be constructed while the geometry
  is tricky.
  For simplicity we may assume that all the cells of $M$ are convex,
  although for non-simplicial or non-brick meshes in dimensions $3$ and above
  the resulting mesh may contain non-flat polytopes.)

  For $p_f \in [0, d]$, a $p_f$-cell of $K$ is constructed as follows.
  Let $p \in [p_f, d]$, $s = p - p_f$ consider two cells
  \begin{equation}
    c(p, i) \succeq c(s, l)
  \end{equation}
  Then a $p_f$-cell is such a pair $(c(p, i), c(s, l))$.
  If $q_f \in [0, p_f]$, $q \in [q_f, d]$, $r = q - q_f$,
  \begin{equation}
    c(q, j) \succeq c(r, k),
  \end{equation}
  then $(c(q, j), c(r, k))$ is a subface of $(c(p, i), c(s, l))$ if
  \begin{equation}
    c(p, i) \succeq c(q, j) \succeq c(r, k) \succeq c(s, l).
  \end{equation}
  The constructed space is a mesh which we call the \textbf{Forman subdivision}.
\end{definition}

\begin{definition}
  Let $d \in \mathbb{N}$, $M$ be a mesh of dimension $d$, $K$ be its Forman
  subdivision.
  Let $\varepsilon_M$ be the relative orientation on $M$.
  We construct the relative orientation $\varepsilon_K$ as follows.
  Let $p_f \in [1, d]$ $p \in [p_f, d]$, $s = p - p_f$ and we have the
  $p_f$-cell $c_K(p_f, i_f) = (c(p, i), c(s, l))$ on $K$.
  Its hypefaces come in two forms.
  \begin{enumerate}
    \item
      $c_K(p_f - 1, j_f) = (c(p - 1, j), c(s, l))$, where
      $c(p, i) \succ c(p - 1, j) \succeq c(s, l)$.
      In this case
      \begin{equation}
        \varepsilon_K(c_K(p_f, i_f), c_K(p_f - 1, j_f))
        = \varepsilon_M(c_M(p, i), c_M(p - 1, j))
      \end{equation}
    \item
      $c_K(p_f - 1, j_f) = (c(p, i), c(s + 1, k))$, where
      $c(p, i) \succeq c(s + 1, k) \succ c(s, l)$.
      In this case
      \begin{equation}
        \varepsilon_K(c_K(p_f, i_f), c_K(p_f - 1, j_f))
        = (-1)^{p_f} \varepsilon_M(c_M(s + 1, k), c_M(s, l)).
      \end{equation}
  \end{enumerate}
\end{definition}

\begin{theorem}
  Let $d \in \mathbb{N}$, $M$ be a mesh of dimension $d$,
  $\varepsilon_M$ be a relative orientation on $M$ with corresponding boundary
  operator $\partial_M$ and discrete differential $D_M$.
  Let $K$ be the Forman subdivision of $M$,
  $\varepsilon_K$ be the orientation on $K$ constructed above,
  $d_K$ be the corresponding coboundary operator on $K$.
  Then
  \begin{equation}
    (\Omega^p M, D_M) \cong (C^p K, d_K).
  \end{equation}
  with the isomorphism being the mapping of the basis forms to basis cochains
  introduced in the construction of $K$.
\end{theorem}

\section{Approximating with vector fields with 1-cochains}

\begin{definition}
  Let $m, n \in \mathbb{N}$, $A$ be a real $m \times n$ matrix.
  An $n \times m$ matrix $B$ is called \textbf{Moore-Penrose inverse} or
  \textbf{pseudo-inverse} if
  \begin{subequations}
    \begin{align}
      A B A & = A, \\
      B A B & = B, \\
      (A B)^T & = A B, \\
      (B A)^T & = B A.
    \end{align}
  \end{subequations}
\end{definition}

\begin{theorem}
  Let $m, n \in \mathbb{N}$, $A$ be a real $m \times n$ matrix.
  Then $A$ has a unique Moore-Penrose inverse, denoted by $A^*$.
\end{theorem}

\begin{remark}
  If the matrix $A$ is of full rank there exists a closed formula for $A^*$.
  \begin{enumerate}
    \item
      If $A$ is a square matrix with full rank, i.e., an invertible one, then
      $A^* = A^{-1}$.
    \item
      If $m > n$ and $A$ is an $m \times n$ matrix of full rank, then its
      columns are linearly independent which means that $A^T A$ is
      symmetric and positive definite and hence invertible.
      (Its inverse $(A^T A)^{-1}$ is also symmetric and positive definite.)
      It is then easy to check that
      \begin{equation}
        B := (A^T A)^{-1} A^T
      \end{equation}
      is the Moore-Penrose inverse of $A$.
      Indeed, since obviously $B$ is a left inverse of $A$, and 
      \begin{subequations}
        \begin{align}
          & A B A && = A (B A) = A I_n = A, \\
          & B A B && = (B A) B = I_n B = B, \\
          & (A B)^T && = (A (A^T A)^{-1} A^T)^T
            = (A^T)^T ((A^T A)^{-1})^T A^T = A (A^T A)^{-1} A^T = A B, \\
          & (B A)^T && = I_n^T = I_n = B A.
        \end{align}
      \end{subequations}
    \item
      If $m < n$ and $A$ is an $m \times n$ matrix of full rank, then an
      analogous reasoning to the previous point shows that
      \begin{equation}
        A^* = A^T (A A^T)^{-1}.
      \end{equation}
  \end{enumerate}
\end{remark}

\begin{definition}
  Let $d\in \mathbb{N}$ and $M$ be a mesh of dimension $d$.
  We say that $M$ is \textbf{flat} if the following conditions are satisfied:
  \begin{enumerate}
    \item
      it is pure, i.e., all of its cells lie within some $d$-cell;
    \item
      all its cells are flat (but possibly degenerate) polytopes;
    \item
      it can be embedded in $\mathbb{R}^d$.
  \end{enumerate}
\end{definition}

\begin{definition}
  Let:
  \begin{enumerate}
    \item
      $d \in \mathbb{N},\ d \geq 1$;
    \item
      $K$ be a flat mesh of dimension $d$
      (with a chosen embeddeding in $\mathbb{R}^d$);
    \item
      $\epsilon$ be a relative orientation on $K$;
    \item
      $\mathcal{N}_i$ be a node in $K$ connected to $n > 0$ edges;
    \item
      $\mathcal{E}_{j_0}, ..., \mathcal{E}_{j_{n - 1}}$ be all the edges
      containing $\mathcal{N}_i$ as a node;
    \item
      $\mathcal{N}_{i_0}, ..., \mathcal{N}_{i_{n - 1}}$ be the other than
      other $\mathcal{N}_i$ of
      $\mathcal{E}_{j_0}, ..., \mathcal{E}_{j_{n - 1}}$ respectively.
  \end{enumerate}
  We define the \textbf{node matrix} $\mathcal{L}_{\mathcal{N}_i}$ of
  $\mathcal{N}_i$ by
  \begin{equation}
    \mathcal{L}_{\mathcal{N}_i} :=
    \begin{pmatrix}
      x_{i_0, 0} - x_{i, 0} & \cdots & x_{i_0, d - 1} - x_{i, d - 1} \\
      \vdots & \ddots & \vdots \\
      x_{i_{n - 1}, 0} - x_{i, 0} & \cdots & x_{i_{n - 1}, d - 1} - x_{i, d - 1}
    \end{pmatrix}
    \in \mathbb{R}^{n \times d}.
  \end{equation}
\end{definition}

\begin{definition}
  Let $d \in \mathbb{N}\ \text{with}\ d \geq 1$, $K$ be a mesh of dimension $d$,
  $\pi^1 \in C^1 K$, $\mathcal{N}_i \in K_0$.
  Define the \textbf{neighbor representation} $\widehat{\pi^1}_{\mathcal{N}_i}$
  of $\pi^1$ at $\mathcal{N}_i$ by
  \begin{equation}
    \widehat{\pi^1}_{\mathcal{N}_i} :=
    ( \epsilon_K(\mathcal{E}_{j_0}, \mathcal{N}_{i_0}) \pi^1(\mathcal{E}_{j_0}),
      \cdots,
        \epsilon_K(\mathcal{E}_{j_{k - 1}}, \mathcal{N}_{i_{k - 1}})
        \pi^1(\mathcal{E}_{j_{k - 1}})
    ) \in \mathbb{R}^n.
  \end{equation}
  Define the \textbf{1-cochain embedding} ${\overline{\pi^1}}_{\mathcal{N}_i}$
  of $\pi^1$ at $\mathcal{N}_i$ by
  \begin{equation}
    {\overline{\pi^1}}_{\mathcal{N}_i} :=
    \left(\mathcal{L}_{\mathcal{N}_i}\right)^* \widehat{\pi^1}_{\mathcal{N}_i}
    \in \mathbb{R}^d.
  \end{equation}
\end{definition}

\begin{example}
  Let $h \in \mathbb{R}^+$ and $K$ be a regular subdivision with size $h$
  of some interval,
  all the edges in $K$ are oriented from left to right,
  $\pi^1 \in C^1 K$.
  \begin{enumerate}
    \item
      Consider an interior point $\mathcal{N}_i$ with neighboring edges
      $\mathcal{E}_{i - 1}$ and $\mathcal{E}_i$ and corresponing neighboring
      nodes $\mathcal{N}_{i - 1}$ and $\mathcal{N}_{i + 1}$.
      Then
      \begin{equation}
        \mathcal{L}_{\mathcal{N}_i} =
        \begin{pmatrix}
          -h \\
          h
        \end{pmatrix}
        \Rightarrow
        (\mathcal{L}_{\mathcal{N}_i})^* =
          \frac{1}{2 h}
          \begin{pmatrix}
            -1 & 1
          \end{pmatrix}
      \end{equation}
      and
      \begin{equation}
        \widehat{\pi^1}_{\mathcal{N}_i} =
        \begin{pmatrix}
          - \pi^1 \mathcal{E}_{i - 1} \\
          \pi^1 \mathcal{E}_{i + 1}
        \end{pmatrix}
        \Rightarrow
        {\overline{\pi^1}}_{\mathcal{N}_i} =
          \frac{1}{2 h}
          \begin{pmatrix}
            -1 & 1
          \end{pmatrix}
          \begin{pmatrix}
            - \pi^1 \mathcal{E}_{i - 1} \\
            \pi^1 \mathcal{E}_{i + 1}
          \end{pmatrix}
        = \frac{1}{2 h}
          \left( \pi^1 \mathcal{E}_{i - 1} + \pi^1 \mathcal{E}_{i + 1} \right).
      \end{equation}
  \end{enumerate}
\end{example}

\begin{definition}
  Let $K$ be an embedded flat mesh and $u$ be a vector field on $|K|$.
  Define the approximation $\underline{u} \in C^1 K$ as follows.
  Let $\mathcal{E}_i$ be an edge with endpoints $\mathcal{N}_{i_0}$ and
  $\mathcal{N}_{i_1}$, oriented from $\mathcal{N}_{i_0}$ to $\mathcal{N}_{i_1}$.
  Then
  \begin{equation}
    \underline{u}_{\mathcal{E}_i} := 
      \frac{1}{2}
      ( \mathcal{L}_{\mathcal{N}_{i_0}} u_{\mathcal{N}_{i_0}} 
        + \mathcal{L}_{\mathcal{N}_{i_1}} u_{\mathcal{N}_{i_1}} 
      ).
  \end{equation}
\end{definition}

\end{document}
